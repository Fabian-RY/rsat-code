\chapter{Introduction}

Since a few years, large scale biological studies produced huge
amounts of data about networks of molecular interactions (protein
interactions, gene regulation, metabolic reactions, signal
transduction). The integration of these data sets can be combined to
acquire a global view of the pieces that, altogether, contribute to
the complexity of biological processes. High-throughput data is
however notoriously noisy and incomplete, and it is important to
evaluate the quality of the different pieces of information that are
taken in consideration for building higher views of biological
networks.

An important effort will be required to extract reliable information
from the ever-increasing ocean of high-throughput data. This will
require the utilization of powerful tools that enable us to apply
statistical analysis on large graphs. For this purpose, we developed
the \textbf{Network Analysis Tools} (\neat), as set of tools performing
basic operations on networks and clusters. 

The tools can be used in three ways:

\begin{enumerate}

\item \textbf{Web server interface}

  \url{http://rsat.ulb.ac.be/neat/}
  
  The Web interface gives a convenient and intuitive access to the
  tools, and allows you to bring your data sets through some typical
  analysis work flows in order to extract the best of it.
  
\item \textbf{Stand-alone application}
  
  \url{http://rsat.ulb.ac.be/rsat/distrib/}
  
  Most of the tools are freely available to academic users, according
  to a licence for non-commercial and non-military usage.

  The license covers both the Regulatory Sequence Analysis Tools
  (\RSAT) and the Network Analysis Tools (\neat). It can be downloaded
  from the RSAT Web site.

\item \textbf{Web services}

  
  In addition, people having computer skills can also use be same
  tools via a Web services interface, in order to integrate them in
  automatic work-flows. To obtain information on the Web services,
  connect the \neat web server, and in the left menu, select
  \textbf{Information - Web services}.

\end{enumerate}

