\documentclass{book}
%\documentstyle[makeidx]{book}
\makeindex
%\usepackage{color}
\usepackage[usenames]{color}
\usepackage{times}
\usepackage{graphics}
\usepackage{latexsym}
\usepackage{makeidx}


%%%%%%%%%%%%%%%%%%%%%%%%%%%%%%%%%%%%%%%%%%%%%%%%%%%%%%%%%%%%%%%%
%%%%%%%%%%%%%%%%%%%%%%%%%%% commands %%%%%%%%%%%%%%%%%%%%%%%%%%%
\newcommand{\tbw}{\textbf{TO BE WRITTEN}}
\newcommand{\RSAT}{\textbf{\textit{RSAT}}}
\newcommand{\file}[1]{\textit{#1}}
\newcommand{\concept}[1]{\index{#1}\textsl{#1}}
\newcommand{\command}[1]{\begin{footnotesize}\begin{quote}\textcolor{Blue}{\texttt{#1}}\end{quote}\end{footnotesize}}
\newcommand{\result}[1]{\begin{footnotesize}\begin{quote}\textcolor{OliveGreen}{\texttt{#1}}\end{quote}\end{footnotesize}}
\newcommand{\program}[1]{\textbf{\textsl{#1}}}
\newcommand{\option}[1]{\texttt{#1}}
\newcommand{\email}[1]{\textit{#1}}

\newcommand{\address}[1]{\small{#1}}
\newcommand{\org}[1]{\textit{#1}}
\newcommand{\gene}[1]{\textit{#1}}
\newcommand{\seq}[1]{\texttt{#1}}

\newcommand{\url}[1]{\textit{#1}}
\newcommand{\urlref}[1]{\footnote{\textit{#1}}}

\newcommand{\scmbb}{
	Service de Conformation des Macromol\'{e}cules Biologiques et de Bioinformatique, \\
	Universit\'{e} Libre de Bruxelles, \\
	Campus Plaine, CP 263, Boulevard du Triomphe, B-1050 Bruxelles, Belgium. \\
	Tel: +32 2 650 2013 - Fax: +32 2 650 5425
}

%%%%%%%%%%%%%%%%%%%%%%%%%%%%%%%%%%%%%%%%%%%%%%%%%%%%%%%%%%%%%%%%
%%%%%%%%%%%%%%%%%%%%%%%%% environments %%%%%%%%%%%%%%%%%%%%%%%%%

\def\ex@mple{example}
\def\ex@rcise{exercise}
\def\theoremfont{\ifx\@currenvir\ex@mple
	\def\@thmfont{\rm} \else
	\ifx\@currenvir\ex@rcise
	\def\@thmfont{\rm} \else
	\def\@thmfont{\it}\fi\fi}
\newtheorem{exercise}{Exercise}[chapter]
\newtheorem{example}{Example}[chapter]




\begin{document}

\title{Network Analysis Tools \\
Web server installation}

\author{
	Sylvain Broh\'ee \\
	\email{sbrohee@ulb.ac.be} \\
        \and \\
	Karoline Faust \\
	\email{kfaust@ulb.ac.be} \\
        \and \\
	Jacques van Helden \\
	\email{jvhelden@ulb.ac.be}\\
        \\
        \\
        \bigre
}


\maketitle

\newpage
\tableofcontents
\newpage

\section{Description}

This documents describes the installation procedure for the web server
of the \textbf{Network Analysis Tools} (\neat).

It assumes that you already installed the perl scripts and the
genomes, as described in the \RSAT installation guide.

%%%%%%%%%%%%%%%%%%%%%%%%%%%%%%%%%%%%%%%%%%%%%%%%%%%%%%%%%%%%%%%%
% Web server installation

\section{Installing a local web server}

As the Regulatory Sequence Analysis Tools, \neat includes a web server, which
offers a user-friendly interface for biologists. The main server is
available for academic users at \url{http://rsat.ulb.ac.be/neat/}. A
few additional mirrors have been installed in different countries.

\subsection{Web server pages}

The web pages are located in the directory
\file{rsa-tools/public\_html}. This directory contains both the HTML help
pages, and the PHP and CGI scripts.

\subsection{Apache modules}

The \neat interface mainly relies on PHP and CGI (only for the roc-stats tool). 
These modules should be installed on the web server, and activated in the Apache configuration files. 
The installation and configuration of CGI is described in manual of the web server of \RSAT.

\subsubsection{PHP module for Mac OSX}

If your server is running under Mac OSX, you need to install a recent
version (at least v5) of the php module, which can be found at the following site. 
\url{href=http://www.entropy.ch/software/macosx/php/}
%Karoline, if you find something to add as a MAC specialist, you're welcome!

\subsubsection{PHP module for LINUX}

Generally, PHP5 is included with the Linux distribution or can easyly be installed using the installer (YAST, YUM, etc). Take care that the PHP5 Module for Apache 2.0 (apache2-mod_php5) is installed. 

PHP5 can also be installed manually from the PHP website (\textit{http://www.php.net/downloads.php}).

