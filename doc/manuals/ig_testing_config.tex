\chapter{Testing the command-line tools}

\section{Testing the access to the programs}

\subsection{Perl scripts}

From now on, you should be able to use the perl scripts from the
command line. To test this, run:

\begin{lstlisting}
random-seq -help
\end{lstlisting}


This should display the on-line help for the random sequence
generator.

\begin{lstlisting}
random-seq -l 200 -n 4
\end{lstlisting}

Should generate a random sequence of 200 nucleotides.


\subsection{Testing Perl graphical librairies}

\RSAT includes some graphical tools (\program{feature-map} and
\program{XYgraph}), which require a proper installation of Perl
modules.

\begin{description}
\item[GD.pm] Interface to Gd Graphics Library.
\item[PostScript::Simple]  Produce PostScript files from Perl.
\end{description}

To test if these modules are available on your machine, type.

\begin{lstlisting}
feature-map -help
\end{lstlisting}

If the modules are available, you should see the help message of the
program feature-map. If not, you will see an error message complaining
about the missing librairies. In such a case, ask your system
administrator to install the missing modules.

\subsection{Python scripts}

The \RSAT distribution includes some Python scripts. To test if they
are running correctly, you can try the proram \program{random-motif}.

\begin{lstlisting}
random-motif  -l 10 -c 0.85 -n 3
\end{lstlisting}

This command will generate 3 position-specific scoring matrix (PSSM)
of 10-columns with 85\% conservation of one residue in each column.

\subsection{C programs}

You can test the correct installation of the C programs with the
following command.

\begin{lstlisting}random-seq -l 1000 -n 10 | count-words -l 2 -v 1 -2str -i /dev/stdin
\end{lstlisting}

The first program (\program{random-seq}) is a Perl script, which
generates a random sequence. The output is directly piped to the C
program \program{count-words}, which computes the frequencies and
occurrences of each dinucleotide.

\section{Testing genome installation}

We will now test if the genomes are correctly installed. You can
obtain the list of supported organisms with the command:

\begin{lstlisting}
supported-organisms
\end{lstlisting}


If this command returns no result, it means that genomes were either
not installed, or not correctly configured. In such a case, check the
directories in the \file{data/genomes} directory, and check that the
file \file{data/supported\_organisms.pl}.

Once you can obtain the list of installed organisms, try to retrieve
some upstream sequences. You can first read the list of options for the
\program{retrieve-seq} program.

\begin{lstlisting}
retrieve-seq -help
\end{lstlisting}


Select an organism (say \org{Saccharomyces cerevisiae}), and
retrieve all the start codons with the following options :

\begin{lstlisting}
retrieve-seq -org Saccharomyces_cerevisiae \
        -type upstream -from 0 -to +2 -all \
        -format wc -nocomment
\end{lstlisting}


This should return a set of 3 bp sequences, mostly ATG (in the case of
\org{Saccharomyces cerevisiae} at least)

