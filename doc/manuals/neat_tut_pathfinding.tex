\chapter{Path finding}

\section{Introduction}
% what is k shortest path finding
Given a biological network and two nodes of interest, the aim of k shortest path finding is
to enumerate the requested number of shortest paths connecting these nodes ordered according to their weight.
For instance, we might look for all shortest paths between a receptor and a DNA binding protein to predict a signal
transduction pathway from a protein protein interaction network. Another example is the prediction of a metabolic pathway
given two reactions or compounds of interest and a metabolic network.\\
% ubiquitous nodes
A problem encountered in many biological networks is the presence of so-called hub nodes, that is nodes with
a large number of connections. For example, in bacterial protein-protein interaction networks,
CRP has the role of a hub node because it interacts with many targets. Likewise,
in metabolic networks, compounds such as ADP or water are hubs, since they are generated and consumed by thousands of
reactions.\\
% weighted networks
The shortest path very likely traverses the hub nodes of a network. It depends on the biological context, whether this
behaviour is desired or not. In metabolic networks, we are less interested in paths going through water or ADP, since
those paths are often not biological relevant. For instance, we can bypass the glycolysis pathway by connecting glucose
via ADP to 3-Phosphoglycerate. To avoid finding irrelevant pathways like this one,
we tested different strategies and concluded that using a weighted network gave the best results \cite{croes05},\cite{croes06}.
In a weighted network, not the shortest, but the lightest paths are searched. Hub nodes receive large weights, making them
less likely to appear in a solution path.\\
Whether weights are used and how they are set has to be decided depending on the biological network of interest.

% what will be covered in this chapter
In this chapter, we will demonstrate path finding on the example of metabolic networks. We will work on a network assembled
from all metabolic pathways annotated for the yeast S. cerevisiae in BioCyc (Release 10.6). We will also show
the influence of the weighting scheme on path finding results.

\section{Computing the k shortest paths in weighted networks}

\subsection{Study case}

% network
The yeast network constructed from BioCyc data consists of 1,185 nodes and 2,656 edges.
It has been obtained by unifying 171 metabolic pathways. Note that this network is bipartite, which means that it is made up of
two different node types: reactions and compounds. An edge never connects two nodes of the same type.

% pathway
We will recover the heme biosynthesis II pathway given its start and end compound, namely glycine and protoheme. First, we will
use the "unit" weighting scheme, which sets the same weight on all nodes. Second, we will infer the path using the "degree"
weighting scheme and compare the results.

\subsection{Protocol for the web server}

\begin{enumerate}

\item In the \neat menu, select the command \program{k shortest path finding}.

  In the right panel, you should now see a form entitled
  ``Pathfinder''.

\item Click on the button \option{DEMO}.

  The form is now filled with the BioCyc demo network, and the parameters have been
  set up to their appropriate value for the demonstration. At the top
  of the form, you can read some information about the goal of the
  demo, and the source of the data.

\item Click on the button \option{GO}.

  The computation should take a few seconds only. The result page
  shows you some statistics about the comparison (see interpretation
  below), and a link pointing to the full result file.

\item Click on the link to see the full result file.


\end{enumerate}



\subsection{Protocol for the command-line tools}

\subsection{Interpretation of the results}

\subsection{Summary}

\subsection{Strengths and Weaknesses of the approach}

\subsection{Troubleshooting}









