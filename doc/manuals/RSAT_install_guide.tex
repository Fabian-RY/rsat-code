%%%%%%%%%%%%%%%%%%%%%%%%%%%%%%%%%%%%%%%%%%%%%%%%%%%%%%%%%%%%%%%%
%
% Installation guide for regulatory Sequence Analysis Tools
%
%%%%%%%%%%%%%%%%%%%%%%%%%%%%%%%%%%%%%%%%%%%%%%%%%%%%%%%%%%%%%%%%

%\documentclass{book}
\documentclass[12pt,a4paper, oneside]{scrreprt} % KOMA-class Neukam and Kohm, scrbook alternatively
%\documentstyle[makeidx]{book}
\makeindex
%\usepackage{color}
\usepackage[usenames]{color}
\usepackage{times}
\usepackage{graphics}
\usepackage{latexsym}
\usepackage{makeidx}


%%%%%%%%%%%%%%%%%%%%%%%%%%%%%%%%%%%%%%%%%%%%%%%%%%%%%%%%%%%%%%%%
%%%%%%%%%%%%%%%%%%%%%%%%%%% commands %%%%%%%%%%%%%%%%%%%%%%%%%%%
\newcommand{\tbw}{\textbf{TO BE WRITTEN}}
\newcommand{\RSAT}{\textbf{\textit{RSAT}}}
\newcommand{\file}[1]{\textit{#1}}
\newcommand{\concept}[1]{\index{#1}\textsl{#1}}
\newcommand{\command}[1]{\begin{footnotesize}\begin{quote}\textcolor{Blue}{\texttt{#1}}\end{quote}\end{footnotesize}}
\newcommand{\result}[1]{\begin{footnotesize}\begin{quote}\textcolor{OliveGreen}{\texttt{#1}}\end{quote}\end{footnotesize}}
\newcommand{\program}[1]{\textbf{\textsl{#1}}}
\newcommand{\option}[1]{\texttt{#1}}
\newcommand{\email}[1]{\textit{#1}}

\newcommand{\address}[1]{\small{#1}}
\newcommand{\org}[1]{\textit{#1}}
\newcommand{\gene}[1]{\textit{#1}}
\newcommand{\seq}[1]{\texttt{#1}}

\newcommand{\url}[1]{\textit{#1}}
\newcommand{\urlref}[1]{\footnote{\textit{#1}}}

\newcommand{\scmbb}{
	Service de Conformation des Macromol\'{e}cules Biologiques et de Bioinformatique, \\
	Universit\'{e} Libre de Bruxelles, \\
	Campus Plaine, CP 263, Boulevard du Triomphe, B-1050 Bruxelles, Belgium. \\
	Tel: +32 2 650 2013 - Fax: +32 2 650 5425
}

%%%%%%%%%%%%%%%%%%%%%%%%%%%%%%%%%%%%%%%%%%%%%%%%%%%%%%%%%%%%%%%%
%%%%%%%%%%%%%%%%%%%%%%%%% environments %%%%%%%%%%%%%%%%%%%%%%%%%

\def\ex@mple{example}
\def\ex@rcise{exercise}
\def\theoremfont{\ifx\@currenvir\ex@mple
	\def\@thmfont{\rm} \else
	\ifx\@currenvir\ex@rcise
	\def\@thmfont{\rm} \else
	\def\@thmfont{\it}\fi\fi}
\newtheorem{exercise}{Exercise}[chapter]
\newtheorem{example}{Example}[chapter]




\begin{document}

\RSATtitlePage{Installation guide}

% \title{Regulatory Sequence Analysis Tools \\
% Installation guide}
% \author{
% 	Jacques van Helden \\
% 	\email{jvhelden@ulb.ac.be} \\
% 	\bigre
% }
% \maketitle

\newpage
\tableofcontents
\newpage

\chapter{Introduction}

This documents describes the installation procedure for the software
suite \textbf{Regulatory Sequence Analysis Tools} (\RSAT), which
integrates several dozens of tools to detect cis-regulatory elements
in DNA sequences
\cite{Thomas-Chollier:2011:W86-91,Thomas-Chollier:2008:W119-27,vanHelden:2003:3593-6,vanHelden:2000:177-87}.

\chapter{Quick installation guide}

The full installation of all the \RSAT functionalities requires some
external libraries and companion programs. Some installation steps are
tricky, because they depend on the operating system. However, the
majority of the tools does not require a full installation.

In order to give you a quick start, this section proposes a simplified
installation procedure.

The detailed installation instructions willl be presented in the next
chapters. In case of trouble, don't hesitate to contact
\href{mailto:me@example.com}{Jacques.van-Helden@univ-amu.fr}.

\section{Requirements}

Make sure that the following programs are installed on your machine.

\begin{itemize}
\item \program{tar}
\item \program{git}
\item \program{cvs}
\item \program{wget}
\item \program{perl} version 5.14 or later
\end{itemize}


\section{Basic installation steps}


\begin{enumerate}
\item \textbf{Download} the compressed tar archive from the \RSAT
  distribution site, and uncompress it.

  \begin{lstlisting}
tar -xpzf rsat_yyyymmdd.tar.gz
  \end{lstlisting}
  
  
\item Open a terminal, set your working directory to the \file{rsat}
  folder extracted from this archive, and run the \RSAT
  \textbf{configuration script}.

  \begin{lstlisting}
cd rsat; 
perl perl-scripts/configure_rsat.pl
  \end{lstlisting}

  \textbf{Tip:} at this stage, the way to specify some parameters
  might seem obscure to you (e.g. URL of your web site, cluster
  parameters, NeAT, pathway tools, paths for helper
  programs). Actually these parameters are only required for the full
  installation of a web- and cluster-enabled \RSAT instance. 

  If you don't know how to specify a parameter, just leave it to its
  default value. You will have the possibility to re-run the
  configuration script later.

\item Define RSAT environment variable and adapt your path

  \begin{lstlisting}
source RSAT_config.bashrc
  \end{lstlisting}

Check the RSAT environment variable

  \begin{lstlisting}
echo $RSAT
  \end{lstlisting}

\item Include RSAT parameters in your default bash config.

For future sessions, include the following line in your bash
configuration file (\file{\$HOME/.bashrc} or, even better,
\file{/etc/bashrc} if you have admin rights on the machine).

\begin{verbatim}
source [RSAT_FULL_PATH]/RSAT_config.bashrc
\end{verbatim}


Note: for Ubuntu distribution , the simplest solution is to create a
soft link from the directory \file{/etc/profile.d/}. This requires
admin rights.

\begin{verbatim}
sudo ln -s $RSAT/RSAT_config.bashrc /etc/profile.d/rsat.sh
\end{verbatim}

\item \textbf{Initialize \RSAT folders}

  \begin{lstlisting}
make -f makefiles/init_rsat.mk init
  \end{lstlisting}
  
\item Install the required Perl libraries. For this you need admin
  rights.

  \begin{lstlisting}
## The following commands should be executed with admin rights.
sudo bash

## Check that RSAT path has been defined
echo $RSAT

## If not, try this and check again
##   source /etc/profile.d/rsat.sh

## Set working directory to RSAT
cd $RSAT

## Get the list of Perl modules to be installed
make -f makefiles/install_rsat.mk  perl_modules_list

## Check which Perl modules are already installed
make -f makefiles/install_rsat.mk perl_modules_check
## The locations of installed modules are stored in perl_modules_check.txt

## Install these modules
make -f makefiles/install_rsat.mk perl_modules_install
  \end{lstlisting}


  \begin{itemize}
  \item \textbf{Beware: } this step requires admin rights (you will be
    prompted for sudo password in order to install modules in cpan).
    
  \item \program{cpan} installation takes some time, and is
    interrupted by many prompts to let you choose some installation
    options. We recommend to use the default option for all questions
    (just press the ``Enter'' key). If you are bored to confirm each
    installation step, you can run the following target, which will
    automatically blindly accept all the default options.

\begin{lstlisting}
    make -f makefiles/install_rsat.mk perl_modules_install_noprompt
\end{lstlisting}
    
  \item Even though some Perl modules may fail to install, don't worry
    too much. At this stage, you should be able to use most of \RSAT
    functionalities.
  \end{itemize}

\item compile \RSAT programs written in C

\begin{lstlisting}
make -f makefiles/init_rsat.mk compile_all
\end{lstlisting}

\item Install some third-party programs required by some \RSAT
  scripts.

\begin{lstlisting}
make -f makefiles/install_software.mk install_ext_apps
\end{lstlisting}


\item Install two model organisms, required for some of the Web tools.

  \begin{lstlisting}
download-organism -v 1 -org Saccharomyces_cerevisiae
download-organism -v 1 -org Escherichia_coli_K_12_substr__MG1655_uid57779
  \end{lstlisting}
  
  Optionally, you can download additional organisms with the same
  command. The list of supported-organisms can be obtained with the
  command \program{supported-organisms-server}.

\end{enumerate}

\section{Testing the basic installation}

At this stage, you should now dispose of a local installation of \RSAT
with all the basic functionalities enabled. We can now test the proper
functioning of the different types of programs.

\subsection{Testing the path}

\subsubsection{RSAT envionment variable}

\begin{lstlisting}
echo $RSAT
\end{lstlisting}
%$

\subsubsection{RSAT exec dirs in the PATH}

Check that the folders containing RSAT executables are included in
your path.

\begin{lstlisting}
echo $PATH | perl -pe 's|:|\n|g' | grep rsat
\end{lstlisting}

The result should contain the full path to the folders \file{bin},
\file{perl-scripts}, and \file{python-scripts}.

\subsubsection{Testing a \RSAT Perl script}

Test a simple Perl script that does not require organisms to be
installed.

\begin{lstlisting}
random-seq -l 100 -n 2
\end{lstlisting}

\subsubsection{Testing a \RSAT python script}

Test a simple python script that does not require organisms to be
installed.

\begin{lstlisting}
random-motif -l 10 -c 0.90
\end{lstlisting}

\subsubsection{Testing a compiled C program}

\begin{lstlisting}
random-seq -l 1000 -n 100  | count-words -v 1 -1str
\end{lstlisting}

\subsubsection{Testing external programs}

\program{vmatch}

The program \program{vmatch} is an precious companion to \RSAT
discovery tools (\program{oligo-analysis}, \program{dyad-analysis},
\program{position-analysis}, \program{local-word-analysis}).

It requires a freeware license (\url{http://www.vmatch.de/}).

\begin{lstlisting}
## Quick test with a random sequence
random-seq -l 100 -n 1  | purge-sequence

## Realistic application: retrieve two overlapping promoters ->
## contain redundant sequences (on opposite strands). After purging,
## the second sequence should contain 'n' characters over the last
## ~600 base pairs.
retrieve-seq -org Saccharomyces_cerevisiae -q GAL1 -q GAL10  | purge-sequence
\end{lstlisting}

If you get an error message, see section \ref{sect:install_vmatch} of
this manual.

\program{seqllogo}

The program \program{seqlogo} is used to draw logos from
position-specific scoring matrices. It is required for several \RSAT
tools (\program{convert-matrix}, \program{peak-motifs},
\program{matrix-clustering}, \program{footprint-discovery}, \ldots).

\begin{lstlisting}
## Locate the path of seqlogo
which seqlogo

## get the help for seqlogo
seqlogo

## ghostscript
which gs
gs --version
\end{lstlisting}


\subsubsection{Testing supported organisms}

Get the list of organisms supported on your computer.

\begin{lstlisting}
supported-organisms
\end{lstlisting}

Get the list of organisms supported on the server
\url{http://rsat.eu/}. This script requires some Perl libraries for
the SOAP/WSDL protocol. If it works, it means that you are ready to
use \RSAT Web services.

\begin{lstlisting}
supported-organisms-server -v 2 -server http://rsat.eu/ \
   -o supported_on_rsat_eu.tab ; 

## Count the number of supported organisms on the remote server
wc -l  supported_on_rsat_eu.tab

## Check the list of supported organisms on the remote server
more   supported_on_rsat_eu.tab
\end{lstlisting}

Get the list of organisms supported on another server
\url{http://rsat.fr/}.

\begin{lstlisting}
supported-organisms-server -server http://rsat.fr/ \
   -o supported_on_rsat_fr.tab ; 
wc -l  supported_on_rsat_fr.tab
\end{lstlisting}


\chapter{Full installation}

The full installation of \RSAT sowftare suite includes some additional steps.

You should read this chapter only if you want to enable some of the
following functionalities:
\begin{enumerate}
  \item programs complementary to \RSAT, developed by other teams
  \item local Web server
  \item Web services
  \item distributed computing on a cluster (or on multiple processors
    of a single computer)
  \item metabolic pawthay analysis tools
\end{enumerate}

\section{Installing third-party programs}

\subsection{vmatch and mkvtree}
\label{sect:install_vmatch}

The programs \program{vmatch} and \program{mkvtree} are required by
the \RSAT progam \program{purge-sequence}, which plays an important
role to discard redundant sequences before running motif discovery
algorithms (\program{oligo-analysis}, \program{dyad-analysis},
\program{position-analysis}, \program{local-word-analysis}).

A free academic license can be obtained at Stefan Kurt's web page:

\url{http://www.vmatch.de/}

After having obtained the licence, install the 3 following files in
the \file{\$RSAT/bin} folder: \file{vmatch}, \file{mkvtree},
\file{vmatch.lic}.

Quick test for the correct functioning of \program{purge-sequence}:

\begin{lstlisting}
retrieve-seq -org Saccharomyces_cerevisiae -q GAL1 -q GAL10 -noorf \
  | purge-sequence
\end{lstlisting}

The second sequence (GAL10) should be masked (replaced by ``n''),
because GAL10 and GAL1 shar the same promoter (the genes are
transcribed in opposite direction).

\subsection{Other third-pary programs}

Some additional freeware programs are required for some particular
tasks in \RSAT. The list of these programs can be obtained as follows.

\begin{lstlisting}
make -f makefiles/install_software.mk list_ext_apps
\end{lstlisting}

Calling the makefile with the target \option{install\_ext\_apps} will
start the automatic installation of all these programs. 

\begin{lstlisting}
make -f makefiles/install_software.mk install_ext_apps
\end{lstlisting}

Note: for some programs, you may be prompted for the sudo password,
depending on the configuration you entered in the previous step (with
the script \file{configure\_rsat.pl}).

In case of trouble, try to install the programs one by one by calling
separately each target listed by \option{list\_ext\_apps}.

\subsection{Configuration of the Apache server}

In order to provide web access to the Regulatory Sequence Analysis
Tools (\RSAT), you need to adapt the configuration of your web
server. This requires root privileges (can be done only by the system
administrator of the computer).


\begin{enumerate}
\item A default configuration file is provided with the \RSAT
  distribution (\file{rsat\_apache\_default.conf}). Edit this file to
  replace the string \texttt{\[RSAT\_PARENT\_PATH\]} by the actual
  location of your rsat folder.
  
\item The file should then be copied to some appropriate place in the
  Apache configuration folder of your computer. This place depends on
  the operating system (Mac OSX or Linux) and on the distribution
  (Linux Ubuntu, Centos, ...).
  
  Some Usual places:
  \begin{itemize}
  \item On Centos: \file{/etc/httpd/conf.d/rsat.conf}
  \item On Ubuntu: \file{/etc/apache2/sites-enabled/rsat.conf}
  \item On Mac OSX: \file{/etc/apache2/users/rsat.conf}
  \end{itemize}
  
\item You need to restart the Web server (the command depends on your
  OS. Can be \program{apachectl}, \program{apache2ctl} or
  \program{httpd}.

  \begin{lstlisting}
sudo apache2ctl restart
  \end{lstlisting}

\item Check that all properties related to the Web site URL are
  properly defined in the \RSAT property files
  \file{\$RSAT/RSAT\_config.props} and
  \file{\$RSAT/RSAT\_config.mk}. 

  In principle you already configured these files in the beginning of
  the installation, with the command
  \begin{lstlisting}
perl perl-scripts/configure_rsta.pl
  \end{lstlisting}

  \textbf{Note:} it is important to properly define the URL fo the Web
  server (\texttt{RSAT\_WWW} and related variables). The default URL
  (http://localhost/rsat/) only works if the server and client (your
  Web browser) are on the same machine. This internal access is very
  convenient to work in places where yo don't have internet
  connections, but does not allow other computers to use your Web
  server.  If you want to enable Web queries from remote computers,
  you should specify an externally visible URL.

\end{enumerate}



\subsection{Activating web services on your \RSAT instance}

By default, Web services requests are redirected towards the main
\RSAT server. To configure your \RSAT instance as a Web services
provider, you first need to update the WSDL file, which provides all
technical information about supported web services.

  \begin{lstlisting}
cd $RSAT ;
make -f makefiles/init_rsat.mk ws_init
  \end{lstlisting}

This will also display the parameters of your local web services,
which depend on the variable RSAT\_WWW when you ran the \RSAT
configuration script (\file{perl-scripts/configure\_rsat.pl}). Check
that these parameters are correct.

After this, you should generate the web services stubb, with the
following command.

  \begin{lstlisting}
make -f makefiles/init_rsat.mk ws_stubb
  \end{lstlisting}

You can test if the web services are working.

  \begin{lstlisting}
make -f makefiles/init_rsat.mk ws_stubb_test
  \end{lstlisting}

\chapter{Description and requirements}

\section{Description}

The Regulatory Sequence Analysis Tools (\RSAT) is a software suite
specialized for the detection of cis-regulatory elements in genomic
sequences. It also contains a series of complementary tools for genome
management, statistics, and other related analyses. 

The \RSAT package comes along with the Network Analysis Tools (NeAT),
a software suite combining a variety of tools for the analysis of
biomolecular networks (interactome, regulatory networks, metabolic
pathways).

\section{Requirements}

\subsection{Operating system}

\RSAT is a unix-based software suite. It has been installed
successfully on the following operating systems.

\begin{enumerate}
\item Linux

\item Mac OSX (latest version tested: 10.8.3)

\item Sun Solaris

\item Dec Alpha

\end{enumerate}

\RSAT is not compatible with any version of Microsoft Windows and we
have no intention to make it compatible in a foreseeable future.

\subsection{Perl language}

Most of the programs in \RSAT are written in Perl. Version 5.1 or
later is recommended. A set of Perl modules is required, the \RSAT
package includes a script to install them automatically (see
Chapter~\ref{chap:perl_modules}).

\subsection{Python language}

Some of the programs in \RSAT are written in Python. 

Python release
2.7\footnote{\url{http://www.python.org/getit/releases/2.7/}} is
recommended, because it contains some required libraries for remote
access to external resources (UCSC genome browser).

The following Python libraries are required for various programs.

\begin{itemize}
\item
  \textbf{setuptools}\footnote{\url{http://pypi.python.org/pypi/setuptools}}
  is required to install other Python libraries (see installation
  instructions\footnote{\url{http://pypi.python.org/pypi/setuptools\#installation-instructions}}).

\item \textbf{suds}\footnote{\url{https://fedorahosted.org/suds/}} is
  used for accessing the SOAP interface.

\end{itemize}


\subsection{Helper applications}



\subsubsection{wget}

The program \program{wget}, is used  to download
\begin{enumerate}

\item some helper programs developed by third-parties, which can be
  installed in \RSAT;

\item genomes from the \RSAT server to your local \RSAT installation.

\end{enumerate}

\program{wget} is part of linux distribution. If it is not installed
on your computer, you can download it from
\url{http://www.gnu.org/software/wget/}.  An installation package for
Mac OSX can be found at
\url{http://download.cnet.com/Wget/3000-18506_4-128268.html}.


\subsubsection{gnuplot}

The standard version of the \RSAT program \program{XYgraph} export
figures in bitmapformat (png, jpeg). If you want to support vectorial
drawings (pdf), which give a much better resolution for printing, you
need to install the freeware software \program{gnuplot} (4.2 or
later), which can be downloaded from \url{http://www.gnuplot.info/}.


\subsubsection{git (only for developers)}

For co-developers of the \RSAT suite, the code is distributed program
\program{git}. For external users, there is no need to use
\program{git} since the code is distributed as a compressed archive on
a Web page.


%%%%%%%%%%%%%%%%%%%%%%%%%%%%%%%%%%%%%%%%%%%%%%%%%%%%%%%%%%%%%%%%
% Installation
\chapter{Obtaining \RSAT distribution}

For the time being, \RSAT is distributed as a compressed archive. 

The license can be obtained from the \RSAT web site (\rsaturl).

\section{Installation from a compressed archive}

Download the latest version of the \RSAT distribution. Uncompress the
archive containing the programs. The archive is distributed
\texttt{tar} format. The \texttt{.tar.gz} file can be uncompressed
with the command \program{tar}, which is included in most Unix
distributions.

\begin{lstlisting}
tar -xpzf rsat_yyyymmdd.tar.gz
\end{lstlisting}

%%%%%%%%%%%%%%%%%%%%%%%%%%%%%%%%%%%%%%%%%%%%%%%%%%%%%%%%%%%%%%%%
%% The section about CVS has been moved to a separate file ig_cvs.tex,
%% and is not displayed anymore for two reasons: 
%% (1) the CVS distribution is reserved to RSAT team members.
%% (2) We are in the process to move from CVS to GIT. 
%%
%% \section{Downloading \RSAT from the CVS server}

\textbf{Warning:} the CVS distribution is reserved to the members of
the \RSAT team. For external users, \RSAT is distributed as a tar
archive (file \file{rsa-tools\_YYYY-MM-DD.tar.gz}, where YYYY-MM-DD
indicates the release date). If you are not a member of the \RSAT team
you can thus skip this section.

\subsection{CVS configuration}

You need to indicate your \program{cvs} client to use \program{ssh}
as remote shell application. For this, you can specify an environment
variable.

if your shell is tcsh, add the following line to the \file{.cshrc}
file in your home directory.

\begin{lstlisting}
setenv CVS_RSH ssh
\end{lstlisting}


If your shell is bash, add the following line to the \file{.bashrc}
file in your home directory.

\begin{lstlisting}
export CVS_RSH=ssh
\end{lstlisting}

\subsection{Obtaining a first version of \RSAT programs}

The following command should be used the first time you retrieve the
tools from the server (you need to replace \texttt{[mylogin]} by the
login name you received when signing the \RSAT license).

\begin{lstlisting}
cvs -d [mylogin]@cvs.bigre.ulb.ac.be:/cvs/rsat co rsa-tools
\end{lstlisting}


This will create a directory \file{rsa-tools} on your computer, and
store the programs in it. Note that at this stage the programs are not
yet functional, because you still need to install genomes, which are
not included in the CVS distribution.

\subsection{Updating \RSAT programs}

Once the tools have been retrieved, you can obtain updates very
easily. For this, you need to change your directory to the rsa-tools
directory, and use the \texttt{cvs} command in the following way.

\begin{lstlisting}
cd rsa-tools
cvs -q update -d
\end{lstlisting}





%%%%%%%%%%%%%%%%%%%%%%%%%%%%%%%%%%%%%%%%%%%%%%%%%%%%%%%%%%%%%%%%
%M initialization
\chapter{Initializing \RSAT}

% \section{Configuring \RSAT}


In order to use the command-line version of \RSAT, you first need an
account on a Unix machine where \RSAT has been installed, and you
should know the directory where the tools have been installed (if you
don't know, ask assistance to your system administrator).

In the following instruction, we will assume that \RSAT is installed
in the directory \texttt{/home/rsat/rsa-tools}. This path has to be
replaced by the actual path where \RSAT has been installed on your
computer.

%%%%%%%%%%%%%%%%%%%%%%%%%%%%%%%%%%%%%%%%%%%%%%%%%%%%%%%%%%%%%%%%
%%%% Accessing the programs
%%%%%%%%%%%%%%%%%%%%%%%%%%%%%%%%%%%%%%%%%%%%%%%%%%%%%%%%%%%%%%%%
\subsection{Adding \RSAT  to your path}

Before starting to use the tools, you need to define an environment
variable (\texttt{RSAT}), and to add some directories to your path.

For the following instructions, we will denote as \file{[RSAT\_PARENT\_PATH]}
the path of the parent folder in which the \RSAT suite has been
downloaded.

For example, if the \file{rsa-tools} folder has been installed in your
home directory (it is thus found at \file{/home/fred/rsa-tools}), you
should replace \file{[RSAT\_PARENT\_PATH]} by \file{/home/fred} in all the
following instructions. 


\begin{enumerate}

\item Identify your \texttt{SHELL}.

  The way to execute the following instructions depends on your
  ``shell'' environment. If you don't know which is your default
  shell, type

\lstset{language=csh}

\begin{lstlisting}
echo $SHELL
\end{lstlisting}
%$

The answer should be something like \result{/sbin/bash} or
\result{/bin/tcsh}

\item Declare the \RSAT environment variables.

  We will define configuration parameters necessary for \RSAT.  This
  includes an environment variable named \texttt{RSAT}.  We will then
  add the path of the \RSAT \ perl scripts, python scripts and
  binaries to your path. In addition, add java jar files to your
  \texttt{CLASSPATH}.

\begin{itemize}

\item \emph{If your default shell is \textbf{tcsh} or \textbf{csh}},
  copy the following lines in a file named \file{.chsrc} that should
  be found at root of your account.

\begin{lstlisting}
## Note: replace [PARENT_PATH] by the full path of the directory in
## which the rsa-tools folder has been created
setenv RSAT [PARENT_PATH]/rsa-tools
set path=($RSAT/bin $path)
set path=($RSAT/perl-scripts $path)
set path=($RSAT/python-scripts $path)
set CLASSPATH=($RSAT/java/lib/NeAT_javatools.jar=${CLASSPATH})
\end{lstlisting}
%% closing $

\item \emph{If your shell is bash}, you should copy the following
  lines in a file named \file{.bash\_profile} at the root of your
  account (depending on the Unix distribution, the bash custom
  parameters may be declered in a file named \file{.bash\_profile} or)
  \file{.bashrc}, in case of doubt ask your system administrator).

\begin{lstlisting}
## Note: replace [PARENT_PATH] by the full path of the directory in
## which the rsa-tools folder has been created
export RSAT=[PARENT_PATH]/rsa-tools
export PATH=${RSAT}/bin:${PATH}
export PATH=${RSAT}/perl-scripts:${PATH}
export PATH=${RSAT}/python-scripts:${PATH}
export CLASSPATH=${RSAT}/java/lib/NeAT_javatools.jar:${CLASSPATH}
\end{lstlisting}


\item If you are using a different shell than bash, csh or tcsh, the
specification of environment variables might differ from the syntax
above.  In case of doubt, ask your system administrator how to
configure your environment variables and your path.

\end{itemize}

\item In order for the variables to be taken in consideration, you
  need to log out and open a new terminal session. To check that the
  variables are correctly defined, type.

\begin{lstlisting}
  echo $RSAT
\end{lstlisting}
%%$

In the example above, this command should return

\begin{lstlisting}
/home/fred/rsa-tools
\end{lstlisting}


\end{enumerate}



\subsection{Configuring your \RSAT\ server}

RSAT requires to specify a set of parameters, which will be stored in
three property files:
\begin{enumerate}
\item \file{RSAT\_config.props}: this file describes site-specific
  parameters. The same property file is loaded by the Perl, python and
  java programs, thereby ensuring the consistency of site-specific
  configuration.
\item \file{RSAT\_config.mk}: a subset of parameters from the props
  file, which are required to run the makefiles (in particular during
  RSAT installation).
\item \file{RSAT\_config.bashrc}: definiton of environment variables,
  paths and Perl library paths required to run the RSAT/NeAT tools.
\item \file{RSAT\_config.conf}: configuration of the web site (for
  Apache servers).
\end{enumerate}

In particular, \emph{it is crucial to specify the full path of the
  variable RSAT}, which specifies the RSAT main directory.

The simplest way to update the configuration file is to run the
following script.

\begin{lstlisting}
## Enter in RSAT distribution folder
cd rsat

## Run the configuration script
perl perl-scripts/configure_rsat.pl
\end{lstlisting}

Alternatively, you can edit the files with a text editor of your
choice.

\subsection{Loading RSAT environment variables}

The configuration script has created a bash file
\file{RSAT\_config.bashrc} in the \RSAT distribution directory (folder
\file{rsat}). 

This file should be loaded each time you enter a session. There are
several alternative ways to do this.

\begin{enumerate}

\item Source the file manually at each new session

\begin{lstlisting}
## Load the RSAT environment variables
source RSAT_config.bashrc
\end{lstlisting}

\item Copy the content of this file in the bash configuration file in
  your home directory (\file{~/.bashrc} or
  \file{~/.bash\_profile}). \RSAT environment variables will then be
  loaded for you at each connexion.

\item If you dispose of admin rights, you can copy the content of this
  file in the main bash configuration file: 
  \begin {itemize}
  \item in Lunix Centos, you can create a file
    \file{/etc/profile.d/rsat.bashrc}; 
  \item Mac OSX, copy it to the main bashrc file (\file{/etc/bashrc}).
  \end{itemize}
  \RSAT environment variables will then be automatically loaded for
  each user of this computer.
\end{enumerate}


%%%%%%%%%%%%%%%%%%%%%%%%%%%%%%%%%%%%%%%%%%%%%%%%%%%%%%%%%%%%%%%%
% Directories

\section{Initializing the directories}

In addition to the programs, the installation of rsat requires
the creation of a few directories for storing data, access logs (for
the web server), and temporary files.

The distribution includes a series of make scripts which will
facilitate this step. You just need go to the rsat directory, and
start the appropriate make file.

\begin{lstlisting}
cd $RSAT ; 
make -f makefiles/init_rsat.mk init
\end{lstlisting}
% $

%% %%%%%%%%%%%%%%%%%%%%%%%%%%%%%%%%%%%%%%%%%%%%%%%%%%%%%%%%%%%%%%%%
%% % Config file
%% \section{Adapting \RSAT \  local configuration}


%% \subsection{Creating site-specific configuration files}

%% The \RSAT distribution comes with two template configuration files
%% located in the \file{rsat} directory:

%% \begin{itemize}
%% \item \file{RSAT\_config\_default.props} 
%% \item \file{RSAT\_config\_default.mk}
%% \end{itemize}

%% During the initialization script executed in the previous section,
%% these two default files were automatically copied to your
%% site-specific configuration files \file{RSAT\_config.props} and
%% \file{RSAT\_config.mk}.

%% Check this before going any further: 

%% \begin{lstlisting}
%% ls -1 RSAT_config*
%% \end{lstlisting}

%% You should see this:
%% \begin{verbatim}
%% RSAT_config_default.mk
%% RSAT_config_default.props
%% RSAT_config.mk
%% RSAT_config.props
%% \end{verbatim}



%%%%%%%%%%%%%%%%%%%%%%%%%%%%%%%%%%%%%%%%%%%%%%%%%%%%%%%%%%%%%%%%
%% Check the RSAT paths
\subsection{Checking the RSAT path}

The \RSAT programs should now be included in your path.  To check if
this is done properly, just type:

\begin{lstlisting}
random-seq -l 350
\end{lstlisting}

If your configuration is correct, this command should return a random
sequence of 350 nucleotides.

Don't worry if you see a warning looking like this:

\begin{footnotesize}
\begin{verbatim}
; WARNING	The tabular file with the list of supported organism cannot be read
; WARNING	Missing file	[RSAT_PARENT_PATH]/rsat/public_html/data/supported_organisms.tab
\end{verbatim}
\end{footnotesize}

This warning will disappear as soon as you download the first organism
in \RSAT.


\chapter{Installing Perl modules}
\label{chap:perl_modules}

Some Perl modules are required for the graphical tools of \RSAT, and
for some other specific programs. The perl modules can be found in the
Comprehensive Perl Archive Network (\url{http://www.cpan.org/}), or
can be installed with the command \program{cpan}.

\section{Before installing Perl modules: install the GD library}

The Perl module \program{GD.pm} requires prior installation of the
\program{GD} library.

\begin{itemize}

\item On \emph{Linux} systems, this library can be installed with the
  package manager of the distribution. for example:
  \begin{itemize}
  \item \program{apt-get} for Ubuntu
  \item \program{aptitude} for Ubuntu (better treatment of dependencies than apt-get)
  \item  \program{yum} for Centos
  \item \program{yast} for Suze
  \item \ldots
  \end{itemize}
  
\item On \emph{Mac OSX} systems, the installation of the GD library
  can be done with the program \program{brew} (\url{http://brew.sh/}).

  After having installed brew, you can install the GD library in your
  system by typing.
  \begin{lstlisting}
brew install gd
\end{lstlisting}
   
  

%% First install libpng and libjpeg. Only then install the GD library. Here is an example that worked on OSX 10.8:
%% \begin{lstlisting}
%% curl -O http://www.ijg.org/files/jpegsrc.v9.tar.gz
%% tar -xzvf jpegsrc.v9.tar.gz 
%% cd jpeg-9/
%% ./configure
%% make
%% sudo make install

%% curl -O ftp://ftp.simplesystems.org/pub/libpng/png/src/libpng15/libpng-1.5.16.tar.gz
%% tar -xvzf libpng-1.5.16.tar.gz 
%% cd libpng-1.5.16
%% ./configure
%% make
%% sudo make install

%% curl -O https://bitbucket.org/libgd/gd-libgd/downloads/libgd-2.1.0-rc1.tar.gz
%% tar -xvzf libgd-2.1.0-rc1.tar.gz
%% cd libgd-2.1.0-rc1
%% ./configure --with-png=/usr/local
%% \end{lstlisting}
\end{itemize}

\section{Automatic installation of Perl modules}

The simplest way to install all the required Perl modules is to tye
the command below. \emph{Beware:} this command sudo requires
administrator rights on the computer. If you don't have the root
password, please consult your system administrator.

\begin{lstlisting}
## Acquire the system administrator rights
sudo bash;

## Define the RSAT environment variable.
##
## You must replace [RSAT_PATH] by the full path to your rsat folder.
export RSAT=[RSAT_PATH]

## Check that the RSAT environment variable has been defined
echo $RSAT

## Check that the RSAT environment variable points towards the right directory
ls -1 $RSAT
## This should give you the list of the files and folders included in your rsat folder.

## Set your working directory to the rsat folder
cd $RSAT

## Display the list of Perl modules that will be installed
make -f makefiles/install_rsat.mk perl_modules_list;

## Print the command that will be used to install the Perl modules (just for checking)
make -f makefiles/install_rsat.mk perl_modules_cmd;

## Install the Perl modules
make -f makefiles/install_rsat.mk perl_modules_install;
\end{lstlisting}
%% closing $

Beware, \program{cpan} will frequently ask you to confirm the
installation steps. you should thus check the CPAN process and answer
"yes" at each prompt.

The following command enables to install all the Perl modules in a
somewhat risky, but less cumbersome way. It relies on the command
\program{yes} to automatically answer each question by a carriage
return, which will lead \program{cpan} to chose the default option.

\begin{lstlisting}
## Install the Perl modules
make -f makefiles/install_rsat.mk perl_modules_install;
\end{lstlisting}

In case some modules would not be properly installed with the above
commands, you can try installing them manually (the list of required
modules is listed in the next section).


\section{Additional Perl modules required to support EnsEMBL genomes}
\label{sect:ensembl_libraries}

This section is required only if you intend to use the \RSAT programs
interfaced to the Ensembl database.  Since 2008, a series of \RSAT
programs support a direct access to the EnsEMBL database in order to
ensure a convenient access to genomes from higher organisms
\cite{Sand:2009}.

\begin{itemize}
\item \program{supported-organisms-ensembl}
\item \program{ensembl-org-info}
\item \program{retrieve-ensembl-seq.pl} 
\item \program{get-ensembl-genome.pl}
\end{itemize}

Those programs require to install a few Perl libraries as well as a
MySQL client on your machine.

The first requirement is the \program{BioPerl} module, which has in
principle been installed in Chapter~\ref{chap:perl_modules}). The
MySQL client should also have been installed in
Chapter~\ref{chap:perl_modules}).

To obtain EnsEMBL \footnote{Full instructions at
  \url{http://useast.ensembl.org/info/docs/api/api_cvs.html}}.

\begin{lstlisting}

## Make sure you start from the right directory
cd $RSAT

## Display the parameters for installing Ensembl API (in particular,
## the version for Ensembl and EnsemblGenomes).
##
## Check the number of the latest release on the respective web sites. 
##   Ensembl: http://www.ensembl.org/index.html
##   EnsemblGenomes: http://ensemblgenomes.org/
make -f makefiles/install_software.mk install_ensembl_api_param

## Install the ensembl library
make -f makefiles/install_software.mk install_ensembl_api 

## Notes: you need to enter the following passwords for the CVS
## servers.
## - For Ensembl:  CVSUSER.
## - For bioperl:  cvs

\end{lstlisting}
%%$ 


Note that there may be incompatibilities between successive versions
of the Ensembl API. The install script includes a parameter
\texttt{ENSEMBL\_VERSION} to specify the version ("branch") of the
Ensembl API distribution.  Moreover, there are different release
numbers of the ``historical'' Ensembl database, and for the
EnsemblGenomes databases (Bacteria, Fungi, Plants, Metazoa).

In addition, there are dependencies between releases. So,
EnsemblGenomes 20 is compatible with Ensembl verison 73, whereas
Ensembl has already released its version 74. In order to install an
API compatible with Ensembl and EnsemblGenomes, we recommend always
check the latest releases of both databases on the EnsemblGenomes web
page (\url{http://www.ensemblgenomes.org/}), and adapt the following
command accordingly.

\begin{lstlisting}
## Install the ensembl library with a specific branch number.
make -f makefiles/install_software.mk install_ensembl_api \
   ENSEMBL_VERSION=73
\end{lstlisting}
%%$ 

The installation script will print out a series of modifications of
the PERL5LIB variable, that should be added to your bashrc file in
order to provide support for Ensembl Perl API.

You should also check the specification of ensembl paths in the props
file.

\begin{lstlisting}
ensembl=[RSAT_PARENT_PATH]/rsat/lib/ensembl/modules
compara=[RSAT_PARENT_PATH]/rsat/lib/ensembl-compara/modules
variation=[RSAT_PARENT_PATH]/rsat/lib/ensembl-variation/modules
\end{lstlisting}

You also need to define the URL of the Ensembl database in that
configuration file:

\begin{lstlisting}
## EnsEMBL host
## Used by the EnsEMBL-accessing tools (retrieve-ensembl-seq,
## get-ensembl-genome).
## URL of the server for the EnsEMBL DB. By default, the
## main ensembl server is called, but a local server can be specified.
ensembl_host=ensembldb.ensembl.org
\end{lstlisting}

%% Finally, you need to include the BioPerl and Ensembl librairies in the
%% Perl module path (specified by the environment variable named
%% \$PERL5LIB).

%% If your shell is bash, add the following lines in the file
%% \file{.bashrc} at the root of your account.

%% \begin{lstlisting}
%% ## Ensembl Perl API
%% export PERL5LIB=${RSAT}/lib/bioperl-release-1-2-3/bioperl-live::${PERL5LIB}
%% export PERL5LIB=${RSAT}/lib/ensemblgenomes-20-73/ensembl/modules::${PERL5LIB}
%% export PERL5LIB=${RSAT}/lib/ensemblgenomes-20-73/ensembl-compara/modules::${PERL5LIB}
%% export PERL5LIB=${RSAT}/lib/ensemblgenomes-20-73/ensembl-external/modules::${PERL5LIB}
%% export PERL5LIB=${RSAT}/lib/ensemblgenomes-20-73/ensembl-functgenomics/modules::${PERL5LIB}
%% export PERL5LIB=${RSAT}/lib/ensemblgenomes-20-73/ensembl-tools/modules::${PERL5LIB}
%% export PERL5LIB=${RSAT}/lib/ensemblgenomes-20-73/ensembl-variation/modules::${PERL5LIB}
%% \end{lstlisting}
%% %$

%% If your shell is tcsh or chs, add the following lines in the file
%% \file{.cshrc} at the root of your account.

%% \begin{lstlisting}
%% ## Ensembl Perl API
%% setenv PERL5LIB ${RSAT}/lib/bioperl-release-1-2-3/bioperl-live::${PERL5LIB}
%% setenv PERL5LIB ${RSAT}/lib/ensemblgenomes-20-73/ensembl/modules::${PERL5LIB}
%% setenv PERL5LIB ${RSAT}/lib/ensemblgenomes-20-73/ensembl-compara/modules::${PERL5LIB}
%% setenv PERL5LIB ${RSAT}/lib/ensemblgenomes-20-73/ensembl-external/modules::${PERL5LIB}
%% setenv PERL5LIB ${RSAT}/lib/ensemblgenomes-20-73/ensembl-functgenomics/modules::${PERL5LIB}
%% setenv PERL5LIB ${RSAT}/lib/ensemblgenomes-20-73/ensembl-tools/modules::${PERL5LIB}
%% setenv PERL5LIB ${RSAT}/lib/ensemblgenomes-20-73/ensembl-variation/modules::${PERL5LIB}
%% \end{lstlisting}

%% The EnsEMBL libraries also require the SQL client Perl module
%% \file{DBD::mysql}, as well as \file{DBI}, which can be installed with
%% \program{cpan} (for this you need root privileges, see Chapter~\ref{chap:perl_modules}). 

\emph{Notes}: 
\begin{enumerate}
\item to access EnsEMBL versions above 47, you need port 5306 to be
  opened on your machine. This might require an intervention of your
  system administrator of your network in order to ensure that the
  Firewall accepts this port.
\end{enumerate}

Detailed information about the EnsEMBL libraries can be obtained on
the EnsEMBL web site
(\footnote{\url{http://www.ensembl.org/info/using/api/api\_installation.html}}).


%%%%%%%%%%%%%%%%%%%%%%%%%%%%%%%%%%%%%%%%%%%%%%%%%%%%%%%%%%%%%%%%
%% Compile C/C++ programs
\chapter{Compiling C programs in \RSAT}

Some of the tools available in \RSAT (\program{info-gibbs},
\program{matrix-scan-quick}, \program{count-words}) are written in the
\program{C} language. The distribution only contains the sources of
these tools, because the binaries are operating system-dependent. The
programs can be compiled in a very easy way.

\begin{lstlisting}
cd $RSAT;
make -f makefiles/init_rsat.mk compile_all
\end{lstlisting}

This will compile and install the following programs in the directory
\file{\$RSAT/bin}. 


%%%%%%%%%%%%%%%%%%%%%%%%%%%%%%%%%%%%%%%%%%%%%%%%%%%%%%%%%%%%%%%%
% Downloading genomes from RSAT main server

%%%%%%%%%%%%%%%%%%%%%%%%%%%%%%%%%%%%%%%%%%%%%%%%%%%%%%%%%%%%%%%%
%%%% Downloading organisms from RSAT data repository
%%%%%%%%%%%%%%%%%%%%%%%%%%%%%%%%%%%%%%%%%%%%%%%%%%%%%%%%%%%%%%%%

\chapter{Downloading genomes}
\label{downloading_genomes}

\RSAT includes a series of tools to install and maintain the latest
version of genomes.

The most convenient way to add support for one or several organisms on
your machine is to use the programs \program{supported-organisms} and
\program{download-organism}.

Beware, the complete data required for a single genome may occupy
several hundreds of Mb, because \RSAT not only stores the genome
sequence, but also the oligonucleotide frequency tables used to
estimate background models, and the tables of BLAST hits used to get
orthologs for comparative genomics. If you want to install many
genomes on your computer, you should thus reserve a sufficient amount
of space.

\section{Original data sources}

Genomes supported on \RSAT were obtained from various sources.

Genomes can be installed either from the \RSAT web site, or from their
original sources.  

\begin{itemize}
\item NCBI/Genbank (\url{ftp://ftp.ncbi.nih.gov/genomes/})

\item ENSEMBL (\url{http://www.ensembl.org/})

\item The EBI genome directory (\url{ftp://ftp.ebi.ac.uk/pub/databases/genomes/Eukaryota/})

\end{itemize}

Other genomes can also be found on the web site of a diversity of
genome-sequencing centers.

\section{Requirement : wget}

The download of genomes relies on the application \program{wget},
which is part of linux distribution\footnote{For Linux:
  \url{http://www.gnu.org/software/wget/}; for Mac OSX
  \url{http://download.cnet.com/Wget/3000-18506_4-128268.html}}.


\program{wget} is a ``web aspirator'', which allows to download whole
directories from ftp and http sites. You can check if the program is
installed on your machine.

\begin{lstlisting}
wget --help
\end{lstlisting}


This command should return the help pages for \program{wget}.  If you
obtain an error message (``command not found''), you need to ask your
system administrator to install it.

\section{Importing organisms from the \RSAT main server}

The simplest way to install organisms on our \RSAT site is to download
the RSAT-formatted files from the web server. For this, you can use a
web aspirator (for example the program \program{wget}). 

Beware, the full installation (including Mammals) requires a large
disk space (several dozens of Gb). You should better start installting
a small genome and test it before processing to the full
installation. We illustrate the approach with the genome of our
preferred model organism: the yeast \textit{Saccharomyces cerevisiae}.

\subsection{Obtaining the list of organisms supported on the \RSAT server}

By default, the program \program{supported-organisms} returns the list
of organisms supported on your local \RSAT installation. You can
however use the option \option{-server} to obtain the list of
organisms supported on a remote server.


\begin{lstlisting}
supported-organisms -server
\end{lstlisting}

The command can be refined by restricting the list to a given taxon of
interest.

\begin{lstlisting}
supported-organisms -server -taxon Fungi
\end{lstlisting}

You can also ask additional information, for example the date of the
last update and the source of each genome.

\begin{lstlisting}
supported-organisms -server -taxon Fungi -return last_update,source,ID
\end{lstlisting}


\subsection{Importing a single organism}


The command \command{download-organism} allows you to download one or
several organisms. 

Beware, the complete data for a single genome may occupy several tens
of Megabytes (Bacterial genomes) or a few Gigabases
(Mammalian). Downloading tenomes thus requires a fast Internet
connection, and may take time. If possible, please download genomes
during the night (European time).

As a first step, we recommend to download the genome of the yeast
\org{Saccharomyces cerevisiae}, since this is the model organism used
in our tutorials.


\begin{lstlisting}
download-organism -v 1 -org Saccharomyces_cerevisiae
\end{lstlisting}

In principle, the download should start immediately. \emph{Beware},
the data volume to be downloaded is important, because the genome
comes together with extra files (blast hits with other genoems,
oligonucleotide and dyad frequencies). Depending on the network
bandwidth, the download of a genome may take several minutes or tens
of minutes. 

After the task is completed, you can check if the configuration file
has been correctly updated by typing the command.

\begin{lstlisting}
supported-organisms
\end{lstlisting}

In principle, the following information should be displayed on your
terminal.

\result{Saccharomyces\_cerevisiae}

You can also add parameters to get specific information on the
supported organisms.

\begin{lstlisting}
supported-organisms -return ID,last_update
\end{lstlisting}


\subsection{Importing a few selected organisms}

The program \program{download-organism} can be launched with a list of
organisms by using iteratively the option \option{-org}.


\begin{lstlisting}
download-organism -v 1 -org Escherichia_coli_K12 -org Salmonella_typhi
\end{lstlisting}

\subsection{Importing all the organisms from a given taxon}

For comparative genomics, it is convenient to install on your server
all the organisms of a given taxon. For this, you can simply use the
option \option{-taxon} of \program{download-organism}.

Before doing this, it is wise to check the number of genomes that
belong to this taxon on the server.

\begin{lstlisting}
## Get the list of organisms belonging to the order "Enterobacteriales" on the server
supported-organisms -taxon Enterobacteriales -server

## Count the number of organisms
supported-organisms -taxon Enterobacteriales -server | wc -l
\end{lstlisting}

In Oct 2009, there are 94 Enterobacteriales supported on the \RSAT
server. Before starting the download, you should check two things:
\begin{enumerate}
\item Has your network a sufficient bandwidth to ensure the transfer
  in a reasonable time ?
\item Do you have enough free space on your hard drive to store all those genomes ? 
\end{enumerate}

If the answer to both questions is ``yes'', you can start the
download.

\begin{lstlisting}
download-organism -v 1 -taxon Enterobacteriales 
\end{lstlisting}


\section{Adding support for Ensembl genomes}

In addition to the genomes imported and maintained on your local \RSAT
server, the program \program{retrieve-ensembl-seq} allows you to
retrieve sequences for any organism supported in the Ensembl database
(\url{http://ensembl.org}).

For this, you first need to install the Bioperl and Ensembl Perl
libraries (see section \ref{sect:ensembl_libraries}).

\subsection{Handling genomes from Ensembl}

The first step to work with Ensembl genomes is to check the list of
organisms currently supported on their Web server.

\begin{lstlisting}
supported-organisms-ensembl
\end{lstlisting}

You can then get more precise information about a given organism
(build, chromosomes) with the command \program{ensembl-org-info}.

\begin{lstlisting}
ensembl-org-info -org Drosophila_melanogaster
\end{lstlisting}

Sequences can be retrieved from Ensembl with the command
\program{retrieve-ensembl-seq}. 

You can for example retrieve the 2kb sequence upstream of the
transcription start site of the gene \gene{PAX6} of the mouse. 

\begin{lstlisting}
retrieve-ensembl-seq.pl -org Mus_musculus -q PAX6 \
  -type upstream -feattype mrna -from -2000 -to -1 -nogene -rm \
  -alltranscripts -uniqseqs
\end{lstlisting}

Options

\begin{itemize}

\item \option{-type upstream} specifies that we want to collect the
  sequences located upstream of the gene (more procisely, upstream of
  the mRNA).

\item \option{-feattype mrna} indicates that the reference for computing
  coordinates is the mRNA. Since we collect upstream sequences, the
  5'most position of the mRNA has coordinate 0, and upstream sequences
  have negative coordinates. Note that many genes are annotated with
  multiple RNAs for different reasons (alternative splicing,
  alternative transcription start sites). By default, the program will
  return the sequences upstream of each mRNA annotated for the query
  gene.

\item \option{-nogene} clip the sequences to avoid overlapping the next
  upstream gene.

\item \option{-rm} repeat masking (important for pattern
  discovery). Repetitive sequences are replaced by \seq{N} characters.

\end{itemize} 





%%%%%%%%%%%%%%%%%%%%%%%%%%%%%%%%%%%%%%%%%%%%%%%%%%%%%%%%%%%%%%%%
% Tests for the command line tools
\chapter{Testing the command-line tools}

\section{Testing the access to the programs}

\subsection{Perl scripts}

From now on, you should be able to use the perl scripts from the
command line. To test this, run:

\begin{lstlisting}
random-seq -help
\end{lstlisting}


This should display the on-line help for the random sequence
generator.

\begin{lstlisting}
random-seq -l 200 -n 4
\end{lstlisting}

Should generate a random sequence of 200 nucleotides.


\subsection{Testing Perl graphical librairies}

\RSAT includes some graphical tools (\program{feature-map} and
\program{XYgraph}), which require a proper installation of Perl
modules.

\begin{description}
\item[GD.pm] Interface to Gd Graphics Library.
\item[PostScript::Simple]  Produce PostScript files from Perl.
\end{description}

To test if these modules are available on your machine, type.

\begin{lstlisting}
feature-map -help
\end{lstlisting}

If the modules are available, you should see the help message of the
program feature-map. If not, you will see an error message complaining
about the missing librairies. In such a case, ask your system
administrator to install the missing modules.

\subsection{Python scripts}

The \RSAT distribution includes some Python scripts. To test if they
are running correctly, you can try the proram \program{random-motif}.

\begin{lstlisting}
random-motif  -l 10 -c 0.85 -n 3
\end{lstlisting}

This command will generate 3 position-specific scoring matrix (PSSM)
of 10-columns with 85\% conservation of one residue in each column.

\subsection{C programs}

You can test the correct installation of the C programs with the
following command.

\begin{lstlisting}
random-seq -l 1000 -n 10 | count-words -l 2 -v 1 -2str -i /dev/stdin
\end{lstlisting}

The first program (\program{random-seq}) is a Perl script, which
generates a random sequence. The output is directly piped to the C
program \program{count-words}, which computes the frequencies and
occurrences of each dinucleotide.

\section{Testing genome installation}

We will now test if the genomes are correctly installed. You can
obtain the list of supported organisms with the command:

\begin{lstlisting}
supported-organisms
\end{lstlisting}


If this command returns no result, it means that genomes were either
not installed, or not correctly configured. In such a case, check the
directories in the \file{data/genomes} directory, and check that the
file \file{data/supported\_organisms.pl}.

Once you can obtain the list of installed organisms, try to retrieve
some upstream sequences. You can first read the list of options for the
\program{retrieve-seq} program.

\begin{lstlisting}
retrieve-seq -help
\end{lstlisting}


Select an organism (say \org{Saccharomyces cerevisiae}), and
retrieve all the start codons with the following options :

\begin{lstlisting}
retrieve-seq -org Saccharomyces_cerevisiae \
        -type upstream -from 0 -to +2 -all \
        -format wc -nocomment
\end{lstlisting}


This should return a set of 3 bp sequences, mostly ATG (in the case of
\org{Saccharomyces cerevisiae} at least)




%% \section{Further steps}

%% The installation is now finished, you can start the user's guide.

%% In case you would like to install additional genomes that are not
%% supported on \RSAT main server, the next chapter indicates yiou how to
%% proceed.


\chapter{Installing third-party programs}

\section{Complementary programs for the analysis of regulatory
  sequences}

The \RSAT distribution only contains the programs developed by the
\RSAT team.

A few additional programs, developed by third parties, can optionally
be integrated in the package. All third-party programs may be loacated
in the directory \emph{bin} directory of the \RSAT distribution.  

In order to add functionalities to \RSAT, install some or all of these
programs and include their binaries path \file{\$RSAT/bin}. If you are
not familiar with the installation of unix programs, ask assistance to
your system administrator.

Some of those can be downloaded and installed automatically using the
makefile \file{install\_rsat.mk}. Before doing this, you must make
sure that the program \program{wget} (this program is supported on
Linux \urlref{http://www.gnu.org/software/wget/} and Mac OSX
\urlref{http://download.cnet.com/Wget/3000-18506_4-128268.html}
systems).

You can then run the following commands to install some of the
third-party programs that are complementary to \RSAT.

\begin{lstlisting}
cd $RSAT;
make -f makefiles/install_software.mk install_ext_apps
\end{lstlisting}
%$

Some other third-party programs will require a manual installation (in
particular, \program{vmatch} and \program{mkvtree}).


\begin{description}
\item[\program{vmatch} and \program{mkvtree}]: developed by Stefan
  Kurtz, are used by the program \program{purge-sequences}, to mask
  redundant sequences that bias motif discovery statistics.

\item[\program{seqlogo}]: developed by Thomas D. Schneider, is used
  used by the programs \program{convert-matrix},
  \program{compare-matrices}, \program{peak-motifs},
  \program{matrix-quality} and a few others, to generate
  logos. \program{seqlogo} is the command-line version of
  \program{WebLogo}\urlref{http://weblogo.berkeley.edu/}.

  Download the source code archive and uncompress it. Copy the
  following files to the directory \emph{bin} of your \RSAT
  distribution: \file{seqlogo}, \file{logo.pm}, \file{template.pm} and
  \file{template.eps}.

\item \program{seqlogo} requires a recent version of \program{gs}
  (ghostscript\urlref{http://www.ghostscript.com/}) to create PNG and
  PDF output, and \program{ImageMagic's
    convert}\urlref{http://www.imagemagick.org/} to create GIFs.

\item[matrix-based pattern discovery]: several third-party pattern
  discovery programs can be optionally called from some \RSAT task
  managers (e.g. \program{multiple-family-analysis},
  \program{peak-motifs}). 
  \begin{itemize}
  \item \program{meme} (Tim Bailey)
  \item \program{consensus} (Jerry Hertz)
  \item \program{MotifSampler} (Gert Thijs)
  \item \program{gibbs} (Andrew Neuwald)
  \end{itemize}

  Their installation is not properly required for \RSAT functioning,
  but it may be convenient to install them in order to compare the
  results retured by alternative motif discovery approaches on the
  same data sets.
\end{description}


\begin{table}
\begin{center}
\begin{tabular}{lll}
  \hline
  Program & author  & URL \\
  \hline
  vmatch+mkvtree & Stefan Kurtz & \footnotesize{\url{http://www.vmatch.de/}} \\
  seqlogo & Thomas Sneider & \footnotesize{\url{http://weblogo.berkeley.edu/}} \\
  patser & Jerry Hertz & \footnotesize{\url{ftp://ftp.genetics.wustl.edu/pub/stormo/Consensus/}} \\
  consensus & Jerry Hertz &  \footnotesize{\url{ftp://ftp.genetics.wustl.edu/pub/stormo/Consensus/}} \\
  meme & Tim Bailey & \footnotesize{\url{http://meme.sdsc.edu/}} \\
  MotifSampler & Gert Thijs & \footnotesize{\url{http://www.esat.kuleuven.ac.be/~thijs/download.html}} \\
%%  gibbs & Andrew Neuwald & \footnotesize{\url{ftp://ftp.ncbi.nih.gov/pub/neuwald/gibbs9\_95/}} \\
  \hline
\end{tabular}
\end{center}
\caption{\label{table:other_programs} Programs from other developers
  which are complementary to the \RSAT package.}
\end{table}


%\section{Recommended programs for the Network Analysis Tools (NeAT)}
%
%
%For the Network Analysis Tools (NeAT), we recommend to install the
%following programs, which offer complementary functionalities for the
%analysis of networks/graphs.
%
%Some programs come in the contrib directory of the \RSAT
%distribution. Some others have to be downloaded from their original
%distribution site.
%
%To compile the program \program{floydwarshall} located in the
%\file{contrib/floydwarshall} directory of \RSAT, use this command :
%
%\begin{lstlisting}
%gcc $RSAT/contrib/floydwarshall/floydwarshall.c -o $RSAT/bin/floydwarshall
%\end{lstlisting}
%
%
%In addition, the contributed programs REA and kWalks need to be installed.
%
%To install kWalks, type:
%
%
%\begin{lstlisting}
%cd $RSAT/contrib; 
%tar -xzvf kwalks/kwalks.tgz; 
%cd $RSAT/contrib/kwalks/src; 
%make clean; make
%\end{lstlisting}
%
%You can test taht the compilation worked by running the following
%command.
%
%\begin{lstlisting}
%$RSAT/contrib/kwalks/bin/lkwalk
%\end{lstlisting}
%%% $
%
%This should display the help message of \program{lkwalk}.
%
%Check that the KWALKS\_ROOT variable in the RSAT config file
%(\file{\$RSAT/RSAT\_config.props}) points to the correct path (it
%should be the absolute path of \file{\$RSAT/contrib/kwalks/bin/}).
%
%To install REA, type:
%
%\begin{lstlisting}
%cd $RSAT/contrib/REA; 
%tar xzvf REA.tgz; 
%rm -f *.o; 
%make
%\end{lstlisting}
%
%You can test taht the compilation worked by running the following
%command.
%
%\begin{lstlisting}
%$RSAT/contrib/REA/REA
%\end{lstlisting}
%
%This should display the help message of \program{REA}.
%
%Check that the REA\_ROOT variable in the RSAT config file
%(\file{\$RSAT/RSAT\_config.props}) points to the correct path (it
%should be the absolute path of \file{\$RSAT/contrib/REA}).

%%%%%%%%%%%%%%%%%%%%%%%%%%%%%%%%%%%%%%%%%%%%%%%%%%%%%%%%%%%%%%%%
% Parsing and installing additional genomes

%%%%%%%%%%%%%%%%%%%%%%%%%%%%%%%%%%%%%%%%%%%%%%%%%%%%%%%%%%%%%%%%
%% Parsing and installing new organisms
%%%%%%%%%%%%%%%%%%%%%%%%%%%%%%%%%%%%%%%%%%%%%%%%%%%%%%%%%%%%%%%%


\chapter{Installing additional genomes on your machine}
\label{chap:install_genomes}

The easiest way to install genomes on your machine is to download them
from the main \RSAT server, as indicated in the Chapter ``Downloading
genomes'' (Chap.~\ref{downloading_genomes} of the installation guide).

In some cases, you may however wish to install a genome by yourself,
because this genome is not supported on the main \RSAT server. For
this, you can use the programs that we use to install new genomes on
the main \RSAT server.

\section{Adding support for Ensembl genomes}

In addition to the genomes imported and maintained on your local \RSAT
server, the program \program{retrieve-ensembl-seq} allows you to
retrieve sequences for any organism supported in the Ensembl database
(\url{http://ensembl.org}).

For this, you first need to install the Bioperl and Ensembl Perl
libraries (see section \ref{sect:ensembl_libraries}).

\subsection{Handling genomes from Ensembl}

The first step to work with Ensembl genomes is to check the list of
organisms currently supported on their Web server.

\begin{lstlisting}
supported-organisms-ensembl
\end{lstlisting}

You can then get more precise information about a given organism
(build, chromosomes) with the command \program{ensembl-org-info}.

\begin{lstlisting}
ensembl-org-info -org Drosophila_melanogaster
\end{lstlisting}

Sequences can be retrieved from Ensembl with the command
\program{retrieve-ensembl-seq}. 

You can for example retrieve the 2kb sequence upstream of the
transcription start site of the gene \gene{PAX6} of the mouse. 

\begin{lstlisting}
retrieve-ensembl-seq.pl -org Mus_musculus -q PAX6 \
  -type upstream -feattype mrna -from -2000 -to -1 -nogene -rm \
  -alltranscripts -uniqseqs
\end{lstlisting}

Options

\begin{itemize}

\item \option{-type upstream} specifies that we want to collect the
  sequences located upstream of the gene (more procisely, upstream of
  the mRNA).

\item \option{-feattype mrna} indicates that the reference for computing
  coordinates is the mRNA. Since we collect upstream sequences, the
  5'most position of the mRNA has coordinate 0, and upstream sequences
  have negative coordinates. Note that many genes are annotated with
  multiple RNAs for different reasons (alternative splicing,
  alternative transcription start sites). By default, the program will
  return the sequences upstream of each mRNA annotated for the query
  gene.

\item \option{-nogene} clip the sequences to avoid overlapping the next
  upstream gene.

\item \option{-rm} repeat masking (important for pattern
  discovery). Repetitive sequences are replaced by \seq{N} characters.

\end{itemize} 

\section{Installing genomes and variations from \ensembl}
\ref{sect:install_ensembl_genome}

\RSAT includes a series of programs to download and install genomes
from Ensembl.

\begin{enumerate}

\item \program{install-ensembl-genome} is a wrapper enabling to
  autmoatize the download (genome sequences, features, variations) and
  configuration tasks.

\item \program{download-ensembl-genome} downloads the genomics
  sequences and converts them in the raw format required for \RSAT.

\item \program{download-ensembl-features} downloads tab-delimited text
  files describing genomic features (transcripts, CDS, genes, \ldots).

\item \program{download-ensembl-variations} downloads tab-delimited
  text files describing genomic variations (polymorphism).

\end{enumerate}

\subsection{Installing genomes from Ensembl}

The program \program{install-ensembl-genome} manages all the required
steps to download and install a genome (sequence, features, and
optionally variations) from Ensembl to \RSAT.

It performs the following tasks: 
\begin{enumerate}

\item The option \option{-task genome} runs the program
  \program{download-ensembl-genome} to download the complete genomic
  sequence of a given organism from the \ensembl Web site, and formats
  it according to \RSAT requirements (conversion from the original
  fasta sequence file to one file per chromosome, in raw format).

\item The option \option{-task features} runs
  \program{download-ensembl-features} to download the positions and
  descriptions of genomic features (genes, CDS, mRNAs, ...).

\item Optionally, when the option \option{-task variations} is
  activated, run \program{download-ensembl-variations} to download the
  description of genomic variations (polymorphism). Note that
  variations are supported only for a subset of genomes.

\item Update \RSAT configuration (\option{-task config}) to make the
  genome available to other programs in the current \RSAT site. 

\item Run the additional tasks (\option{-task install}) required to
  have a fully functional genome on the local \RSAT site: compute
  genomic statisics (intergenic sizes, \ldots) and background models
  (oligonucleotide and dyad frequencies).

\item With the option \option{-available\_species}, the program
  returns the list species available on the Ensembl server, together
  with their status of availability for the 3 data types (genome
  sequence, features, variations). When this option is called, the
  program does not install any genome.

\end{enumerate}

The detailed description of the program and the list of options can be
obtained with the option \option{-help}.

\begin{lstlisting}
## Get the description of the program + all options
install-ensembl-genome -help
\end{lstlisting}

\subsubsection{Getting the list of available genomes}

Before installing a genome, it is generally a good idea to know which
genomes are available. For this, use the option
\option{-available\_species}.

\begin{lstlisting}
## Retrieve the list of supported species on EnsEMBL
install-ensembl-genome -v 1  -available_species -o available_species_ensembl.tab

## Read the result file
more available_species_ensembl.tab
\end{lstlisting}

\emph{Note:} inter-individual variations are available for a subset
only of the genomes available in \ensembl. The option
\option{-available\_species} indicates, for each species, the
availability (genome, features, variations). Obviously, the programs
to analyse regulatory variations (\program{convert-variations},
\program{retrieve-variation-seq}, \program{variation-scan}) are
working only for the genomes documented with variations.

\subsubsection{Installing a genome from Ensembl}

We can now download and install the complete genomic sequence for the
species of our choice. For the sake of space and time economy, we will
use a small genome for this manual: the budding yeast
\org{Saccharomyces cerevisiae}.

\emph{Beware}: some installation steps take a lot of time. For large
genomes (e.g. Vertebrate organisms), the full installation can thus
take several hours. This should in principle not be a big issue, since
installing a genome is not a daily task, but it is worth knowing that
the whole process requires a continuous connection during several
hours.

\begin{lstlisting}
## Install the genome sequences for a selected organism
install-ensembl-genome -v 2 -species Saccharomyces_cerevisiae
\end{lstlisting}

This command will automatically run all the installation tasks
described above, except the installation of variations (see
Section~\ref{sect:download_ensembl_variations}).

\subsection{Installing genomes from EnsemblGenomes}

The historical \ensembl project \urlref{http://www.ensembl.org/}
was focused on vertebrate genomes + a few model organisms
(\org{Saccharomyces cerevisiae}, \org{Drosophila melanogaster},
\ldots).

A more recent project called \ensemblgenomes
\urlref{http://ensemblgenomes.org/} extends the \ensembl project to a
wider taxonomic range (in Oct 2014, there are >15,000 genomes
available at EnsemblGenomes, where as Ensembl only provides 69
genomes).

The program \program{install-ensembl-genome} supports the installation
of genomes from \ensembl as well as \ensemblgenomes. By default, it
opens a connection to the historical \ensembl database, but the option
\option{-ensembl\_genomes} enables to install genomes from the new
project \ensemblgenomes.

\begin{lstlisting}
## Get the list of available species from the extended project
## EnsemblGenomes
install-ensembl-genome -v 2 -available_species -EnsemblGenomes \
   -o available_species_at_EnsemblGenome.txt
\end{lstlisting}

You can then identify your genome of interest in the file
\file{available\_species\_at\_EnsemblGenome.txt}, and start the
installation (don't forget the option \option{-ensembl\_genomes}.

\begin{lstlisting}
## Install Escherichia coli (strain K12 MG1665) from EnsemblGenomes
install-ensembl-genome -v 2 -EnsemblGenomes \
   -species Escherichia_coli_str_k_12_substr_mg1655
\end{lstlisting}


\subsection{Downloading variations}
\label{sect:download_ensembl_variations}

The program \program{download-ensembl-variations} downloads variations
from the \ensembl Web site, and installs it on the local \RSAT
site. 

This program relies on \program{wget}, which must be installed
beforehand on your computer.

\begin{lstlisting}
## Retrieve the list of supported species in the EnsEMBL variation database
download-ensembl-variations -v 1  -available_species
\end{lstlisting}

We can now download all the variations available for the yeast.

\begin{lstlisting}
## Download all variations for a selected organism on your server
download-ensembl-variations -v 1 -species Saccharomyces_cerevisiae
\end{lstlisting}


\section{Importing genomes from NCBI BioProject}

Tne BioProject database hosts the results of genome sequencing and
transcriptome projects. 

\begin{enumerate}
\item Open a connection to the Bioproject Web site \\
  \url{http://www.ncbi.nlm.nih.gov/bioproject}

\item Enter a query to select the organism of interest.
  E.g. \texttt{ostreococcus+tauri[orgn]}

\item If the organism genome has been sequenced, you should see a
  title ``Genome Sequencing Projects'' in the record. Find the
  relevant project and open the link.

  For example, for
  \org{Ostreococcus tauri}, the most relevant project is PRJNA51609 \\
  \url{http://www.ncbi.nlm.nih.gov/bioproject/51609}

\item Take note of the \option{Accesssion} of this genome project:
  since a same organism might have been sequenced several times, it
  will be useful to include this Accession in the suffix of the name
  of the file fo be downloaded.

\item On the left side of the page, under \option{Related
    information}, click the link ``\option{Nucleotide genomic
    data}''. This will display a list of Genbank entries (one per contig).

\item \textbf{Important:} we recommend to create one separate
  directory per organism, and to name this directory according to the
  organism name followed by the genome project Accession number. For
  example, for \org{Ostreococcus tauri}, the folder name would be
  \file{Ostreococcus\_tauri\_PRJNA51609}.

  This convention will facilitate the further steps of installation,
  in particular the parsing of genbank-formatted files with the
  program \program{parse-genbank.pl}.

\item In the top corner of the page, click on the \option{Send to}
  link and activate the following options.

  \option{Send to > File >  Genbank full > Create file} 

  Save the file in the organism-specific directory described in the
  previous step.

\item You can now parse the genome with the program
  \program{parse-genbank.pl}. Note that \program{parse-genbank.pl}
  expects files with extension .gbk or .gbk.gz (as in the NCBI genome
  repository), whereas the BioProject genome appends the extension
  \file{.gb}. You should thus use the option \option{-ext gb}.

\begin{lstlisting}
parse-genbank.pl -v 2 -i Ostreococcus_tauri_PRJNA51609 -ext gb
\end{lstlisting}

After parsing, run the program \program{install-organism} with the
following parameters (adapt organism name).

\begin{lstlisting}
install-organism  -v 2 -org Ostreococcus_tauri_PRJNA51609 \
  -task config,phylogeny,start_stop,allup,seq_len_distrib \
  -task genome_segments,upstream_freq,oligos,dyads,protein_freq
\end{lstlisting}


%% \item In 'Project Data', click the 'Genome: 20' link
%%5) now you will be presented with a list of all 20 chromosomes. To get the full sequence + annotation in genbank format, set 'Display' to 'Genbank' (now your browser will take a while to load all the sequences).
%%6) Use the download button to extract the files: Download -> GenBank (Full)
\end{enumerate}


\section{Importing multi-genome alignment files from UCSC}


\subsection{Warning: disk space requirement}

The UCSC multi-genome alignment files occupy a huge disk space. The
alignments of 30 vertebrates onto the mouse genome (mm9 multiz30)
requires 70Gb. If you intend to offer support for multi-genome
alignments, it might be safe to acquire a separate hard drive for this
data.

The complete data set available at UCSC in April 2012 occupies 1Tb in
compressed form, and probably 7 times more once uncompressed. For
efficiency reasons, it is necessary to uncompress these files for
using them with the indexing system of \program{peak-footprints}.

\subsection{Checking supported genomes at UCSC}

As a first step, we will check the list of supported genomes at the
UCSC Genome Browser.

\begin{lstlisting}
supported-organisms-ucsc
\end{lstlisting}

Each genome is assocaited with a short identifier, followed by a
description. For example, several versions of the mouse genome are
currently available.

\begin{small}
\begin{verbatim}
mm10	Mouse Dec. 2011 (GRCm38/mm10) Genome at UCSC
mm9	Mouse July 2007 (NCBI37/mm9) Genome at UCSC
mm8	Mouse Feb. 2006 (NCBI36/mm8) Genome at UCSC
mm7	Mouse Aug. 2005 (NCBI35/mm7) Genome at UCSC
\end{verbatim}
\end{small}

\subsection{Downloading multiz files from UCSC}

Multi-genome alignments at UCSC are generated with the program
\program{multiz}, which produces files in a custom text format called
\concept{maf} for Multi-Alignment file.

We show hereafter the command to download the mm9 version of the mouse
genome, and install it in the proper directory for
\program{peak-footprints} (\file{\$RSAT/data/UCSC\_multiz}).

\begin{lstlisting}
download-ucsc-multiz -v 1 -org mm9
\end{lstlisting}

\emph{Beware: } the download of all the multi-species alignments can
take several hours for one genome.

The program will create the sub-directory for the mm9 genome, download
the coresponding compressed multiz files (files with extension
\file{.maf.gz}), uncompress them, and call \program{peak-footprint}
with specific options in order to create a position index, which will
be further used for fast retrieval of the conserved regions under
peaks.


\section{Installing genomes from  NCBI/Genbank files}

In the section \ref{downloading_genomes}, we saw that the genomes
installed on the main \RSAT server can easily be installed on your
local site. In some cases, you would like to install additional
genomes, which are not published yet, or which are not supported on
the main \RSAT server.

If your genomes are available in Genbank (files .gbk) or EMBL (files
.embl) format, this can be done without too much effort, using the
installation tools of \RSAT. 

The parsing of genomes from their original data sources is however
more tricky than the synchronization from the \RSAT server, so this
procedure should be used only if you need to install a genome that is
not yet supported. 

If this is not your case, you can skip the rest of this section.

\subsection{Organization of the genome files}

In order for a genome to be supported, \RSAT needs to find at least
the following files.

\begin{enumerate}
\item organism description
\item genome sequences
\item feature tables (CDS, mRNA, \ldots)
\item lists of names/synonyms
\end{enumerate}

From these files, a set of additional installation steps will be done
by \RSAT programs in order to compute the frequencies of
oligonucleotides and dyads in upstream sequences.

If you installed \RSAT as specified above, you can have a look at the
organization of a supported genome, for example the yeast
\org{Saccharomyces cerevisiae}.

\begin{footnotesize}
\begin{verbatim}
cd ${RSAT}/public_html/data/genomes/Saccharomyces_cerevisiae/genome
ls -l
\end{verbatim}
\end{footnotesize}

As you see, the folder \file{genome} contains the sequence files and
the tables describing the organism and its features (CDSs, mRNAs,
\ldots). The \RSAT parser exports tables for all the feature types
found in the original genbank file. There are thus a lot of distinct
files, but you should not worry too much, for the two following
reasons:
\begin{enumerate}
\item \RSAT only requires a subset of these files (basically, those
  describing organisms, CDSs, mRNAs, rRNAs and tRNAs).
\item All these files can be generated automatically by \RSAT parsers.
\end{enumerate}

\subsubsection{Organism description} 

The description of the organism is given in two separate files.


\begin{footnotesize}
\begin{verbatim}
cd ${RSAT}/public_html/data/genomes/Saccharomyces_cerevisiae/genome
ls -l organism*.tab

more organism.tab

more organism_names.tab
\end{verbatim}
\end{footnotesize}

\begin{enumerate}
\item \file{organism.tab} specifies the ID of the organism and its
  taxonomy. The ID of an organism is the TAXID defined by the NCBI
  taxonomical database, and its taxonomy is usually parsed from the
  .gbk files (but yo may need to specify it yourself in case it would
  be missing in your own data files).

\item \file{organism\_name.tab} indicates the name of the organism.
\end{enumerate}


\subsubsection{Genome sequence} 

A genome sequence is composed of one or more contigs. A contig is a
contigous sequence, resulting from the assembly of short sequence
fragments obtained during the sequencing. When a genome is completely
sequenced and assembled, each chomosome comes as a single contig. 

In \RSAT, the genome sequence is specified as one separate file per
contig (chromosome) sequence. Each sequence file must be in raw format
(i.e. a text file containing the sequence without any space or
carriage return). 

In addition, the genome directory contains one file indicating the
list of the contig (chromosome) files.

\begin{footnotesize}
\begin{verbatim}
cd $RSAT/data/genomes/Saccharomyces_cerevisiae/genome/

## The list of sequence files
cat contigs.txt

## The sequence files
ls -l *.raw

\end{verbatim}
\end{footnotesize}

\subsubsection{Feature table}

The \file{genome} directory also contains a set of feature tables
giving the basic information about gene locations. Several feature
types (CDS, mRNA, tRNA, rRNA) can be specified in separate files
(\file{cds.tab}, \file{mrna.tab}, \file{trna.tab}, \file{rrna.tab}).

Each feature table is a tab-delimited text file, with one row per
feature (cds, mrna, \ldots) and one column per parameter. The
following information is expected to be found.

\begin{enumerate}

\item Identifier

\item Feature type (e.g. ORF, tRNA, ...)

\item Name

\item Chromosome. This must correspond to one of the sequence
identifiers from the fasta file.

\item Left limit

\item Right limit

\item Strand (D for direct, R for reverse complemet)

\item Description. A one-sentence description of the gene function.

\end{enumerate}


\begin{footnotesize}
\begin{verbatim}
## The feature table
head -30 cds.tab
\end{verbatim}
\end{footnotesize}


\subsubsection{Feature names/synonyms}

Some genes can have several names (synonyms), which are specified in
separate tables.

\begin{enumerate}
\item ID. This must be one identifier found in the feature table
\item Synonym
\item Name priority (primary or alternate)
\end{enumerate}


\begin{footnotesize}
\begin{verbatim}
## View the first row of the file specifying gene names/synonyms
head -30 cds_names.tab
\end{verbatim}
\end{footnotesize}


Multiple synonyms can be given for a gene, by adding several lines with
the same ID in the first column.

\begin{footnotesize}
\begin{verbatim}
## An example of yeast genes with multiple names
grep YFL021W cds_names.tab 
\end{verbatim}
\end{footnotesize}



\subsection{Downloading genomes from NCBI/Genbank}

The normal way to install an organism in \RSAT is to download the
complete genome files from the NCBI
\urlref{ftp://ftp.ncbi.nih.gov/genomes/}, and to parse it with the
program \program{parse-genbank.pl}.

However, rather than downloading genomes directly from the NCBI site,
we will obtain them from a mirror
\urlref{bio-mirror.net/biomirror/ncbigenomes/} which presents two
advantages?

\begin{itemize}
\item Genome files are compressed (gzipped), which strongly reduces
  the transfer and storage volume. 
\item This mirror can be queried by \program{rsync}, which facilitates
  the updates (with the appropriate options, \program{rsync} will only
  download the files which are newer on the server than on your
  computer).
\end{itemize}

\RSAT includes a makefile to download genomes from different sources.
We provide hereafter a protocol to create a download directory in your
account, and download genomes in this directory. Beware, genomes
require a lot of disk space, especially for those of higher
organisms. To avoid filling up your hard drive, we illustrate the protocol
with the smallest procaryote genome to date: \textit{Mycoplasma
  genitamlium}.


\begin{footnotesize}
\begin{verbatim}
## Creating a directory for downloading genomes in your home account
cd $RSAT
mkdir -p downloads
cd downloads

## Creating a link to the makefile which allows you to dowload genomes
ln -s $RSAT/makefiles/downloads.mk ./makefile
\end{verbatim}
\end{footnotesize}

We will now download a small genome from NCBI/Genbank. 

\begin{footnotesize}
\begin{verbatim}
## Downloading one directory from NCBI Genbank
cd $RSAT/downloads/
make one_genbank_dir NCBI_DIR=Bacteria/Mycoplasma_genitalium
\end{verbatim}
\end{footnotesize}

We can now check the list of files that have been downloaded.

\begin{footnotesize}
\begin{verbatim}
## Downloading one directory from NCBI Genbank
cd $RSAT/downloads/
ls -l ftp.ncbi.nih.gov/genomes/Bacteria/Mycoplasma_genitalium/
\end{verbatim}
\end{footnotesize}

\RSAT parsers only use the files with extension \file{.gbk.gz}.

You can also adapt the commande to download (for example) all the
Fungal genomes in a single run.

\begin{footnotesize}
\begin{verbatim}
## Downloading one directory from NCBI Genbank
cd $RSAT/downloads/
make one_ncbi_dir NCBI_DIR=Fungi
\end{verbatim}
\end{footnotesize}

You can do the same for Bacteria, of for the whole NCBI genome
repository, but this requires sveral Gb of free disck space.

\subsection{Parsing a genome from NCBI/Genbank}

The program \program{parse-genbank.pl} extract genome information
(sequence, gene location, ...) from Genbank flat files, and exports
the result in a set of tab-delimited files.

\begin{footnotesize}
\begin{verbatim}
parse-genbank.pl -v 1 \
    -i $RSAT/downloads/ftp.ncbi.nih.gov/genomes/Bacteria/Mycoplasma_genitalium 
\end{verbatim}
\end{footnotesize}

\subsection{Parsing a genome from the Broad institute (MIT)}

The website \url{http://www.broad.mit.edu/} offers a large collection of
genomes that are not available on the NCBI website. We wrote a specific parser 
for the Broad files.

To this, download the following files for the organism of interest : the supercontig file, the protein sequences and the annotation file in the GTF format.

These files contain sometimes too much information that shoud be removed. 

This is an example of the beginning of the fasta file containing the protein traduction. In this file, we should remove 
everything that follows the protein name.
\begin{footnotesize}
\begin{verbatim}
>LELG_00001 | Lodderomyces elongisporus hypothetical protein (translation) (1085 aa)
MKYDTAAQLSLINPQTLKGLPIKPFPLSQPVFVQGVNNDTKAITQGVFLDVTVHFISLPA
ILYLHEQIPVGQVLLGLPFQDAHKLSIGFTDDGDKRELRFRANGNIHKFPIRYDGDSNYH
IDSFPTVQVSQTVVIPPLSEMLRPAFTGSRASEDDIRYFVDQCAEVSDVFYIKGGDPGRL
\end{verbatim}
\end{footnotesize}
This is an example of the beginning of the fasta file containing the contigs. In this file, we should remove 
everything that follows the name of the contig. 
\begin{footnotesize}
\begin{verbatim}
>supercontig_1.1 of Lodderomyces elongisporus
AAGAGCATCGGGCAAATGATGTTTTTCAGTCCATCAATGTGATGGATCTGATAGTTGAAG
GTCCTGATGAAGTTCAACCATTTGTAAACCCGATTTACAAAGTGTGAATTATCGAGTGGT
TTATTCATCACAAGGACAAGAGCTTTGTTGGTTGACAGAGATGTTTTGCAGAAAGCCCTT
AAGGATGGTATTGCCTTGTTCAAGAAGAAACCAGTTGTTACTGAAGTAAATCTGACGACC
\end{verbatim}
\end{footnotesize}


This is an example of the beginning of the GTF file containing the contigs annotation. We should rename the contig
name so that it corresponds to the fasta file of contig. To this, we will remove the text in the name of the contig (only keep the supercontig number)
and add a prefix.
\begin{footnotesize}
\begin{verbatim}
supercont1.1%20of%20Lodderomyces%20elongisporus	LE1_FINAL_GENECALL	start_codon	322	324	.	+	0	gene_id "LELG_00001"; transcript_id "LELT_00001";
supercont1.1%20of%20Lodderomyces%20elongisporus	LE1_FINAL_GENECALL	stop_codon	3574	3576	.	+	0	gene_id "LELG_00001"; transcript_id "LELT_00001";
supercont1.1%20of%20Lodderomyces%20elongisporus	LE1_FINAL_GENECALL	exon	322	3576	.	+	.	gene_id "LELG_00001"; transcript_id "LELT_00001";
supercont1.1%20of%20Lodderomyces%20elongisporus	LE1_FINAL_GENECALL	CDS	322	3573	.	+	0	gene_id "LELG_00001"; transcript_id "LELT_00001";
\end{verbatim}
\end{footnotesize}
We use the parse \program{parse-broad-mit}.
\begin{footnotesize}
\begin{verbatim}
parse-broad-mit.pl -taxid 36914 -org Lodderomyces_elongisporus \
		   -nuc_seq lodderomyces_elongisporus_1_supercontigs.fasta \
		   -gtf lodderomyces_elongisporus_1_transcripts.gtf \
		   -gtf_remove 'supercont' \
		   -gtf_remove '%20of%20Lodderomyces%20elongisporus' \
		   -contig_prefix LELG_ -nuc_remove supercontig_ \
		   -nuc_remove ' of Lodderomyces elongisporus' \
		   -aa lodderomyces_elongisporus_1_proteins.fasta -aa_remove ' .*'
\end{verbatim}
\end{footnotesize}

This will create the raw files, the feature files and the protein sequence file.

\subsection{Updating the configuration file}

After having parsed the genome, you need to perform one additional
operation in order for \RSAT to be aware of the new organism: update
the configuration file.

\begin{footnotesize}
\begin{verbatim}
install-organism -v 1 -org Mycoplasma_genitalium -task config

## Check the last lines of the configuration file
tail -15 $RSAT/data/supported_organisms.pl
\end{verbatim}
\end{footnotesize}

From now on, the genome is considered as supported on your local \RSAT
site. You can check this with the command \program{supported-organisms}. 

\subsection{Checking the start and stop codon composition}

Once the organism is found in your configuration, you need to check
whether sequences are retrieved properly. A good test for this is to
retrieve all the start codons, and check whether they are made of the
expected codons (mainly ATG, plus some alternative start codons like
GTG or TTG for bacteria).

The script \program{install-organism} allows you to perform some
additional steps for checking the conformity of the newly installed
genome. For example, we will compute the frequencies of all the start
and stop codons, i order to check that gene locations were corectly
parsed.

\begin{footnotesize}
\begin{verbatim}
install-organism -v 1 -org Mycoplasma_genitalium -task start_stop

ls -l $RSAT/data/genomes/Mycoplasma_genitalium/genome/*start*

ls -l $RSAT/data/genomes/Mycoplasma_genitalium/genome/*stop*
\end{verbatim}
\end{footnotesize}


The stop codons should be TAA, TAG or TGA, for any organism. For
eucaryotes, all start codons should be ATG. For some procaryotes,
alternative start codons (GTG, TGG) are found with some
genome-specific frequency.

\begin{footnotesize}
\begin{verbatim}
cd $RSAT/data/genomes/Mycoplasma_genitalium/genome/

## A file containing all the start codons
more Mycoplasma_genitalium_start_codons.wc

## A file with trinucleotide frequencies in all start codons
more Mycoplasma_genitalium_start_codon_frequencies

## A file containing all the stop codons
more Mycoplasma_genitalium_stop_codons.wc

## A file with trinucleotide frequencies in all stop codons
more Mycoplasma_genitalium_stop_codon_frequencies
\end{verbatim}
\end{footnotesize}

\subsection{Calibrating oligonucleotide and dyad frequencies with \program{install-organisms}}

The programs \program{oligo-analysis} and \program{dyad-analysis}
require calibrated frequencies for the background models. These
frequencies are calculated automatically with
\program{install-organism}.

\begin{footnotesize}
\begin{verbatim}
install-organism -v 1 -org Debaryomyces_hansenii \
    -task allup,oligos,dyads,upstream_freq,protein_freq
\end{verbatim}
\end{footnotesize}

\textbf{Warning: } this task may require several hours of computation,
depending on the genome size. For the \RSAT server, we use a PC
cluster to regularly install and update genomes. As the task
\textit{allup}, is a prerequisite for the computation of all
oligonucleotide and dyad frequencies, it should be run directly on the
main server before computing oligo and dyad frequencies on the nodes
of the cluster. We will thus proceed in two steps. Note that this
requires a PC cluster and a proper configuration of the batch
management program.

\begin{footnotesize}
\begin{verbatim}
## Retrieve all upstream sequences
## Executed directly on the server
install-organism -v 1 -org Debaryomyces_hansenii \
    -task allup

## Launch a batch queue for computing all oligo and dyad frequencies
## Executed on the nodes of a cluster
install-organism -v 1 -org Debaryomyces_hansenii \
    -task oligos,dyads,upstream_freq,protein_freq -batch
\end{verbatim}
\end{footnotesize}

\subsection{Installing a genome in your own account}

In some cases, you might want to install a genome in your own account
rather than in the \RSAT folder, in order to be able to analyze this
genome before putting it in public access.


In this chapter, we explain how to add support for an organism on your
local configuration of \RSAT. This assumes that you have the complete
sequence of a genome, and a table describing the predicted location of
genes.

First, prepare a directory where you will store the data for your
organism. For example:

\begin{footnotesize}
\begin{verbatim}
mkdir -p $HOME/rsat-add/data/Mygenus_myspecies/
\end{verbatim}
\end{footnotesize}


One you have this information, start the program
\program{install-organism}. You will be asked to enter the information
needed for genome installation.

\subsubsection{Updating your local configuration}


\begin{itemize}
\item Modify the local config file

\item You need to define an environment variable called
  RSA\_LOCAL\_CONFIG, containing the full path of the local config
  file.

\end{itemize}


\section{Installing genomes from EMBL files}

\RSAT also includes a script \program{parse-embl.pl} to parse genomes
from EMBL files. However, for practicaly reasons we generally parse
genomes from the NCBI genome repository. Thus, unless you have a
specific reason to parse EMBL files, you can skip this section.

The program \program{parse-embl.pl} reads flat files in EMBL format,
and exports genome sequences and features (CDS, tRNA, ...) in
different files.

As an example, we can parse a yeast genome sequenced by the
``Genolevures'' project
\urlref{http://natchaug.labri.u-bordeaux.fr/Genolevures/download.php}.

Let us assume that you want to parse the genome of the species
\textit{Debaryomyces hansenii}.

Before parsing, you need to download the files in your account, 

\begin{itemize}
\item Create a directory for storing the EMBL files. The last level of
  the directory should be the name of the organism, where spaces are
  replaced by underscores. Let us assume that you store them in
  the directory \file{\$RSAT/downloads/Debaryomyces\_hansenii}.

\item Download all the EMBL file for the selected organism. Save each
  name under its original name (the contig ID), followed by the
  extension \texttt{.embl})

\end{itemize}

We will check the content of this directory.

\begin{footnotesize}
\begin{verbatim}
ls -1 $RSAT/downloads/Debaryomyces_hansenii
\end{verbatim}
\end{footnotesize}

On my computer, it gives the following result

\begin{footnotesize}
\begin{verbatim}
CR382133.embl
CR382134.embl
CR382135.embl
CR382136.embl
CR382137.embl
CR382138.embl
CR382139.embl
\end{verbatim}
\end{footnotesize}

The following instruction will parse this genome.

\begin{footnotesize}
\begin{verbatim}
parse-embl.pl -v 1 -i  $RSAT/downloads/Debaryomyces_hansenii
\end{verbatim}
\end{footnotesize}

If you do not specify the output directory, a directory is
automatically created by combining the current date and the organism
name.  The verbose messages will indicate you the path of this
directory, something like
\file{\$HOME/parsed\_data/embl/20050309/Debaryomyces\_hanseni}.

You can now perform all the steps above (updating the config file,
installing oligo- and dyad frequencies, \ldots) as for genomes parsed
from NCBI.



\subsubsection{Installing a genome in the main \RSAT directory}

Once the genome has been parsed, the simplest way to make it available
 for all the users is to install it in the \RSAT genome directory. You
 can already check the genomes installed in this directory.

\begin{footnotesize}
\begin{verbatim}
ls -1 $RSAT/data/genomes/
\end{verbatim}
\end{footnotesize}

There is one subdirectory per organism. For example, the yeast data is
 in \file{\$RSAT/data/genomes/Saccharomyces\_cerevisiae/}. This
 directory is further subdivided in folders: \file{genome} and
 \file{oligo-frequencies}.

We will now create a directory to store data about
 Debaryomyces\_hansenii, and transfer the newly parsed genome in this
 directory.

\begin{footnotesize}
\begin{verbatim}
## Create the directory
mkdir -p $RSAT/data/genomes/Debaryomyces_hansenii/genome

## Transfer the data in this directory
mv $HOME/parsed_data/embl/20050309/Debaryomyces_hanseni/* \
  $RSAT/data/genomes/Debaryomyces_hansenii/genome

## Check the transfer
ls -ltr $RSAT/data/genomes/Debaryomyces_hansenii/genome
\end{verbatim}
\end{footnotesize}


% Bibliography is not working yet. I have to debug
\bibliographystyle{plain}
\bibliography{rsat_bibliography}

\end{document}

%%%%%%%%%%%%%%%%%%%%%%%%%%%%%%%%%%%%%%%%%%%%%%%%%%%%%%%%%%%%%%%%

