%%%%%%%%%%%%%%%%%%%%%%%%%%%%%%%%%%%%%%%%%%%%%%%%%%%%%%%%%%%%%%%%
%
% Installation guide for regulatory Sequence Analysis Tools
%
%%%%%%%%%%%%%%%%%%%%%%%%%%%%%%%%%%%%%%%%%%%%%%%%%%%%%%%%%%%%%%%%

\documentclass{article}
%\documentstyle[makeidx]{book}
\makeindex
%\usepackage{color}
\usepackage[usenames]{color}
\usepackage{times}
\usepackage{graphics}
\usepackage{latexsym}
\usepackage{makeidx}


%%%%%%%%%%%%%%%%%%%%%%%%%%%%%%%%%%%%%%%%%%%%%%%%%%%%%%%%%%%%%%%%
%%%%%%%%%%%%%%%%%%%%%%%%%%% commands %%%%%%%%%%%%%%%%%%%%%%%%%%%
\newcommand{\tbw}{\textbf{TO BE WRITTEN}}
\newcommand{\RSAT}{\textbf{\textit{RSAT}}}
\newcommand{\file}[1]{\textit{#1}}
\newcommand{\concept}[1]{\index{#1}\textsl{#1}}
\newcommand{\command}[1]{\begin{footnotesize}\begin{quote}\textcolor{Blue}{\texttt{#1}}\end{quote}\end{footnotesize}}
\newcommand{\result}[1]{\begin{footnotesize}\begin{quote}\textcolor{OliveGreen}{\texttt{#1}}\end{quote}\end{footnotesize}}
\newcommand{\program}[1]{\textbf{\textsl{#1}}}
\newcommand{\option}[1]{\texttt{#1}}
\newcommand{\email}[1]{\textit{#1}}

\newcommand{\address}[1]{\small{#1}}
\newcommand{\org}[1]{\textit{#1}}
\newcommand{\gene}[1]{\textit{#1}}
\newcommand{\seq}[1]{\texttt{#1}}

\newcommand{\url}[1]{\textit{#1}}
\newcommand{\urlref}[1]{\footnote{\textit{#1}}}

\newcommand{\scmbb}{
	Service de Conformation des Macromol\'{e}cules Biologiques et de Bioinformatique, \\
	Universit\'{e} Libre de Bruxelles, \\
	Campus Plaine, CP 263, Boulevard du Triomphe, B-1050 Bruxelles, Belgium. \\
	Tel: +32 2 650 2013 - Fax: +32 2 650 5425
}

%%%%%%%%%%%%%%%%%%%%%%%%%%%%%%%%%%%%%%%%%%%%%%%%%%%%%%%%%%%%%%%%
%%%%%%%%%%%%%%%%%%%%%%%%% environments %%%%%%%%%%%%%%%%%%%%%%%%%

\def\ex@mple{example}
\def\ex@rcise{exercise}
\def\theoremfont{\ifx\@currenvir\ex@mple
	\def\@thmfont{\rm} \else
	\ifx\@currenvir\ex@rcise
	\def\@thmfont{\rm} \else
	\def\@thmfont{\it}\fi\fi}
\newtheorem{exercise}{Exercise}[chapter]
\newtheorem{example}{Example}[chapter]




\begin{document}

\title{Regulatory Sequence Analysis Tools \\
Installation guide}

\author{
	Jacques van Helden \\
	\email{jvanheld@ucmb.ulb.ac.be} \\
	\scmb 
}


%\address{\scmb}

\maketitle

%\newpage
%\tableofcontents
%\newpage

\section{Description}

This documents contain the installation guide for the package
\textbf{Regulatory Sequence Analysis Tools (RSAT)}.

In the current release, the version is limited to the perl scripts,
and does not contain the web interface.


\section{Installation}

\begin{enumerate}

\item Uncompress the archive containing the programs. 
\begin{verbatim}
tar -xpzf rsat_yyyymmdd.tgz
\end{verbatim}

where yyyymmdd stands for the version number (delivery date).

\end{enumerate}

\section{Adding rsa-tools to your path}

\begin{enumerate}

\item Create an environment variable named RSAT and containing the
path of rsa-tools. For example, assuming rsa-tools have been installed
in the directory \texttt{/usr/local/rsa-tools}, and your shell is bash
:

\begin{verbatim}
export RSAT=/usr/local/rsa-tools
\end{verbatim}

\item add the path of rsa-tools/perl-scripts to your path.
\begin{verbatim}
export PATH=${PATH}:${RSAT}/perl-scripts
\end{verbatim}

\end{enumerate}

\section{Testing the installation}

From now on, you should be able to use the perl scripts from the
command line. To test this, run: 

\begin{verbatim}
random-seq -help
\end{verbatim}

This should display the on-line help for the random sequence
generator. 

\begin{verbatim}
random-seq -l 200 -r 4 -a a:t 0.3 c:g 0.2
\end{verbatim}

Should generate a random sequence.

We will now testif the genomes are correctly installed. You will
obtain the list of supported organisms with the command:

\begin{verbatim}
retrieve-seq -help
\end{verbatim}

Select an organism (say \organism{Saccharomyces cerevisiae}), and
retrieve all the start codons with the following options :

\begin{verbatim}
retrieve-seq -org Saccharomyces_cerevisiae \
        -type upstream -from 0 -to +2 -all \
        -format wc -nocomment 
\end{verbatim}

This should return a set of 3 bp sequences, mostly ATG (in the case of
\organism{Saccharomyces cerevisiae} at least)

\end{document}

%%%%%%%%%%%%%%%%%%%%%%%%%%%%%%%%%%%%%%%%%%%%%%%%%%%%%%%%%%%%%%%%

