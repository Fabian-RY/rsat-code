\section{Installing genomes and variations from \ensembl}
\ref{sect:install_ensembl_genome}

\RSAT includes a series of programs to download and install genomes
from Ensembl.

\begin{enumerate}

\item \program{install-ensembl-genome} is a wrapper enabling to
  autmoatize the download (genome sequences, features, variations) and
  configuration tasks.

\item \program{download-ensembl-genome} downloads the genomics
  sequences and converts them in the raw format required for \RSAT.

\item \program{download-ensembl-features} downloads tab-delimited text
  files describing genomic features (transcripts, CDS, genes, \ldots).

\item \program{download-ensembl-variations} downloads tab-delimited
  text files describing genomic variations (polymorphism).

\end{enumerate}

\subsection{Installing genomes from Ensembl}

The program \program{install-ensembl-genome} manages all the required
steps to download and install a genome (sequence, features, and
optionally variations) from Ensembl to \RSAT.

It performs the following tasks: 
\begin{enumerate}

\item The option \option{-task genome} runs the program
  \program{download-ensembl-genome} to download the complete genomic
  sequence of a given organism from the \ensembl Web site, and formats
  it according to \RSAT requirements (conversion from the original
  fasta sequence file to one file per chromosome, in raw format).

\item The option \option{-task features} runs
  \program{download-ensembl-features} to download the positions and
  descriptions of genomic features (genes, CDS, mRNAs, ...).

\item Optionally, when the option \option{-task variations} is
  activated, run \program{download-ensembl-variations} to download the
  description of genomic variations (polymorphism). Note that
  variations are supported only for a subset of genomes.

\item Update \RSAT configuration (\option{-task config}) to make the
  genome available to other programs in the current \RSAT site. 

\item Run the additional tasks (\option{-task install}) required to
  have a fully functional genome on the local \RSAT site: compute
  genomic statisics (intergenic sizes, \ldots) and background models
  (oligonucleotide and dyad frequencies).

\item With the option \option{-available\_species}, the program
  returns the list species available on the Ensembl server, together
  with their status of availability for the 3 data types (genome
  sequence, features, variations). When this option is called, the
  program does not install any genome.

\end{enumerate}

The detailed description of the program and the list of options can be
obtained with the option \option{-help}.

\begin{lstlisting}
## Get the description of the program + all options
install-ensembl-genome -help
\end{lstlisting}

\subsubsection{Getting the list of available genomes}

Before installing a genome, it is generally a good idea to know which
genomes are available. For this, use the option
\option{-available\_species}.

\begin{lstlisting}
## Retrieve the list of supported species on EnsEMBL
install-ensembl-genome -v 1  -available_species -o available_species_ensembl.tab

## Read the result file
more available_species_ensembl.tab
\end{lstlisting}

\emph{Note:} inter-individual variations are available for a subset
only of the genomes available in \ensembl. The option
\option{-available\_species} indicates, for each species, the
availability (genome, features, variations). Obviously, the programs
to analyse regulatory variations (\program{convert-variations},
\program{retrieve-variation-seq}, \program{variation-scan}) are
working only for the genomes documented with variations.

\subsubsection{Installing a genome from Ensembl}

We can now download and install the complete genomic sequence for the
species of our choice. For the sake of space and time economy, we will
use a small genome for this manual: the budding yeast
\org{Saccharomyces cerevisiae}.

\emph{Beware}: some installation steps take a lot of time. For large
genomes (e.g. Vertebrate organisms), the full installation can thus
take several hours. This should in principle not be a big issue, since
installing a genome is not a daily task, but it is worth knowing that
the whole process requires a continuous connection during several
hours.

\begin{lstlisting}
## Install the genome sequences for a selected organism
install-ensembl-genome -v 2 -species Saccharomyces_cerevisiae
\end{lstlisting}

This command will automatically run all the installation tasks
described above, except the installation of variations (see
Section~\ref{sect:download_ensembl_variations}).

\subsection{Installing genomes from EnsemblGenomes}

The historical \ensembl project \urlref{http://www.ensembl.org/}
was focused on vertebrate genomes + a few model organisms
(\org{Saccharomyces cerevisiae}, \org{Drosophila melanogaster},
\ldots).

A more recent project called \ensemblgenomes
\urlref{http://ensemblgenomes.org/} extends the \ensembl project to a
wider taxonomic range (in Oct 2014, there are >15,000 genomes
available at EnsemblGenomes, where as Ensembl only provides 69
genomes).

The program \program{install-ensembl-genome} supports the installation
of genomes from \ensembl as well as \ensemblgenomes. By default, it
opens a connection to the historical \ensembl database, but the option
\option{-db ensemblgenomes} enables to install genomes from the new
project \ensemblgenomes.

\begin{lstlisting}
## Get the list of available species from the extended project
## EnsemblGenomes
install-ensembl-genome -v 2 -available_species -db ensemblgenomes \
   -o available_species_at_EnsemblGenome.txt
\end{lstlisting}

You can then identify your genome of interest in the file
\file{available\_species\_at\_EnsemblGenome.txt}, and start the
installation (don't forget the option \option{-db ensemblgenomes}.

\begin{lstlisting}
## Install Escherichia coli (strain K12 MG1665) from EnsemblGenomes
install-ensembl-genome -v 2 -db ensemblgenomes \
   -species Escherichia_coli_str_k_12_substr_mg1655
\end{lstlisting}


\subsection{Downloading variations}
\label{sect:download_ensembl_variations}

The program \program{download-ensembl-variations} downloads variations
from the \ensembl Web site, and installs it on the local \RSAT
site. 

This program relies on \program{wget}, which must be installed
beforehand on your computer.

\begin{lstlisting}
## Retrieve the list of supported species in the EnsEMBL variation database
download-ensembl-variations -v 1  -available_species
\end{lstlisting}

We can now download all the variations available for the yeast.

\begin{lstlisting}
## Download all variations for a selected organism on your server
download-ensembl-variations -v 1 -species Saccharomyces_cerevisiae
\end{lstlisting}
