
%%%%%%%%%%%%%%%%%%%%%%%%%%%%%%%%%%%%%%%%%%%%%%%%%%%%%%%%%%%%%%%%
%%%% Introduction
%%%%%%%%%%%%%%%%%%%%%%%%%%%%%%%%%%%%%%%%%%%%%%%%%%%%%%%%%%%%%%%%
\chapter{Introduction}

This tutorial aims at introducing how to use Regulatory Sequence
Analysis Tools (\RSAT) directly from the Unix shell.

\RSAT is a package combining a series of specialized programs for the
detection of regulatory signals in non-coding sequences. A variety of
tasks can be performed: retrieval of upstream or downstream sequences,
pattern discovery, pattern matching, graphical representation of
regulatory regions, sequence conversions, \ldots.

A web interface has been developed for the most common tools, and is
freely available for academic users.

\url{http://rsat.scmbb.ulb.ac.be/rsat/}

All the programs in \RSAT can also be used directly from the Unix
shell. The shell access is less intuitive than the web interface, but
it allows to perform more complex analyses, and it is very convenient
for automatizing repetitive tasks.

This tutorial was written by Jacques van Helden
(\email{Jacques.van.Helden@ulb.ac.be}).  Unless otherwise specified,
the programs presented here were written by Jacques van Helden.

%%%%%%%%%%%%%%%%%%%%%%%%%%%%%%%%%%%%%%%%%%%%%%%%%%%%%%%%%%%%%%%%
%% Prerequisites
%%%%%%%%%%%%%%%%%%%%%%%%%%%%%%%%%%%%%%%%%%%%%%%%%%%%%%%%%%%%%%%%
\section{Prerequisites}

This program requires a basic knowledge of the Unix
environment. Before starting you should be familiar with the concepts
of Unix shell, directory, file, path.

%%%%%%%%%%%%%%%%%%%%%%%%%%%%%%%%%%%%%%%%%%%%%%%%%%%%%%%%%%%%%%%%
%%%% Accessing the programs
%%%%%%%%%%%%%%%%%%%%%%%%%%%%%%%%%%%%%%%%%%%%%%%%%%%%%%%%%%%%%%%%
\section{Configuration}

In order to use the command-line version of \RSAT, you first need an
account on a Unix machine where \RSAT has been installed, and you
should know the directory where the tools have been installed (if you
don't know, ask assistance to your system administrator).

For this tutorial, let us assume that \RSAT is installed in the
directory \texttt{/home/rsat/rsa-tools}

\subsection{\RSAT environment and path}

Before starting to use the tools, you need to define an environment
variable (\texttt{RSAT}), and to add some directories to your path.

\begin{enumerate}

\item Open a terminal and login in your account.

\item check your shell environment by typing the following command.
  \command{echo \$SHELL}
  The answer should be something like
  \result{/sbin/bash} 
  or
  \result{/bin/tcsh}.

\item If your default shell is \textbf{tcsh}, type the following
  commands (you probably need to update the firts command to specify
  the RSAT path of your machine.

{\color{blue} \begin{footnotesize}
\begin{verbatim}
setenv RSAT /home/rsat/rsa-tools
set path=($path $RSAT/perl-scripts)
set path=($path $RSAT/perl-scripts/parsers)
set path=($path $RSAT/bin/)
rehash
\end{verbatim} \end{footnotesize}
}

\item If your default shell is \textbf{bash}, the commands are
  slightly different.

{\color{blue} \begin{footnotesize}
\begin{verbatim}
export RSAT=/home/rsat/rsa-tools
export path=($path $RSAT/perl-scripts)
export path=($path $RSAT/perl-scripts/parsers)
export path=($path $RSAT/bin/)
rehash
\end{verbatim} \end{footnotesize}
}

If you are using yet a different shell, you might need a slightly
different command to obtain the same result. See you system manager in
case of doubt.

\end{enumerate}

\subsection{Checking the RSAT path}

 The previous step should have included all the \RSAT programs in
your path.  To check if it worked, just type:

\command{random-seq -l 350}

If your configuration is correct, this command should return a random
sequence of 350 nucleotides.

You are now able to use any program from the \RSAT package, untill you
quit your session. It is however not very convenient to set the path
manually each time you open a new connection. You can modify your
default configuration by including the above commands in the file
\file{.cshrc} (in tcsh) or \file{.bashrc} (in bash) which should be
found at the root of your home directory. If you don't know how to
modify this file, see the system administrator.

\subsection{Creating a directory for this tutorial}

During this tutorial, we wll frequently save data and result files. I
propose to create a dedicated directory for these files. In the
following chapters, we will assume that this directory is named
\file{practical\_rsat} and is located at the root of your personnal
account (everyone is of course allowed to change the name and location
of this directory).

To create the directory for the tutorials, you can simply type the
following commands.

{\color{blue} \begin{footnotesize}
\begin{verbatim}
cd $HOME ## Go to your home directory
mkdir -p practical_rsat ## Create the directory for the tutorial
cd  practical_rsat ## Go to this directory
pwd ## Check the path of your directory
\end{verbatim} \end{footnotesize}
}

From now on, we will assume that all the exercises are executed
from this directory.

%%%%%%%%%%%%%%%%%%%%%%%%%%%%%%%%%%%%%%%%%%%%%%%%%%%%%%%%%%%%%%%%
%%%% Wargning
%%%%%%%%%%%%%%%%%%%%%%%%%%%%%%%%%%%%%%%%%%%%%%%%%%%%%%%%%%%%%%%%

\section{Warning}

This tutorial is under construction. Some sections are still to be
written, and only appear as a title without any further text. The
tutorial will be progressively completed. We provided it as it is.
