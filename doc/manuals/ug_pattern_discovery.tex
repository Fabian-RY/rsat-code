%%%%%%%%%%%%%%%%%%%%%%%%%%%%%%%%%%%%%%%%%%%%%%%%%%%%%%%%%%%%%%%%
%%%% PATTERN DISCOVERY
%%%%%%%%%%%%%%%%%%%%%%%%%%%%%%%%%%%%%%%%%%%%%%%%%%%%%%%%%%%%%%%%
\chapter{Pattern discovery}

In a pattern discovery problem, you start from a set of functionally
related sequences (e.g.  upstream sequences for a set of co-regulated
genes) and you try to extract motifs (e.g. regulatory elements) that
are characteristic of these sequences.

Several approaches exist, either string-based or
matrix-based. \concept{String-based pattern discovery} is based on an
analysis of the number of occurrences of all possibles words
(\program{oligo-analysis}), or spaced pairs
(\program{dyad-analysis}). The methods for \concept{matrix-based
  pattern discovery} rely on the utlisation of some machine-learning
method (e.g. greedy algorithm, expectation-maxiisation, gibbs
sampling, ...) in order to optimise of some scoring function
(log-likelihood, information,...) which is likely to return
significant motifs.

In this chapter we will mainly focus on string-based approaches, and
illustrate some of their advantages. A further chapter will be
dedicated to matrix-based pattern discovery.

For microbial cis-acting elements, string-based approaches give
excellent results. The main advantages of these methods:

\begin{itemize}
\item[+] Simple to use
\item[+] Deterministic (if you run it repeatedly, you always get the
  same result), in contrast with stochastic optimization methods.
\item[+] Exhaustive : each word or space pair is tested
  independently. Consequently, if a set of sequences contains several
  exceptional motifs, all of them can be detected in a single run. 
\item[+] The tests of significances can be performed on both tails of
  the theoretical distribution, in order to detect either
  over-represented, or under-represented patterns.
\item[+] Fast. 
\item[+] Able to return a negative answer: if no motif is significant,
  the programs return no motif at all. This is particularly important
  to reduce the rate of false positive.
\end{itemize}


An obvious advantage of matrix-based approach is that they provide a
more refined description of motifs presenting a high degree of
degeneracy. However, a general problem of matrix-based approaches is
that it is impossible to analyze all possible position-weight
matrices, and thus one has to use heuristics. There is thus a risk to
miss the global optimum because the program is attracted to local
maxima. Another problem is that there are more parameters to select
(typically, matrix width and expected number of occurrences of the
motif), and their choice drastically affects the quality of the
result.

Basically, I would tend to prefer string-based approaches for any
problem of pattern discovery. On the contrary, matrix-based approaches
are much more sensitive for pattern matching problems (see below). My
preference is thus to combine string-based pattern discovery and
matrix-based pattern matching.

But I am obviously biased because I developed string-based
approaches. An important factor in the success obtained with a program
is to understand precisely its functioning. I thus think that each
user should test different programs, compare them and select the one
that best suits his/her needs.

\section{Requirements}

This part of the tutorial assumes that you already performed the
tutorial about sequence retrieval (above), and that you have the
result files in the current directory. Check with the command:

{\color{Blue} \begin{footnotesize} 
\begin{verbatim}
cd ${HOME}/practical_rsat
ls -1
\end{verbatim} \end{footnotesize}
}


You should see the following file list:
{\color{Blue} \begin{footnotesize} 
\begin{verbatim}
Escherichia_coli_K12_start_codons.wc
Escherichia_coli_K12_stop_codons.wc
PHO_genes.txt
PHO_up800-noorf.fasta
MET_genes.txt
MET_up800-noorf.fasta
his.genes.txt
his.up200.noorf.fasta
\end{verbatim} \end{footnotesize}
}


\section{oligo-analysis}

The program \program{oligo-analysis} is the simplest pattern discovery
program. It counts the number of occurrences of all oligonucleotides
(words) of a given length (typically 6), and calculates the
statistical significance of each word by comparing its observed and
expected occurrences. The program returns words with a significant
level of over-representation.

Despite its simplicity, this program generally returns good results
for groups of co-regulated genes in microbes.

For a first trial, we will simply use the program to count word
occurrences. The application will be to check the start and stop
codons retrieved above. We will then use \program{oligo-analysis} in a
pattern discovery process, to detect over-represented words from the
set of upstream sequences retrieved above (the PHO family).  In a
first time, we will use the appropriate parameters, which have been
optimized for pattern discovery in yeast upstream sequences (van
Helden et al., 1998). We will then use the sub-optimal settings to
illustrate the fact that the success of word-based pattern-discovery
crucially depends on a rigorous statistical approach (choice of the
background model and of the scoring function).

\subsection{Counting word occurrences and frequencies}

Try the following command:

{\color{Blue} \begin{footnotesize} 
\begin{verbatim}
oligo-analysis -i Escherichia_coli_K12_start_codons.wc \
    -format wc -l 3 -1str
\end{verbatim} \end{footnotesize}
}


Call the on-line option description to understand the meaning of the
options you used:

{\color{Blue} \begin{footnotesize} 
\begin{verbatim}
oligo-analysis -help
\end{verbatim} \end{footnotesize}
}


Or, to obtain more details:
{\color{Blue} \begin{footnotesize} 
\begin{verbatim}
oligo-analysis -h
\end{verbatim} \end{footnotesize}
}



You can also ask some more information (verbose) and store the result
in a file:

{\color{Blue} \begin{footnotesize} 
\begin{verbatim}
oligo-analysis -i Escherichia_coli_K12_start_codons.wc \
    -format wc -l 3 -1str \
    -return occ,freq -v 1 \
    -o Escherichia_coli_K12_start_codon_frequencies.tab
\end{verbatim} \end{footnotesize}
}


Read the result file:

{\color{Blue} \begin{footnotesize} 
\begin{verbatim}
more Escherichia_coli_K12_start_codon_frequencies.tab
\end{verbatim} \end{footnotesize}
}


Note the effect of the verbose option (\option{-v 1}). You receive
information about sequence length, number of possible
oligonucleotides, the content of the output columns, ...

\textbf{Exercise:} check the frequencies of \textit{E.coli} stop codons.

\subsection{Pattern discovery in yeast upstream regions}

Try the following command:

{\color{Blue} \begin{footnotesize} 
\begin{verbatim} 
oligo-analysis -i PHO_up800-noorf.fasta -format fasta    \
    -v 1 -l 6 -2str -lth occ_sig 0 -noov \
    -return occ,proba,rank -sort \
    -bg upstream -org Saccharomyces_cerevisiae \
    -o PHO_up800-noorf_6nt-2str-noov_ncf_sig0 
\end{verbatim} \end{footnotesize}
}


Note that the return fields (``occ'', ``proba'', and ``rank'') are
separated by a comma \textit{without} space. Call the on-line help to
understand the meaning of the parameters.

{\color{Blue} \begin{footnotesize} 
\begin{verbatim} 
oligo-analysis -h
\end{verbatim} \end{footnotesize}
}


In the previous analysis, the expected frequency of each word was
estimate on the basis of pre-calibrated frequency tables.  These
tables have been previously calculated (with oligo-analysis) by
counting hexanucleotide frequencies in the whole set of yeast upstream
regions. Our experience is that these frequencies are the optimal
estimator for discovering regulatory elements in upstream sequences of
co-regulated genes.

Look the result file:

{\color{Blue} \begin{footnotesize} 
\begin{verbatim}
more PHO_up800-noorf_6nt-2str-noov_ncf_sig0
\end{verbatim} \end{footnotesize}
}


A few questions:
\begin{enumerate}
\item How many hexanucleotides can be formed with the 4-letter alphabet A,T,G,C ?
\item How many possible oligonucleotides were analysed here ? Is it the
  number you would expect ? Why ?
\item How many patterns have been selected as significant ?
\item By simple visual inspection, can you identify some sequence
  similarities between the selected patterns?
\end{enumerate}

\subsection{Answers}

\begin{enumerate}
\item The number of possible hexanucleotides is $4^6=4,096$. 
\item The result file reports $2,080$ possible oligonucleotides. This
  is due to the fact that the analysis was performed on both
  strands. Each oligonucleotide is thus regrouped with its reverse
  complement.
\item Among the 2080 tested oligonucleotides (+reverse complement), no
  more than $10$ were selected as significantly over-represented.
\item Some pairs of words are mutually overlapping
  (e.g. \seq{ACGTGc} and \seq{cACGTG}).
\end{enumerate}

\subsection{Assembling the patterns}

A separate program, \program{pattern-assembly}, allows to assemble a
list of patterns, in order to group those that overlap mutually. Try:

{\color{Blue} \begin{footnotesize} 
\begin{verbatim}
pattern-assembly -i PHO_up800-noorf_6nt-2str-noov_ncf_sig0 \
    -v  1 -subst 1 \
    -2str -o PHO_up800-noorf_6nt-2str-noov_ncf_sig0.asmb
\end{verbatim} \end{footnotesize}
}



Call the on-line help to have a look at the assembly parameters. 

{\color{Blue} \begin{footnotesize} 
\begin{verbatim}
pattern-assembly -h
\end{verbatim} \end{footnotesize}
}


Let us have a look at the assembled motifs.
 
{\color{Blue} \begin{footnotesize} 
\begin{verbatim}
more PHO_up800-noorf_6nt-2str-noov_ncf_sig0.asmb
\end{verbatim} \end{footnotesize}
}


Should give something llike this (the precise result might be slightly
different depending on the version of the genome).

{\color{OliveGreen} \begin{footnotesize} 
\begin{verbatim}
; pattern-assembly  -i PHO_up800-noorf_6nt-2str-noov_ncf_sig0 -v 1 -subst 1 \\
       -2str -o PHO_up800-noorf_6nt-2str-noov_ncf_sig0.asmb
; Input file    PHO_up800-noorf_6nt-2str-noov_ncf_sig0
; Output file   PHO_up800-noorf_6nt-2str-noov_ncf_sig0.asmb
; Input score column            8
; Output score column           0
; two strand assembly
; max flanking bases            1
; max substitutions             1
; max assembly size             50
; max number of patterns        100
; number of input patterns      10
;

;assembly # 1   seed: acgtgc    13 words        length 
;     alignt         rev_cpl    score
tcgcac......    ......gtgcga    0.40
.cgcacg.....    .....cgtgcg.    2.55
.cccacg.....    .....cgtggg.    1.20
..gcacgt....    ....acgtgc..    4.52
..ccacgt....    ....acgtgg..    1.54
...cacgtg...    ...cacgtg...    1.84
...aacgtg...    ...cacgtt...    0.28
...cacgtt...    ...aacgtg...    0.28
....acgtgc..    ..gcacgt....    4.52
....acgtgg..    ..ccacgt....    1.54
.....cgtgcg.    .cgcacg.....    2.55
.....cgtggg.    .cccacg.....    1.20
......gtgcga    tcgcac......    0.40
tcgcacgtgcga    tcgcacgtgcga    4.52    best consensus

; Isolated patterns: 3
;alignt rev_cpl score
atacgc  gcgtat  0.62    isol
acaggg  ccctgt  0.43    isol
tgcaca  tgtgca  0.01    isol
;Job started 26/10/06 04:00:51 CDT
;Job done    26/10/06 04:00:51 CDT
\end{verbatim} \end{footnotesize}
}

There are two alignments (with two contigs), and two isolated
patterns. Each alignment is made of strongly overlapping patterns. The
first alignment contains the motif CACGTG, which corresponds to the
high affinity binding site for Pho4p, the protein controlling
transcriptional response to Phosphate in yeast. the second alignment
corresponds to the medium affinity binding site for Pho4p. Medium
affinity binding sites have been shown to participate in the
transcriptional response to some PHO genes.

\subsection{Suboptimal settings}

This chapter only aims at emphasizing how crucial is the choice of
appropriate statistical parameters. We saw above that a background
model calibrated on all the yeast upstream sequences gives good
results with the PHO family: despite the simplicity of the algorithm
(counting non-degenerate hexanucleotide occurrences), we were able to
extract a description of the regulatory motif over a larger width than
6 (by pattern assembly), and we got some description of the degeneracy
(the high and low affinity sites).

We will now intentionally try other parameter settings and see how
they affect the quality of the results.

\subsubsection{Equiprobable oligonucleotides}

Let us try the simplest approach: each word is considered
equiprobable. For this, we simply suppress the options \option{-bg
  upstream -org Saccharomyces\_cerevisiae} fom the above commands. We
also omit to specify the output file, so results will immediately
appear on the screen.

{\color{Blue} \begin{footnotesize} 
\begin{verbatim}
 oligo-analysis -i PHO_up800-noorf.fasta -format fasta \ 
    -v 1 -l 6 -2str \
    -return occ,proba -lth occ_sig 0 -sort 
\end{verbatim} \end{footnotesize}
}


Note that
\begin{itemize} 
\item The number of selected motifs is higher (27) than in the previous trial
\item The most significant motifs have nothing to do with Pho4p binding
sites. All these false positives are A-rich motifs (or T-rich, since we
are grouping patterns with their reverse-complement).
\item Two patterns (\seq{acgttt} and \seq{acgtgc}) are selected
which are related to Pho4p binding site. However, they come at the
12th and 14th positions only.
\end{itemize}

You can combine oligo-analysis and pattern-assembly in a single
command, by using the pipe character as below.

{\color{Blue} \begin{footnotesize} 
\begin{verbatim}
oligo-analysis -i PHO_up800-noorf.fasta -format fasta -v 1 \
    -l 6 -2str  -return occ,proba -lth occ_sig 0 -sort \
    | pattern-assembly -2str -sc 7 -subst 1 -v 1 
\end{verbatim} \end{footnotesize}
}


On unix systems, this special character is used to concatenate
commands, i.e. the output of the first command (in this case
oligo-analysis) is not printed to the screen, but is sent as input for
the second command (in this case pattern-assembly).

Note that the most significant patterns are associated to the poly-A
(aaaaaa) contig. The true positive come isolated. Due to the bad
choice of expected frequencies (all hexanucleotides were considered
equiprobable here), regulatory sites were lost within a majority of
false positive, and their description is much less accurate than with
the option \option{-bg upstream}.

\textit{\textbf{Markov chains}}

Another possibility is to use Markov chain models to estimate expected
word frequencies. Try the following commands and compare the
results. None is as good as the option \option{-bg upstream},
but in case one would not have the pre-calibrated non-coding
frequencies (for instance if the organism has not been completely
sequenced), markov chains can provide an interesting approach.

{\color{Blue} \begin{footnotesize} 
\begin{verbatim}
oligo-analysis -v 1 -markov 0 \
    -i PHO_up800-noorf.fasta -format fasta \
    -l 6 -lth occ_sig 0 -sort \
    -2str -return occ,proba \
    | pattern-assembly -2str -sc 7 -subst 1 -v 1

oligo-analysis -v 1 -markov 1 \
    -i PHO_up800-noorf.fasta -format fasta \
    -2str -return occ,proba \
    -l 6 -lth occ_sig 0 -sort \
    | pattern-assembly -2str -sc 7 -subst 1 -v 1
	
oligo-analysis -v 1 -markov 2 \
    -i PHO_up800-noorf.fasta -format fasta \
    -2str -return occ,proba \
    -l 6 -lth occ_sig 0 -sort \
    | pattern-assembly -2str -sc 7 -subst 1 -v 1
	
oligo-analysis -v 1 -markov 3 \
    -i PHO_up800-noorf.fasta -format fasta \
    -2str -return occ,proba \
    -l 6 -lth occ_sig 0 -sort \
    | pattern-assembly -2str -sc 7 -subst 1 -v 1
	
oligo-analysis -v 1 -markov 4 \
    -i PHO_up800-noorf.fasta -format fasta \
    -2str -return occ,proba \
    -l 6 -lth occ_sig 0 -sort \
    | pattern-assembly -2str -sc 7 -subst 1 -v 1
\end{verbatim} \end{footnotesize}
}


\textit{\textbf{Remarks}}
\begin{itemize}
\item 
Markov 0 returns AT-rich patterns with the highest significance, but
the Pho4p high affinity site is described with a good accuracy. The
medium affinity site appears as a single word (acgttt) in the isolated
patterns.
\item 
Markov order 1 returns less AT-rich motifs. The poly-A (aaaaaa) is
however still associated with the highest significance, but comes as
isolated pattern.
\item 
The higher the order of the markov chain, the most stringent are the
conditions. For small sequence sets, selecting a too high order
prevents from selecting any pattern. Most of the patterns are missed
with a Markov chain of order 2, and higher orders don't return any
single significant word.
\end{itemize}

\section{dyad-analysis}



\section{gibbs motif sampler (program developed by Andrew Neuwald)}




\section{consensus (program developed by Jerry Hertz)}

An alternative approach for matrix-based pattern discovery is
\textit{consensus}, a program written by Jerry hertz, an based on a
greedy algorithm. We will see how to extract a profile matrix from
upstream regions of the PHO genes.

\subsection{Getting help}

As for RSAT programs, there are two ways to get help from Jerry Hertz'
proigrams: a detailed manual can be obtained with the option
\option{-h}, and a summary of options with \option{-help}. Try these
options and read the manual.

{\color{Blue} \begin{footnotesize} 
\begin{verbatim}
consensus -h
consensus -help
\end{verbatim} \end{footnotesize}
}


\subsection{Sequence conversion}


\textit{consensus} uses a custom sequence format. Fortunately, the RSAT
package contains a sequence conversion program (\textit{convert-seq})
which supports Jerry Hertz' format. We will thus start by converting
the fasta sequences in this format. 

{\color{Blue} \begin{footnotesize} 
\begin{verbatim}
convert-seq -i PHO_up800-noorf.fasta -from fasta -to wc \ 
    -o PHO_up800-noorf.wc
\end{verbatim} \end{footnotesize}
}


\subsection{Running consensus}

Using consensus requires to choose the appropriate value for a series
of parameters. We found the following combination of parameters quite
efficient for discovering patterns in yeast upstream sequences.

{\color{Blue} \begin{footnotesize} 
\begin{verbatim}
consensus -L 10 -f PHO_up800-noorf.wc -A a:t c:g -c2 -N 10
\end{verbatim} \end{footnotesize}
}


The two main options of this command are 

\begin{description}
\item[-L 10] we guess that the pattern has a length of about 10 bp

\item[-N 10] we expect about 10 occurrences in the sequence set. Since
there are 5 genes in the family, this means that we expect on average
2 regulatory sites per gene, which is generally a good guess for
yeast.

\end{description}

Notice that several matrices are returned. Each matrix is followed by
the alignment of the sites on which it is based. Notice that the 4
matrices are highly similar, basically they are all made of several
occurrences of the high afinity site CACGTG, and matrices 1 and 3
contain one occurrence of the medium affinity site CACGTT. 

Also notice that these matrices are not made of exactly 10 sites
each. \textit{consensus} is able to adapt the number of sites in the
alignment in order to get the highest information content. The option
\option{-N 10} was an indication rather than a rigid requirement.

To save the result in a file, you can use the symbol ``greater than''
($>$) which redirects the output of a program to a file.

{\color{Blue} \begin{footnotesize} 
\begin{verbatim}
consensus -L 10 -f PHO_up800-noorf.wc -A a:t c:g -c2 -N 10 \
    > PHO_consensus
\end{verbatim} \end{footnotesize}
}


(this may take a few minutes)

Once the task is achieved, check the result.

{\color{Blue} \begin{footnotesize} 
\begin{verbatim}
more PHO_consensus
\end{verbatim} \end{footnotesize}
}

