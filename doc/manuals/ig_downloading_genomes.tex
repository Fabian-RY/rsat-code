%%%%%%%%%%%%%%%%%%%%%%%%%%%%%%%%%%%%%%%%%%%%%%%%%%%%%%%%%%%%%%%%
%%%% Downloading organisms from RSAT data repository
%%%%%%%%%%%%%%%%%%%%%%%%%%%%%%%%%%%%%%%%%%%%%%%%%%%%%%%%%%%%%%%%

\chapter{Downloading genomes}
\label{downloading_genomes}

\RSAT includes a series of tools to install and maintain the latest
version of genomes.

The most convenient way to add support for one or several organisms on
your machine is to use the programs \program{supported-organisms} and
\program{download-organism}.

Beware, the complete data required for a single genome may occupy
several hundreds of Mb, because \RSAT not only stores the genome
sequence, but also the oligonucleotide frequency tables used to
estimate background models, and the tables of BLAST hits used to get
orthologs for comparative genomics. If you want to install many
genomes on your computer, you should thus reserve a sufficient amount
of space.

\section{Original data sources}

Genomes supported on \RSAT were obtained from various sources.

Genomes can be installed either from the \RSAT web site, or from their
original sources.  

\begin{itemize}

\item NCBI/Genbank (\url{ftp://ftp.ncbi.nih.gov/genomes/}) was the
  primary source for installing genomes on \RSAT. Genomes are
  downloaded from the ftp site and installed locally on the \RSAT
  server by parsing the .gbk files.


\item The EBI genome directory
  (\url{ftp://ftp.ebi.ac.uk/pub/databases/genomes/Eukaryota/})
  contains supplementary genomes, which can be downloaded and
  installed on the \RSAT server by parsing files in embl format.

\item UCSC (\url{http://genome.ucsc.edu/}) for the multi-genome
  alingment files (multiz) used by \program{peak-footprints}.

\item Since 2008, ENSEMBL (\url{http://www.ensembl.org/}) genomes are
  supported by special tools (\program{retrieve-ensembl-seq},
  \program{supported-organisms-ensembl}), that remotely address
  queries to the Ensembl database.

\item Since 2013, genomes can be downloaded and installed on \RSAT
  servers, using the tool \program{install-ensembl-genome}. Once
  installed, ensembl genomes can be queried with the same tools as the
  other genomes installed on \RSAT servers (\program{retrieve-seq},
  \program{gene-info}, \ldots).

\end{itemize}

Other genomes can also be found on the web site of a diversity of
genome-sequencing centers.

\section{Requirement : wget}

The download of genomes relies on the application \program{wget},
which is part of linux distribution\footnote{For Linux:
  \url{http://www.gnu.org/software/wget/}; for Mac OSX
  \url{http://download.cnet.com/Wget/3000-18506_4-128268.html}}.


\program{wget} is a ``web aspirator'', which allows to download whole
directories from ftp and http sites. You can check if the program is
installed on your machine.

\begin{lstlisting}
wget --help
\end{lstlisting}


This command should return the help pages for \program{wget}.  If you
obtain an error message (``command not found''), you need to ask your
system administrator to install it.

\section{Importing organisms from the \RSAT main server}

The simplest way to install organisms on our \RSAT site is to download
the RSAT-formatted files from the web server. For this, you can use a
web aspirator (for example the program \program{wget}). 

Beware, the full installation (including Mammals) requires a large
disk space (several dozens of Gb). You should better start installting
a small genome and test it before processing to the full
installation. We illustrate the approach with the genome of our
preferred model organism: the yeast \textit{Saccharomyces cerevisiae}.

\subsection{Obtaining the list of organisms supported on the \RSAT server}

By default, the program \program{supported-organisms} returns the list
of organisms supported on your local \RSAT installation. You can
however use the option \option{-server} to obtain the list of
organisms supported on a remote server.


\begin{lstlisting}
supported-organisms -server
\end{lstlisting}

The command can be refined by restricting the list to a given taxon of
interest.

\begin{lstlisting}
supported-organisms -server -taxon Fungi
\end{lstlisting}

You can also ask additional information, for example the date of the
last update and the source of each genome.

\begin{lstlisting}
supported-organisms -server -taxon Fungi -return last_update,source,ID
\end{lstlisting}


\subsection{Importing a single organism}


The command \command{download-organism} allows you to download one or
several organisms. 

Beware, the complete data for a single genome may occupy several tens
of Megabytes (Bacterial genomes) or a few Gigabases
(Mammalian). Downloading tenomes thus requires a fast Internet
connection, and may take time. If possible, please download genomes
during the night (European time).

As a first step, we recommend to download the genome of the yeast
\org{Saccharomyces cerevisiae}, since this is the model organism used
in our tutorials.


\begin{lstlisting}
download-organism -v 1 -org Saccharomyces_cerevisiae
\end{lstlisting}

In principle, the download should start immediately. \emph{Beware},
the data volume to be downloaded is important, because the genome
comes together with extra files (blast hits with other genoems,
oligonucleotide and dyad frequencies). Depending on the network
bandwidth, the download of a genome may take several minutes or tens
of minutes. 

After the task is completed, you can check if the configuration file
has been correctly updated by typing the command.

\begin{lstlisting}
supported-organisms
\end{lstlisting}

In principle, the following information should be displayed on your
terminal.

\result{Saccharomyces\_cerevisiae}

You can also add parameters to get specific information on the
supported organisms.

\begin{lstlisting}
supported-organisms -return ID,last_update
\end{lstlisting}


\subsection{Importing a few selected organisms}

The program \program{download-organism} can be launched with a list of
organisms by using iteratively the option \option{-org}.


\begin{lstlisting}
download-organism -v 1 -org Escherichia_coli_K12 -org Mycoplasma_genitalium_G37_uid57707
\end{lstlisting}

Note: genome names may change with time, since genome centers are
occasionally adding new suffixes for genomes. The organism names
indicated after the option \option{-org} should belong to the list of
supported organisms collected with the command
\program{supported-organisms -server}.


\subsection{Importing all the organisms from a given taxon}

For comparative genomics, it is convenient to install on your server
all the organisms of a given taxon. For this, you can simply use the
option \option{-taxon} of \program{download-organism}.

Before doing this, it is wise to check the number of genomes that
belong to this taxon on the server.

\begin{lstlisting}
## Get the list of organisms belonging to the order "Enterobacteriales" on the server
supported-organisms -taxon Enterobacteriales -server

## Count the number of organisms
supported-organisms -taxon Enterobacteriales -server | wc -l
\end{lstlisting}

In Oct 2009, there are 94 Enterobacteriales supported on the \RSAT
server. Before starting the download, you should check two things:
\begin{enumerate}
\item Has your network a sufficient bandwidth to ensure the transfer
  in a reasonable time ?
\item Do you have enough free space on your hard drive to store all those genomes ? 
\end{enumerate}

If the answer to both questions is ``yes'', you can start the
download.

\begin{lstlisting}
download-organism -v 1 -taxon Enterobacteriales 
\end{lstlisting}



