\documentclass{book}
%\documentstyle[makeidx]{book}
\makeindex
%\usepackage{color}
\usepackage[usenames]{color}
\usepackage{times}
\usepackage{graphics}
\usepackage{latexsym}
\usepackage{makeidx}


%%%%%%%%%%%%%%%%%%%%%%%%%%%%%%%%%%%%%%%%%%%%%%%%%%%%%%%%%%%%%%%%
%%%%%%%%%%%%%%%%%%%%%%%%%%% commands %%%%%%%%%%%%%%%%%%%%%%%%%%%
\newcommand{\tbw}{\textbf{TO BE WRITTEN}}
\newcommand{\RSAT}{\textbf{\textit{RSAT}}}
\newcommand{\file}[1]{\textit{#1}}
\newcommand{\concept}[1]{\index{#1}\textsl{#1}}
\newcommand{\command}[1]{\begin{footnotesize}\begin{quote}\textcolor{Blue}{\texttt{#1}}\end{quote}\end{footnotesize}}
\newcommand{\result}[1]{\begin{footnotesize}\begin{quote}\textcolor{OliveGreen}{\texttt{#1}}\end{quote}\end{footnotesize}}
\newcommand{\program}[1]{\textbf{\textsl{#1}}}
\newcommand{\option}[1]{\texttt{#1}}
\newcommand{\email}[1]{\textit{#1}}

\newcommand{\address}[1]{\small{#1}}
\newcommand{\org}[1]{\textit{#1}}
\newcommand{\gene}[1]{\textit{#1}}
\newcommand{\seq}[1]{\texttt{#1}}

\newcommand{\url}[1]{\textit{#1}}
\newcommand{\urlref}[1]{\footnote{\textit{#1}}}

\newcommand{\scmbb}{
	Service de Conformation des Macromol\'{e}cules Biologiques et de Bioinformatique, \\
	Universit\'{e} Libre de Bruxelles, \\
	Campus Plaine, CP 263, Boulevard du Triomphe, B-1050 Bruxelles, Belgium. \\
	Tel: +32 2 650 2013 - Fax: +32 2 650 5425
}

%%%%%%%%%%%%%%%%%%%%%%%%%%%%%%%%%%%%%%%%%%%%%%%%%%%%%%%%%%%%%%%%
%%%%%%%%%%%%%%%%%%%%%%%%% environments %%%%%%%%%%%%%%%%%%%%%%%%%

\def\ex@mple{example}
\def\ex@rcise{exercise}
\def\theoremfont{\ifx\@currenvir\ex@mple
	\def\@thmfont{\rm} \else
	\ifx\@currenvir\ex@rcise
	\def\@thmfont{\rm} \else
	\def\@thmfont{\it}\fi\fi}
\newtheorem{exercise}{Exercise}[chapter]
\newtheorem{example}{Example}[chapter]



\usepackage{url}
\begin{document}
\title{Network Analysis Tools \\
Web server installation}

\author{
	Sylvain Broh\'ee \\
	\email{sbrohee@ulb.ac.be} \\
        \and \\
	Karoline Faust \\
	\email{kfaust@ulb.ac.be} \\
        \and \\
	Jacques van Helden \\
	\email{jvhelden@ulb.ac.be}\\
        \\
        \\
        \bigre
}


\maketitle

\newpage
\tableofcontents
\newpage

\section*{Description}

This document describes the installation procedure for the web server
of the \textbf{Network Analysis Tools} (\neat).

It assumes that you already installed the stand-alone version of the
software suite Regulatory Sequence Analysis Tools (\RSAT), as
described in the \RSAT installation guide, and that you configured a
local web server, as described in the \RSAT webserver install guide.


\section{Installation of stand-alone programs required for \neat}


For the Network Analysis Tools (NeAT), we need to install a set of
programs developed by third-parties.  Some of these programs come in
the contribb directory of the \neat distribution. Some others have to
be downloaded from their original distribution site.

\subsection{Compiling stand-alone programs provided in the RSAT/NEAT distribution}

\begin{lstlisting}
make -f makefiles/init_RSAT.mk   compile_pathway_tools
\end{lstlisting}

\subsubsection{Checking and configuring kwalks for NeAT}

You can test that the compilation worked by running the following
command.

\begin{lstlisting}
$RSAT/contrib/kwalks/bin/lkwalk
\end{lstlisting}
%$

This should display the help message of \program{lkwalk}.

Check that the KWALKS\_ROOT variable in the RSAT config file
(\file{\$RSAT/RSAT\_config.props}) points to the correct path (it
should be the absolute path of \file{\$RSAT/contrib/kwalks/bin/}).

\subsubsection{Checking and configuring REA for NeAT}

You can test that the compilation worked by running the following
command.

\begin{lstlisting}
$RSAT/contrib/REA/REA
\end{lstlisting}
%$

This should display the help message of \program{REA}.

Check that the REA\_ROOT variable in the RSAT config file
(\file{\$RSAT/RSAT\_config.props}) points to the correct path (it
should be the absolute path of \file{\$RSAT/contrib/REA}).

%%%%%%%%%%%%%%%%%%%%%%%%%%%%%%%%%%%%%%%%%%%%%%%%%%%%%%%%%%%%%%%%
% Web server installation
\chapter{NeAT Web server installation}
\section{Installing a local web server}

As the Regulatory Sequence Analysis Tools, \neat includes a web
server, which offers a user-friendly interface for biologists. The
main server is available for academic users at
\url{http://rsat.ulb.ac.be/neat/}. A few additional mirrors have been
installed in different countries.

\subsection{Web server pages}

The web pages are located in the directory
\file{rsa-tools/public\_html}. This directory contains both the HTML help
pages, and the PHP and CGI scripts.

\subsection{Apache modules}

The \neat interface mainly relies on PHP (and CGI only for the roc-stats tool).
These modules should be installed on the web server, and activated in the Apache configuration files.
The installation and configuration of CGI is described in manual of the web server of \RSAT.

To perform the following steps, you might dipose of the administrator permissions.

\subsubsection{PHP module for Mac OSX}

If your server is running under Mac OSX, you need to install a recent
version (at least v5) of the php module, which can be found at the
following site. \\
\url{http://www.entropy.ch/software/macosx/php/}

\subsubsection{PHP module for LINUX}

Generally, PHP5 is included with the Linux distribution or can easyly
be installed using the installer (YAST, YUM, etc). Take care that the
PHP5 Module for Apache 2.0 (apache2-mod\_php5) is installed.

PHP5 can also be installed manually from the PHP website \\
\url{http://www.php.net/downloads.php}

In addition, you will need to install the module php5-SOAP.

\subsubsection{Modification of php.ini}

In order for the server to work with \neat, you have to edit the main
PHP configuration file \file{php.ini}. Depending on your PHP
installation, this file might be in different directory. On Linux
computers, it is usually found here: \\
\file{/etc/php5/apache2/php.ini}
directory. In this file, you must modify the following fields
accordingly.

\begin{lstlisting}
soap.wsdl_cache_enabled=0
max_execution_time = 3600
max_input_time = 1800
memory_limit = 1G
error_reporting = E_ALL & ~E_NOTICE
post_max_size = 50M
upload_max_filesize = 50M
default_socket_timeout = 1800
upload_tmp_dir = "/tmp/php/"
\end{lstlisting}

Pay attention to the location of php5 extension libraries. Indicate
the correct directory, e.g.:

\begin{lstlisting}
extension_dir=/usr/lib/php5/extensions
\end{lstlisting}

You must change permissions so that the directory \url{/tmp/php/} for
temporary upload is writable by eveverybody. To do this, type:

\begin{lstlisting}
sudo mkdir -p /tmp/php; 
sudo chmod 777 /tmp/php
\end{lstlisting}


\subsection{Tomcat}

The path finding and pathway extraction tools are using axis web
services, JSP and Java servlet pages and needs Tomcat
(\url{http://tomcat.apache.org/}) or an equivalent servlet engine to
run.  

Tomcat can be easily installed on SUSE with yast and is usually
located in \file{/usr/share/tomcat} after installation.

%% On MacOS, it is already installed by default.
On MaxOS, you need to install TomCat
(\url{http://tomcat.apache.org/}).

Make sure to install at least Tomcat version 5.  We noticed however
that Tomcat version 6 is more stable.  

We will refer from now on to the Tomcat root directory as
\$CATALINA\_HOME (for example, \file{/usr/share/tomcat7}).

\subsubsection{Tomcat configuration}

If you would like to use the Tomcat manager, make sure to configure
the file tomcat-users.xml located in \$CATALINA\_HOME/conf.

Add a manager role with a special login and password, e.g.
\begin{lstlisting}
<tomcat-users>
  <role rolename="tomcat"/>
  <role rolename="manager"/>
  <user username="tomcat" password="tomcat" roles="tomcat"/>
  <user username="metheadmin" password="mypassword" roles="manager"/>
</tomcat-users>
\end{lstlisting}

Make sure that tomcat-users.xml cannot be read by anyone else than
tomcat or root.

By default, Tomcat takes a lot of memory. You can reduce this amount
by modifying the file tomcat.conf in \$CATALINA\_HOME/conf.
Adjust the variable JAVA\_OPTS as follows:

JAVA\_OPTS="-Xmx800m"

You can start Tomcat on SUSE with

 \textit{/etc/init.d/tomcat6 start}

and stop it with

 \textit{/etc/init.d/tomcat6 stop}

More advice for the installation and configuration of Tomcat on unix systems
can be found at: \textit{http://linux-sxs.org/internet\_serving/book1.html}

\subsubsection{Optional: Mod\_jk}

The installation of Mod\_jk is optional. If you do not want to install it,
skip this section.

Mod\_jk forwards requests for defined contexts from Apache to Tomcat.
Users that have a strict firewall configuration blocking the Tomcat port
will be able to access the web services through the Apache default port.

To install and configure mod\_jk, you can follow the steps below:

\begin{enumerate}
\item Install the mod\_jk module from \file{http://tomcat.apache.org/connectors-doc/}
      or via installation systems like yast.
\item In the Apache configuration folder
(e.g. \file{/etc/apache2/conf.d}), add two configuration files: workers.properties and tomcat.conf
\item Content of workers.properties
\begin{lstlisting}
  # Define 1 real worker using ajp13
    # this coud be al list in the format
    # worker.list=worker1, worker2, worker3, worker4
    worker.list=worker1
    # Set properties for worker1 (ajp13)
    worker.worker1.type=ajp13
    worker.worker1.host=localhost
    worker.worker1.port=8009
\end{lstlisting}
\item Content of tomcat.conf
\begin{lstlisting}
    # Update this path to match your modules location
    LoadModule jk_module /usr/lib/apache2/mod_jk.so
    # Where to find workers.properties
    # Update this path to match your conf directory location
    JkWorkersFile /etc/apache2/conf.d/workers.properties
    # Where to put jk shared memory
    # Update this path to match your local log directory
    # JkShmFile /var/log/apache2/mod_jk.shm
    # Where to put jk logs
    # Update this path to match your logs directory location
    JkLogFile /var/log/apache2/mod_jk.log
    # Set the jk log level [debug/error/info]
    JkLogLevel error
    # Send everything for context to worker named worker1 (ajp13)
    JKMount /be.ac.ulb.bigre.graphtools.server/* worker1
    JKMount /metabolicpathfinding/* worker1
\end{lstlisting}
\item Reload Apache (on SUSE \file{/etc/init.d/apache2} reload).
\end{enumerate}

An alternative to mod\_jk is mod\_proxy, which requires less configuration.


\subsubsection{Preparation of folders for Tomcat}
In Linux, a tomcat user is created if Tomcat is installed via yast. Make sure
that this user has read, write and execution rights for the following folders:

\file{\$RSAT/contrib/REA}

\file{\$RSAT/contrib/kwalks/bin}

\file{\$RSAT/public\_html/data/KEGG}

\file{\$RSAT/public\_html/data/Stored\_networks}

\file{\$RSAT/java/lib}


If any of these folders does not yet exist, create them.

For installation of REA and kwalks see the section on third-party programs in
the RSAT install guide.

\subsection{Java tools server}

The graphtools server contains the java web services of NeAT.

\subsubsection{Installation of Java tools server}

The graphtools server is stored as a war file in \file{\$RSAT/java/web}.

\begin{enumerate}

\item Open the RSAT configuration file
\file{RSAT\_config.props} located in \file{\$RSAT}.
Set the value of the parameter \textit{neat\_java\_ws}
to \file{web\_services/GraphAlgorithms.wsdl}

\item If mod\_jk or equivalent has not been installed,
open the RSAT configuration file and set the parameter
\textit{tomcat\_port} to the port on which Tomcat runs (by default 8080).

\item There are two ways to deploy a war file.
 If you do not use the Tomcat manager, make sure
 Tomcat can open the war file with the command:

 \textit{chown tomcat:tomcat be.ac.ulb.bigre.graphtools.server.war}
 Then place the war file \textit{be.ac.ulb.bigre.graphtools.server.war} in
 \$CATALINA\_HOME/webapps and start Tomcat.
 If you use the Tomcat manager, open
 (\file{http://localhost:8080/manager/html} and go to option \textit{Select WAR file to upload}.
 After having selected the war file to upload, click \textit{Deploy}.

\item Go to \file{\$CATALINA\_HOME/webapps/be.ac.ulb.bigre.graphtools.server/WEB-INF}

\item Open \textit{serverConfig.txt} and set the value of RSAT\_ROOT to the value of \$RSAT.

\item Run install.sh with the following commands:

	  \textit{chmod 755 install.sh}

      \textit{./install.sh}

\end{enumerate}

You are done. You may read the section "Configuration remarks" for additional information.
If you want to install the metabolic pathfinder, continue with section "Metabolic Pathfinder and Pathway extraction".

\subsubsection{Configuration remarks}

The directory \file{\$RSAT/public\_html/data/Stored\_networks}
allows to store graph files for longer time
(since \file{\$RSAT/public\_html/tmp} is cleaned regularly).


Note that the Kegg network provider accesses the metabolic database to add attributes
such as compound name or EC number to networks.
In order to install the database, see the installation of the metabolic pathfinder.
The name of this database, its IP address, its owner and password may be modified.
These parameters may be set in \textit{serverConfig.txt}
located at \file{\$CATALINA\_HOME/webapps/be.ac.ulb.bigre.graphtools.server/WEB-INF}.

\subsubsection{Update the KEGG network provider}

The KEGG network provider places organism-specific KGML files in \file{\$RSAT/public\_html/data/KEGG}
in a folder with the KEGG organism name (e.g. sce).
You may place KGML folders for organisms yourself or you may let the program download
required KGML files on the fly. In the latter case, make sure that the KGML version is set correctly.
To change the KGML version displayed in the web interface,
go to 

\file{\$CATALINA\_HOME/webapps/be.ac.ulb.bigre.graphtools.server/WEB-INF}

Open \textit{serverConfig.txt} with a text editor, and modify the
value of the parameter \texttt{KGML\_VERSION}.


The Kegg network provider queries a metabolic database in order to
annotate the network with certain compound and reaction attributes.

To install and update the metabolic database, use the
\command{KeggLigandDataManager} command line tool.  For details, type:

\begin{lstlisting}
java graphtools.parser.KeggLigandDataManager -h
\end{lstlisting}


The KEGG network provider is also using the file rpairs.tab located in
\file{\$RSAT/data/KEGG} to convert reaction into RPAIR graphs. To
update this file, use the KeggLigandDataManager command line tool.

Finally, the KEGG network provider displays a list of KEGG organisms. To obtain the
organism list for the recent KEGG version, use the MetabolicGraphProvider command line
tool. For details on this tool, type: \textit{java graphtools.util.MetabolicGraphProvider -h}
Place the updated list in the folder \file{\$RSAT/data/KEGG} to replace the old list.

\subsection{Metabolic Pathfinder and Pathway extraction}
After successful installation of the Java web services, you may install metabolic pathfinder
and Pathway extraction, which are clients of the pathfinder and pathwayinference web services respectively.

\subsubsection{Requirements of metabolic pathfinder and pathway extraction}

Metabolic pathfinder and pathway extraction have a number of additional requirements.

\begin{enumerate}

\item Dot

Dot is needed to draw graphs and can be obtained freely from \textit{http://www.graphviz.org/}.

\item Postgres
\begin{enumerate}
\item Installation

Postgres (version 8.2 or later) is needed to store KEGG and MetaCyc data.
Postgres can be obtained freely from
\textit{http://www.postgresql.org/}.
For MacOS, you can use Darwinports to install postgres. On SUSE, you may
install it with yast (install postgresql and postgresql-server).
Usually, a postgres user is created during installation of postgres.

\item Postgres configuration

You may need to configure the postgres server. You can simply
allow all users on your machine (but not from outside) to access
all postgres databases.
This can be achieved by modifying the pg\_hba.conf file
located in the postgres home directory. Paste the following in this file:
\begin{lstlisting}
host    all         all         127.0.0.1/32          trust
# IPv6 local connections:
host    all         all         ::1/128               trust
host    all         all         127.0.0.1             trust
\end{lstlisting}
For more details on postgres configuration, check the postgres manual
on the pg\_hba.conf and pg\_ident.conf files.

\item Start the server

On SUSE, you may start the postgres server with
\textit{/etc/init.d/postgresql start}. On MacOS, you may start the postgres server
using a command similar to

\textit{pg\_ctl -D /usr/local/pgsql/data/ -l logfile start}.


\item Obtain the data file

Download the postgres backup file \file{metabolicdb\_dump\_day\_month\_year.backup}
from the data section of the official NeAT web site.

\item Load data into postgres

Load the \file{metabolicdb\_dump\_day\_month\_year.backup} file into postgres as follows:
\begin{enumerate}
\item Start postgres by typing the following on command line:

\textit{psql -U postgres}
\item In postgres, do:

\textit{create user metabolic with password 'metabolic';}

\textit{create database "metabolicdb" with owner "metabolic" encoding='UTF8';}
\item Quit postgres and type the following command in one line:

\textit{pg\_restore -d metabolicdb metabolicdb\_dump\_day\_month\_year.backup -U postgres}
\end{enumerate}

\end{enumerate}

\end{enumerate}

\subsubsection{Installation of metabolic pathfinder and pathway extraction}
Either place the war file \textit{metabolicpathfinding.war} located
in \file{\$RSAT/java/web} in

\$CATALINA\_HOME/webapps and then start
Tomcat or use the Tomcat manager to deploy the war file.

\subsubsection{Configuration of metabolic pathfinder and pathway extraction}

\begin{enumerate}
\item Go to the folder \file{\$CATALINA\_HOME/webapps/metabolicpathfinding/WEB-INF}.

\item If you are not the rsat user, set the RSAT environment variable, e.g.:

      \textit{export RSAT=/home/rsat/rsa-tools}
      Run the configuration script with the following commands:

      \textit{chmod 755 configureWebxml.pl}

      \textit{./configureWebxml.pl}

\item Reload the metabolic pathfinding web application using the Tomcat manager or
      on command line, restart Tomcat.
\end{enumerate}

\subsubsection{Update of metabolic pathfinder}
The metabolic pathfinder contains by default data from KEGG version 49.0.
In order to update it, you can follow the steps below:

\begin{enumerate}

\item Load KEGG LIGAND compound and reaction file into the metabolic database.
      The command line tool \textit{KeggLigandDataManager} can be used for this.
      If present, delete previous KEGG data in the database. You may change the name,
      the location, owner and password of the metabolic database. In this case, change the
      default values in the web.xml file located at

      \file{\$CATALINA\_HOME/webapps/metabolicpathfinding/WEB-INF}.

\item Generate the preloaded networks with the \textit{KeggLigandDataManager} command line tool.
      Place those networks in \file{\$RSAT/data/Stored\_networks} replacing the old ones.

\item In point 2, example networks have been generated as well.

      Copy them to
      \file{\$CATALINA\_HOME/webapps/metabolicpathfinding/networks}, replacing the old ones.

\item Set the parameter \textit{keggVersion} in the web.xml file to the updated KEGG version.

\item Reload the metabolicpathfinding web application in tomcat.

\end{enumerate}

\subsubsection{Update of pathway extraction}

This section only describes how to update MetaCyc data.
See the section on the update of metabolic pathfinder for the KEGG data.

\begin{enumerate}

\item Load the MetaCyc OWL file into the metabolic database.
      The command line tool \textit{MetabolicXMLFilesParser} can be used for this.
      If present, delete previous MetaCyc data in the database. You may change the name,
      the location, owner and password of the metabolic database. In this case, change the
      default values in the web.xml file located at

      \file{\$CATALINA\_HOME/webapps/metabolicpathfinding/WEB-INF}.

\item Generate the preloaded networks with the \textit{MetabolicGraphProvider} command line tool.
      Place those networks in \file{\$RSAT/data/Stored\_networks} replacing the old ones.

\item In point 2, an example network has been generated as well.

      Copy it to
      \file{\$CATALINA\_HOME/webapps/metabolicpathfinding/networks}, replacing the old one.

\item Set the parameter \textit{metacycVersion} in the web.xml file to the updated MetaCyc version.

\item Reload the metabolicpathfinding web application in tomcat.

\end{enumerate}

\subsection{Web services}

\subsubsection{Edit the WSDL file}
The web interface consists in web services that are called by the PHP web pages. Your computer must thus act as web service server.

First, edit the file \file{RSATWS.wsdl} located in the
\file{\$RSAT/public\_html/web\_services/} directory.
At the very end of the file, the line

\begin{lstlisting}
 <soap:address location="http://rsat.bigre.ulb.ac.be/rsat/web_services/RSATWS.cgi"/>
\end{lstlisting}

must be replaced by

\begin{lstlisting}
 <soap:address location="url_of_the_cgi_file_on_your_server"/>
\end{lstlisting}

The URL can be \file{http://127.0.0.1/rsa-tools/web\_services/RSATWS.cgi}.

\subsubsection{Edit the \neat config file}

Edit the \file{RSAT\_config.props} present in the main RSAT directory and edit the following fields so that they correspond to your local configuration.

\begin{lstlisting}
neat_supported=1
neat_ws=web link to the WSDL file on your computer
(e.g. http://127.0.0.1/rsa-tools/web_services/RSATWS.wsdl)
\end{lstlisting}

\subsubsection{Change permissions of the temporary files and log files directories}

The directories \file{\$RSAT/public\_html/logs/} and \file{\$RSAT/public\_html/tmp/} must be writable. So, change the permissions

\begin{lstlisting}
chmod 777 \$RSAT/public\_html/logs/
chmod 777 \$RSAT/public\_html/tmp/
\end{lstlisting}

\end{document}
