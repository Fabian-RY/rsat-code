%%%%%%%%%%%%%%%%%%%%%%%%%%%%%%%%%%%%%%%%%%%%%%%%%%%%%%%%%%%%%%%%
%%%% COMPARATIVE GENOMICS
%%%%%%%%%%%%%%%%%%%%%%%%%%%%%%%%%%%%%%%%%%%%%%%%%%%%%%%%%%%%%%%%
\section{Comparative genomics}

\subsection{Genome-wise comparison of protein sequences}

In this section, I explain how to use the program
\program{genome-blast}, which runs the sequence similarity search
program \program{BLAST} to detect significant similarities between all
the proteins of a set of genomes. This operation can take time, and
the result tables occupy a considerable amount of space on the hard
disk. The \RSAT distribution does thus not include the complete
comparison of all genomes against all other ones. The current \RSAT
distribution includes the complete comparisons for some model genomes
(\organism{Escherchia coli K12} versus all bacteria,
\organism{Saccharomyces cerevisiae} against all Fungi, ...), but you
might need to perform additional comparisons for your own purpose.

In order to install the tables of gene similarities in the \RSAT data,
you need to have the writing permission in the $\$RSAT$ directory. If
this is not the case, ask your system administrator to do it for you.

\textbf{TO BE WRITTEN}

\subsection{Getting putative homologs, orthologs and paralogs}

In this section, I will explain how to use the program
\program{get-orthologs}. This program takes as input one or several
query genes belonging to a given organism (the \concept{reference
  organism}), and return the genes whose product (peptidic sequence)
show significant similarities with the products of the query
genes. The primary usage of \program{get-orthologs} is thus to return
lists of similar genes, not specialy orthologs. Additional criteria
can be imposed to infer orthology.  In particular, one of the most
common criterion is to select \concept{bidirectional best hits
  (BBH)}. This can be achieved by imposing the rank 1 with the option
\option{-uth rank 1}.

We will illustrate the concept by retrieving the genes whose product
is similar to the protein LexA of \organism{Escherichia coli K12}, in
all the Gammaproteobacteria. We will then refine the query to extract
putative orthologs.

\subsubsection{Getting genes by similarities}

\begin{small}
\begin{verbatim}
get-orthologs -v 1 -org Escherichia_coli_K12 \
  -taxon Gammaproteobacteria \
  -q lexA -o lexA_orthologs_Gammaproteobacteria.tab
\end{verbatim}
\end{small}

The result file is a list of all the Gammaproteobacterial genes whose
product shows some similarity with the LexA protein from E.coli K12.  

\begin{scriptsize}
\begin{verbatim}
...
#ref_id ref_org query
Sde_1787        Saccharophagus_degradans_2-40   b4043
CPS_0237        Colwellia_psychrerythraea_34H   b4043
CPS_2683        Colwellia_psychrerythraea_34H   b4043
CPS_1635        Colwellia_psychrerythraea_34H   b4043
IL0262  Idiomarina_loihiensis_L2TR      b4043
...
c5014   Escherichia_coli_CFT073 b4043
c3190   Escherichia_coli_CFT073 b4043
b4043   Escherichia_coli_K12    b4043
...
\end{verbatim}
\end{scriptsize}

Each similarity is reported by the ID of the gene, the organism to
which is belong, and the ID of the query gene. In this case, the third
column contains the same ID on all lines: b4043, which is the ID of
the gene lexA in \organism{Escherichia coli K12}. It seems thus poorly
informative, but this column becomes useful when several queries are
submitted simultaneously.

\subsubsection{Obtaining information on the BLAST hits}

The program \program{get-orthologs} allows to return additional
information on the hits. The list of supported return fields is
obtained by calling the command with the option \option{-help}. For
example, we can ask to return the percentage of identity, the
alignment length, the E-value and the rank of each hit.

\begin{small}
\begin{verbatim}
get-orthologs -v 1 -org Escherichia_coli_K12 \
  -taxon Gammaproteobacteria \
  -q lexA -o lexA_orthologs_Gammaproteobacteria.tab \
  -return ident,ali_len,e_value,rank 
\end{verbatim}
\end{small}

Which gives the following result:

\begin{scriptsize}
\begin{verbatim}
...
#ref_id ref_org query   ident   ali_len e_value rank
Sde_1787        Saccharophagus_degradans_2-40   b4043   65.33   199     1e-68   1
CPS_0237        Colwellia_psychrerythraea_34H   b4043   65.69   204     6e-75   1
CPS_2683        Colwellia_psychrerythraea_34H   b4043   33.94   109     1e-10   2
CPS_1635        Colwellia_psychrerythraea_34H   b4043   34.12   85      1e-06   3
IL0262  Idiomarina_loihiensis_L2TR      b4043   66.83   202     1e-75   1
...
c5014   Escherichia_coli_CFT073 b4043   100.00  202     2e-111  1
c3190   Escherichia_coli_CFT073 b4043   43.33   90      2e-14   2
b4043   Escherichia_coli_K12    b4043   100.00  202     2e-111  1
...
\end{verbatim}
\end{scriptsize}

Not surprisingly, the answer includes the self-match of lexA (ID
b4043) in \organism{Escherichia coli K12}, with 100\% of identify.

\subsubsection{Selecting bidirectional best hits}

We can see that the output contains several matches per genome. For
instance, there are 3 matches in \organism{Colwellia psychrerythraea
  34H}. If we assume that these similarities reflect homologies, the
result contains thus a combination of paralogs and orthologs. 

The simplest criterion to select ortholog is that of
\concept{bidirectional best hit (BBH)}. We can select BBH by imposing
an upper threshold on the rank, with the option \option{-uth}.

\begin{small}
\begin{verbatim}
get-orthologs -v 1 -org Escherichia_coli_K12 \
  -taxon Gammaproteobacteria \
  -q lexA -o lexA_orthologs_Gammaproteobacteria_bbh.tab \
  -return ident,ali_len,e_value,rank \
  -uth rank 1 
\end{verbatim}
\end{small}

The result has now been reduced to admit at most one hit per genome.

\begin{scriptsize}
\begin{verbatim}
...
#ref_id ref_org query   ident   ali_len e_value rank
Sde_1787        Saccharophagus_degradans_2-40   b4043   65.33   199     1e-68   1
CPS_0237        Colwellia_psychrerythraea_34H   b4043   65.69   204     6e-75   1
IL0262  Idiomarina_loihiensis_L2TR      b4043   66.83   202     1e-75   1
...
c5014   Escherichia_coli_CFT073 b4043   100.00  202     2e-111  1
b4043   Escherichia_coli_K12    b4043   100.00  202     2e-111  1
...
\end{verbatim}
\end{scriptsize}

\subsubsection{Selecting hits with more stringent criteria}

It is well known that the sole criterion of BBH is not sufficient to
infer orthology between two genes. In particular, there is a risk to
obtain irrelevant matches, due to partial matches between a protein
and some spurious domains. To avoid this, we can add a constraint on
the percentage of identity (min 30\%), and on the alignment length
(min 50 aa). These limits are somewhat arbitrary, we use them to
illustrate the principe, and leave to each user the responsibility to
choose the criteria that she/he considers as relevant. Finally, we
will use a more stringent threshold on E-value than the default one,
by imposing an upper threshold of 1e-10.

\begin{small}
\begin{verbatim}
## Note that or this test we suppress the BBH constraint (-uth rank 1)
get-orthologs -v 1 -org Escherichia_coli_K12 \
  -taxon Gammaproteobacteria \
  -q lexA -o lexA_orthologs_Gammaproteobacteria_id30_len50_eval-10.tab \
  -return ident,ali_len,e_value,rank \
  -lth ident 30 -lth ali_len 50 -uth e_value 1e-10
\end{verbatim}
\end{small}

We can now combine the constrains above with the criterion of BBH.

\begin{small}
\begin{verbatim}
## Note that or this test we include the BBH constraint (-uth rank 1)
get-orthologs -v 1 -org Escherichia_coli_K12 \
  -taxon Gammaproteobacteria \
  -q lexA -o lexA_orthologs_Gammaproteobacteria_bbh_id30_len50_eval-10.tab \
  -return ident,ali_len,e_value,rank \
  -lth ident 30 -lth ali_len 50  -uth e_value 1e-10 \
  -uth rank 1 
\end{verbatim}
\end{small}

As expected, the number of selected hits is reduced by adding these
constraints. In Sept 2006, we obtained the following number of hits
for lexA in Gammaproteobacteria.

\begin{itemize}
\item 122 hits without any constraint;
\item 107 hits with contrains on ident,ali\_len and e\_value;
\item 69 hits with the constraint of BBH;
\item 69 hits with the combined constraint of BBH, at least 30\%
  identity and an alignment over more than 50 aminoacids, and an
  E-value <= 1.e-10.
\end{itemize}

Actually, in the particular case of \gene{lexA}, the BBH constraint
already filtered out the spurious matches, but inother cases they can
be useful.

\subsection{Retrieving sequences for multiple organisms}

The program \program{retrieve-seq-multigenome} can be used to retrieve
sequences for a group of genes belonging to different organisms.This
program takes as input a file with two columns. Each row of this file
specifies one query gene.

\begin{enumerate}

\item The first column contains the name or identifier of the gene
  (exactly as for the single-genome program \program{retrieve-seq}).

\item The second column indicates the organism to which the gne belongs.

\end{enumerate}

The output of \program{get-orthologs} can thus directly be used as
input for \program{retrieve-seq-multigenome}.

\begin{small}
\begin{verbatim}
retrieve-seq-multigenome -noorf \
  -i lexA_orthologs_Gammaproteobacteria_bbh_id30_len50_eval-10.tab \
  -o lexA_orthologs_Gammaproteobacteria_up-noorf.fasta
\end{small}
\end{verbatim}
\end{small}

\subsection{Detection of phylogenetic footprints}

\textbf{TO BE WRITTEN}

\begin{small}
\begin{verbatim}
dyad-analysis  -v 1 \
  -i lexA_orthologs_Gammaproteobacteria_up-noorf.fasta \
  -sort -2str -noov -lth occ 1 -lth occ_sig 0 \
  -return occ,freq,proba,rank \
  -l 3 -spacing 0-20 -bg monads \
  -o lexA_orthologs_Gammaproteobacteria_up-noorf_dyads-2str-noov.tab
\end{verbatim}
\end{small}

\subsection{Phylogenetic profiles}

The notion of \concept{phylogenetic profile} was introduced by
Pellegrini et al. (1999). They identified putative orthologs for all
the genes of \organism{Escherichia coli K12} in all the complete genomes
available at that time, and built a table with one row per gene, one
column per genome. Each cell of this table indicates if an ortholog of
the considered gene (row) has been identified in the considered genome
(column). Pellegrini et al. (1999) showed that genes having similar
phylogenetic profiles are generally involved in common biological
processes. The analysis of phylogenetic profiles is thus a powerful
way to identify functional grouping in completely sequenced genomes.

The program \program{get-orthologs} can be used to obtain the
phylogenetic profiles. The principle is to submit the complete list of
protein-coding genes of the query organism. We process in two steps : 

\begin{enumerate}

\item With \program{get-orthologs}, we can identify the putative
  ortholgos for all the genes of the query organism, using the
  criterion of \concept{bidirectional best hit (BBH)}. This generate a
  large table with one row per pair of putative orthologs.

\item We then use \program{convert-classes} to convert the ortholog
  table into profiles (one row per gene, one column per genome).

\end{enumerate}

We will illustrate this by calculating the phylogenetic profiles of
all the genes from \organism{Saccharomyces cerevisiae} across all the
Fungi. We use a level of verbosity of 2, in order to get information
about the progress of the calculations.

\begin{small}
\begin{verbatim}

## Identify all the putative orthologs (BBH)
get-orthologs -v 2 \
  -i $RSAT/data/genomes/Saccharomyces_cerevisiae/genome/cds.tab  \
  -org Saccharomyces_cerevisiae \
  -taxon Fungi \
  -uth rank 1 -lth ali_len 50 -lth ident 30 -uth e_value 1e-10 \
  -return e_value,bit_sc,ident,ali_len \
  -o Saccharomyces_cerevisiae_Fungi_bbh.tab

## Convert ortholog table into a profile table
## with the IDs of the putative orthologs
convert-classes -v 2 \
  -i Saccharomyces_cerevisiae_Fungi_bbh.tab  \
  -from tab -to profiles \
  -ccol 2 -mcol 3 -scol 1 -null "<NA>" \
  -o Saccharomyces_cerevisiae_Fungi_phyloprofiles_ids.tab

\end{verbatim}
\end{small}


The resulting table indicates the identifier of the ortholog
genes. The option \option{-null} was used to specify that the string
\texttt{<NA>} should be used to indicate the absence of putative
orhtolog.

Another option would be to obtain a ``quantitative'' profile, where
each cell indicates the E-value of the match between the two
orthologs. This can be done by specifying a different score column
with the option \option{-scol} of \program{convert-classes}.

\begin{small}
\begin{verbatim}

## Convert ortholog table into a profile table
## with the E-value of the putative orthologs
convert-classes -v 2 \
  -i Saccharomyces_cerevisiae_Fungi_bbh.tab  \
  -from tab -to profiles \
  -ccol 2 -mcol 3 -scol 4 -null "<NA>" \
  -o Saccharomyces_cerevisiae_Fungi_phyloprofiles_evalue.tab

\end{verbatim}
\end{small}

\subsection{Detecting pairs of genes with similar phylogenetic profiles}

\textbf{TO BE WRITTEN}


\begin{small}
\begin{verbatim}
compare-profiles -v 1 \
  -i profiles.tab \
  -return counts,jaccard,hyper,entropy \
  -o phyloprof_gene_pairs.tab
\end{verbatim}
\end{small}