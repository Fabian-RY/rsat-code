%%%%%%%%%%%%%%%%%%%%%%%%%%%%%%%%%%%%%%%%%%%%%%%%%%%%%%%%%%%%%%%%
%%%% Downloading organisms from RSAT data repository
%%%%%%%%%%%%%%%%%%%%%%%%%%%%%%%%%%%%%%%%%%%%%%%%%%%%%%%%%%%%%%%%

\chapter{Downloading genomes}
\label{downloading_genomes}

\RSAT includes a series of tools to install and maintain the latest
version of genomes.

The most convenient way to add support for one or several organisms on
your machine is to use the programs \program{supported-organisms} and
\program{download-organism}.

Beware, the complete data required for a single genome may occupy
several hundreds of Mb, because \RSAT not only stores the genome
sequence, but also the oligonucleotide frequency tables used to
estimate background models, and the tables of BLAST hits used to get
orthologs for comparative genomics. If you want to install many
genomes on your computer, you should thus reserve a sufficient amount
of space.

\section{Original data sources}

Genomes supported on \RSAT were obtained from various sources.

Genomes can be installed either from the \RSAT web site, or from their
original sources.  

\begin{itemize}
\item NCBI/Genbank (\url{ftp://ftp.ncbi.nih.gov/genomes/})

\item UCSC (\url{http://genome.ucsc.edu/}) for the multi-genome
  alingment files (multiz) used by \program{peak-footprints}.

\item ENSEMBL (\url{http://www.ensembl.org/}) genomes are supported by
  special tools (\program{retrieve-ensembl-seq},
  \program{supported-organisms-ensembl}).


\item The EBI genome directory
  (\url{ftp://ftp.ebi.ac.uk/pub/databases/genomes/Eukaryota/})

\end{itemize}

Other genomes can also be found on the web site of a diversity of
genome-sequencing centers.

\section{Requirement : wget}

The download of genomes relies on the application \program{wget},
which is part of linux distribution\footnote{For Linux:
  \url{http://www.gnu.org/software/wget/}; for Mac OSX
  \url{http://download.cnet.com/Wget/3000-18506_4-128268.html}}.


\program{wget} is a ``web aspirator'', which allows to download whole
directories from ftp and http sites. You can check if the program is
installed on your machine.

\begin{lstlisting}
wget --help
\end{lstlisting}


This command should return the help pages for \program{wget}.  If you
obtain an error message (``command not found''), you need to ask your
system administrator to install it.

\section{Importing organisms from the \RSAT main server}

The simplest way to install organisms on our \RSAT site is to download
the RSAT-formatted files from the web server. For this, you can use a
web aspirator (for example the program \program{wget}). 

Beware, the full installation (including Mammals) requires a large
disk space (several dozens of Gb). You should better start installting
a small genome and test it before processing to the full
installation. We illustrate the approach with the genome of our
preferred model organism: the yeast \textit{Saccharomyces cerevisiae}.

\subsection{Obtaining the list of organisms supported on the \RSAT server}

By default, the program \program{supported-organisms} returns the list
of organisms supported on your local \RSAT installation. You can
however use the option \option{-server} to obtain the list of
organisms supported on a remote server.


\begin{lstlisting}
supported-organisms -server
\end{lstlisting}

The command can be refined by restricting the list to a given taxon of
interest.

\begin{lstlisting}
supported-organisms -server -taxon Fungi
\end{lstlisting}

You can also ask additional information, for example the date of the
last update and the source of each genome.

\begin{lstlisting}
supported-organisms -server -taxon Fungi -return last_update,source,ID
\end{lstlisting}


\subsection{Importing a single organism}


The command \command{download-organism} allows you to download one or
several organisms. 

Beware, the complete data for a single genome may occupy several tens
of Megabytes (Bacterial genomes) or a few Gigabases
(Mammalian). Downloading tenomes thus requires a fast Internet
connection, and may take time. If possible, please download genomes
during the night (European time).

As a first step, we recommend to download the genome of the yeast
\org{Saccharomyces cerevisiae}, since this is the model organism used
in our tutorials.


\begin{lstlisting}
download-organism -v 1 -org Saccharomyces_cerevisiae
\end{lstlisting}

In principle, the download should start immediately. \emph{Beware},
the data volume to be downloaded is important, because the genome
comes together with extra files (blast hits with other genoems,
oligonucleotide and dyad frequencies). Depending on the network
bandwidth, the download of a genome may take several minutes or tens
of minutes. 

After the task is completed, you can check if the configuration file
has been correctly updated by typing the command.

\begin{lstlisting}
supported-organisms
\end{lstlisting}

In principle, the following information should be displayed on your
terminal.

\result{Saccharomyces\_cerevisiae}

You can also add parameters to get specific information on the
supported organisms.

\begin{lstlisting}
supported-organisms -return ID,last_update
\end{lstlisting}


\subsection{Importing a few selected organisms}

The program \program{download-organism} can be launched with a list of
organisms by using iteratively the option \option{-org}.


\begin{lstlisting}
download-organism -v 1 -org Escherichia_coli_K12 -org Salmonella_typhi
\end{lstlisting}

\subsection{Importing all the organisms from a given taxon}

For comparative genomics, it is convenient to install on your server
all the organisms of a given taxon. For this, you can simply use the
option \option{-taxon} of \program{download-organism}.

Before doing this, it is wise to check the number of genomes that
belong to this taxon on the server.

\begin{lstlisting}
## Get the list of organisms belonging to the order "Enterobacteriales" on the server
supported-organisms -taxon Enterobacteriales -server

## Count the number of organisms
supported-organisms -taxon Enterobacteriales -server | wc -l
\end{lstlisting}

In Oct 2009, there are 94 Enterobacteriales supported on the \RSAT
server. Before starting the download, you should check two things:
\begin{enumerate}
\item Has your network a sufficient bandwidth to ensure the transfer
  in a reasonable time ?
\item Do you have enough free space on your hard drive to store all those genomes ? 
\end{enumerate}

If the answer to both questions is ``yes'', you can start the
download.

\begin{lstlisting}
download-organism -v 1 -taxon Enterobacteriales 
\end{lstlisting}


\section{Adding support for Ensembl genomes}

In addition to the genomes imported and maintained on your local \RSAT
server, the program \program{retrieve-ensembl-seq} allows you to
retrieve sequences for any organism supported in the Ensembl database
(\url{http://ensembl.org}).

For this, you first need to install the Bioperl and Ensembl Perl
libraries (see section \ref{sect:ensembl_libraries}).

\subsection{Handling genomes from Ensembl}

The first step to work with Ensembl genomes is to check the list of
organisms currently supported on their Web server.

\begin{lstlisting}
supported-organisms-ensembl
\end{lstlisting}

You can then get more precise information about a given organism
(build, chromosomes) with the command \program{ensembl-org-info}.

\begin{lstlisting}
ensembl-org-info -org Drosophila_melanogaster
\end{lstlisting}

Sequences can be retrieved from Ensembl with the command
\program{retrieve-ensembl-seq}. 

You can for example retrieve the 2kb sequence upstream of the
transcription start site of the gene \gene{PAX6} of the mouse. 

\begin{lstlisting}
retrieve-ensembl-seq.pl -org Mus_musculus -q PAX6 \
  -type upstream -feattype mrna -from -2000 -to -1 -nogene -rm \
  -alltranscripts -uniqseqs
\end{lstlisting}

Options

\begin{itemize}

\item \option{-type upstream} specifies that we want to collect the
  sequences located upstream of the gene (more procisely, upstream of
  the mRNA).

\item \option{-feattype mrna} indicates that the reference for computing
  coordinates is the mRNA. Since we collect upstream sequences, the
  5'most position of the mRNA has coordinate 0, and upstream sequences
  have negative coordinates. Note that many genes are annotated with
  multiple RNAs for different reasons (alternative splicing,
  alternative transcription start sites). By default, the program will
  return the sequences upstream of each mRNA annotated for the query
  gene.

\item \option{-nogene} clip the sequences to avoid overlapping the next
  upstream gene.

\item \option{-rm} repeat masking (important for pattern
  discovery). Repetitive sequences are replaced by \seq{N} characters.

\end{itemize} 

\section{Importing multi-genome alignment files from UCSC}

\subsection{Warning: required disk space}

The UCSC multi-genome alignment files occupy a hige disk space. The
alignments of 30 vertebrates onto the mouse genome (mm9 multiz30)
requires 70Gb. If you intend to offer support for multi-genome
alignments, it might be safe acquiring a separate hard drive for this
data.

The complete data set available at UCSC in April 2012 occupies 1Tb in
compressed form, and probably 7 times more once uncompressed. For
efficiency reasons, it is necessary to uncompress these files for
using them with the indexing system of  \program{peak-footprints}.

\subsection{Checking supported genomes at UCSC}

As a first step, we will check the list of supported genomes at the
UCSC Genome Browser.

\begin{lstlisting}
supported-organisms-ucsc
\end{lstlisting}

Each genome is assocaited with a short identifier, followed by a
description. For example, several versions of the mouse genome are
currently available.


\begin{small}
\begin{verbatim}
mm10	Mouse Dec. 2011 (GRCm38/mm10) Genome at UCSC
mm9	Mouse July 2007 (NCBI37/mm9) Genome at UCSC
mm8	Mouse Feb. 2006 (NCBI36/mm8) Genome at UCSC
mm7	Mouse Aug. 2005 (NCBI35/mm7) Genome at UCSC
\end{verbatim}
\end{small}

\subsection{Downloading multiz files from UCSC}

Multi-genome alignments at UCSC are generated with the program
\program{multiz}, which produces files in a custom text format called
\concept{maf} for Multi-Alignment file.

We show hereafter the command to download the mm9 version of the mouse
genome, and install it in the proper directory for
\program{peak-footprints} (\file{\$RSAT/data/UCSC\_multiz}).

\begin{lstlisting}
download-ucsc-multiz -v 1 -org mm9
\end{lstlisting}

The program will create the sub-directory for the mm9 genome, download
the coresponding compressed multiz files (files with extension
\file{.maf.gz}), uncompress them, and call \program{peak-footprint}
with specific options in order to create a position index, which will
be further used for fast retrieval of the conserved regions under
peaks.





