\chapter{KEGG network provider}

\section{Introduction}

% what can you do with the tool
KEGG network provider allows you to extract metabolic networks from KEGG \cite{Kanehisa2008} that are specific to a set of organisms.
In addition, you can exclude certain compounds or reactions from these networks.

% competing tools
A range of tools works with KGML files. Click on ``Manual -> Related tools'' to see a selection of them.
KEGG network provider differs from these tools by allowing also the extraction of RPAIR networks and by supporting
filtering of compounds, reactions and RPAIR classes.

KEGG network provider itself has no network analysis or visualization functions,
but you can use a NeAT tool (a choice of them will appear upon termination of network construction)
or any other graph analysis tool that reads gml, VisML or dot format for these purposes.

For visualization of KEGG networks, you can use iPATH \cite{ipath}, KGML-ED \cite{kgmled} or
metaSHARK \cite{metashark}.
Yanasquare \cite{yana} and Pathway Hunter Tool \cite{pht} offer
organism-specific KEGG network construction in combination
with analysis functions.

% side compounds
It should be noted that KEGG annotators omitted side compounds in the KGML files. Thus, certain molecules
(such as CO2, ATP or ADP) might be absent from the metabolic networks extracted from these files.

% metabolic network reconstruction
It is also worth noting that constructing metabolic networks from KGML files produces networks of much lower quality
than those obtained by manual metabolic reconstruction. In manual reconstruction, several resources are taken into account,
such as the biochemical literature, databases and genome annotations (e.g. \cite{Foerster2003}). This is why
the metabolism of only a few organisms has been manually reconstructed so far.\\
In automatically reconstructed networks, reactions might not be balanced and compounds might occur more
than once with different identifiers (see e.g. \cite{Poolman2006} for annotation problems in KEGG).
For the purpose of path finding the automatically reconstructed metabolic networks may still be of interest.

\section{Construction of yeast and E. coli metabolic networks}

\subsection{Study case}

Our study case consists in the construction of two metabolic networks: one for five yeast species and the other for
\textit{Escherichia coli K-12 MG1655}.
We will compare path finding results obtained for these two networks for a metabolic reference pathway (Lysine biosynthesis).

\subsection{Protocol for the web server}

\begin{enumerate}

\item In the \neat  menu, select the entry \program{Download organism-specific networks from KEGG}.

  In the right panel, you should now see a form entitled
  ``KEGG network provider''.

\item Click on the button \option{DEMO} located at the bottom of the form.

  The KEGG network provider form has now loaded the organism identifiers of five yeast species. As explained
  in the form, the species concerned are: \textit{Saccharomyces bayanus}, \textit{Saccharomyces mikatae},
  \textit{Saccharomyces paradoxus}, \textit{Schizosaccharomyces pombe} and \textit{Saccharomyces cerevisiae}.

\item Click the checkbox \option{directed network} to construct a directed metabolic network.

\item Click on the button \option{GO}.

  The network extraction should take only a few seconds.
  Then, a link to the extracted network is displayed.
  In addition (for formats \textit{tab-delimited} and \textit{gml}), the Next step panel should appear.

% step 5
\item Click on the button ``Find metabolic paths in this graph'' in the Next step panel. This button opens the
Metabolic pathfinder with the yeast network pre-loaded.

\item Enter C00049 (L-Aspartate) as source node and C00047 (L-Lysine) as target node.

\item In section \textbf{Path finding options}, set the rank to 1. We are only interested in the first rank.

\item In section \textbf{Output}, select \option{Graph} as output with ``paths unified into one graph''

\item Click \option{GO}.
  The seed node selection form appears to confirm our seed node choice.

% step 10
\item Click \option{GO}.
	After no more than one minute of computation, the graph unifying first rank paths between L-aspartate and L-lysine should appear.
	You can store the graph image on your machine for later comparison.

\end{enumerate}

Repeat the previous steps, but instead of selecting \option{DEMO} in the KEGG network provider form, enter eco
in the organisms text input field. Make sure to select \option{directed network} in the KEGG network provider form,
then follow steps 4 to 10 as described above.

\subsection{Protocol for the command-line tools}

You may download the command line version of the KEGG network provider, which also includes the path finding tool,
from the manual page of the KEGG network provider (section: standalone tool). \\

The command-line version of this tutorial is restricted to the \textit{E. coli} and \textit{S. cerevisiae}
metabolic networks. It is assumed that you have installed the required command-line tools.

\begin{enumerate}

\item First we construct the directed metabolic network of \textit{E. coli}.

        {\color{Blue} \begin{footnotesize}
		\begin{verbatim}
		java graphtools.util.MetabolicGraphProvider -i eco -d -o eco_metabolic_network_directed.txt
		\end{verbatim} \end{footnotesize}
		}

\item Then, we search for the lightest paths in this network as follows:

		{\color{Blue} \begin{footnotesize}
		\begin{verbatim}
		java graphtools.algorithms.Pathfinder -g eco_metabolic_network_directed.txt -f tab -s C00049 -t C00047 -r 1 -y con -b
		 -n -c -e ExclusionAttribute -T pathsUnion -O gml -o lysinebiosyn_eco.gml
		\end{verbatim} \end{footnotesize}
		}

\item To visualize the inferred pathway, you may open lysinebiosyn\_eco.gml in Cytoscape or in yED.


\item We proceed by constructing the metabolic network of \textit{S. cerevisiae}:

	 {\color{Blue} \begin{footnotesize}
		\begin{verbatim}
		java graphtools.util.MetabolicGraphProvider -i sce -d -o sce_metabolic_network_directed.txt
		\end{verbatim} \end{footnotesize}
		}

\item Then, we enumerate paths between L-aspartate and L-lysine in it:

		{\color{Blue} \begin{footnotesize}
		\begin{verbatim}
		java graphtools.algorithms.Pathfinder -g sce_metabolic_network_directed.txt -f tab -s C00049 -t C00047 -r 1 -y con -b
		 -n -c -e ExclusionAttribute -T pathsUnion -O gml -o lysinebiosyn_sce.gml
		\end{verbatim} \end{footnotesize}
		}

\item As before, we can visualize the lysinebiosyn\_sce.gml file in a graph editor capable of reading gml files
     (such as yED or Cytoscape).

\end{enumerate}

\subsection{Interpretation of the results}

After having executed the steps of this tutorial, you should have obtained two pathway images:
one for the yeast network and one for the \textit{E. coli} network. Both pathways differ quite substantially.
If we compare each of these pathways with the respective organism-specific pathway map in KEGG, we notice that
the pathway inferred for the \textit{E. coli} network reproduces the reference pathway correctly.\\
The yeast pathway deviates from the \textit{S. cerevisiae} KEGG pathway map from L-aspartate to but-1-ene-1,2,4-tricarboxylate,
but recovers otherwise the reference pathway correctly (ignoring the intermediate steps 5-adenyl-2-aminoadipate and
alpha-aminoadipoyl-S-acyl enzyme associated to EC number 1.2.1.31). \\
For comparison purposes, we have chosen the same start and end compound for both metabolic networks, but it should
be noted that the reference lysine biosynthesis pathway in \textit{S. cerevisiae} starts from 2-oxoglutarate.

The lysine biosynthesis KEGG map for yeast is available at:
\begin{verbatim}
http://www.genome.ad.jp/dbget-bin/get_pathway?org_name=sce&mapno=00300
\end{verbatim}

The one for \textit{E. coli} is available at:
\begin{verbatim}
http://www.genome.ad.jp/dbget-bin/get_pathway?org_name=eco&mapno=00300
\end{verbatim}


\subsection{Summary}

The study case demonstrated that different organisms may employ different metabolic pathways for the synthesis or
degradation of a given compound. For this reason, it is useful to be able to construct metabolic networks that are
specific to a selected set of organisms.

\subsection{Troubleshooting}

\begin{enumerate}

\item An empty graph (with zero nodes and edges) is returned.
Make sure that the entered organism identifiers are valid in KEGG.
They should consist of three to four letters only. If in doubt,
check in the provided KEGG organism list.

\end{enumerate}
