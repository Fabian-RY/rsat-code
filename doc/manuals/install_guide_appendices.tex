%% This file contains some details about installation steps, which
%% were previously part of the installation procedure, but are not
%% required anymore since I (Jacques vna Helden) autmoated the
%% procedure.
%%
%% I however maintain this appendix to keep track of the motivation
%% for installing different pieces.

\chapter{Appendices to the installation guide}

\section{List of Perl modules required for full functionalities of
  \RSAT}

For information, we describe hereafter the list of modules that will
be installed with this command, and the reason why it is useful to
install them before running \RSAT programs.

If you are not interested by technical details, you can skip this
section.


\begin{enumerate}
\item \program{GD.pm} Interface to Gd Graphics Library. Used by
  \program{XYgraph} and \program{feature-map}. It is highly recommended to install this perl module manually on MacOS X, as it usually fail with CPAN: 
  
  \begin{lstlisting}
 curl -O  http://search.cpan.org/CPAN/authors/id/L/LD/LDS/GD-2.49.tar.gz
 tar -vxzf GD-2.49.tar.gz
 cd GD-2.49.tar.gz
 sudo perl Makefile.PL
 sudo make
 sudo make install
\end{lstlisting}

\item \program{PostScript::Simple} Produce PostScript files from
  Perl. Used by \program{feature-map}.

\item \program{Util::Properties} is required to load property files,
  which are used to specify the site-specific configuration of your
  \RSAT server. Property files are also useful to write your own perl
  clients for the Web service interface to \RSAT (RSATWS).
  
  Note : CPAN will require to install first the following requisites:
  
  \program{Digest::MD5::File}
  
  \program{IO::All}
  
  \program{LockFile::Simple}
  
  \program{Object::InsideOut}

\item 
  The following modules are required for the Web services. 

  \begin{itemize}
    \item \program{XML::Compile}
    \item \program{XML::Compile::Cache}
    \item \program{XML::Compile::SOAP11}
    \item \program{XML::Compile::WSDL11}
    \item \program{XML::Compile::Transport::SOAPHTTP}
    \item \program{XML::Parser::Expat}
    \item \program{SOAP::WSDL} 
    \item \program{SOAP::Lite}
    Note: SOAP::Lite usually does not compile properly. Use the following command:
    perl -MCPAN -e "CPAN::Shell->force(qw(install SOAP::Lite));"
    \item \program{Module::Build::Compat}
    \item \program{Util::Properties} (already mentioned above) 
  \end{itemize}

  \RSAT Web services is a convenient interface that permits to write
  Perl scripts to run \RSAT queries on a remote server. Some Perl
  scripts of the \RSAT stand-alone commands are using SOAP to connect
  remote servers (e.g. \program{supported-organisms},
  \program{download-organism}).

\item \program{Statistics::Distributions} is used to calculate some
  probability distribution functions. In particular, it is used by the
  programs \program{position-analysis} and \program{chi-square} to
  calculate the chi2 P-value.

  \textbf{Notes} 
  \begin{itemize}
  \item In previous releases, the chi2 P-value was computed using
    \program{Math::CDF}, but the precision was limited to
    1e-15. \program{Statistics::Distributions} can compute P-values
    down to 1e-65.
  \item For the discrete functions (binomial , Poisson,
    hypergeometric) \RSAT relies on a custom library
    (\$RSAT/perl-scripts/lib/RSAT/stats.pm) which reaches a precision
    of 1e-300.
  \end{itemize}


\item \program{Class::Std::Fast} and \program{Storable} are required
  to bring persistence to data structures like organisms. This library
  can be easyly installed via CPAN.

\item \program{File::Spec}, \program{POSIX} and \program{Data::Dumper}
  are required for some functions of \program{matrix-scan}.

\item \program{XML::LibXML} is required for parsing and writing XML and uses
  the XML::Parser::Expat library. It is necessary for some RSAT
  applications.

\item \program{DBI}  and \program{DBD::mysql}: those two libraries are
  required by the program \program{retrieve-ensembl-seq} in order to
  access the ENSEMBL database.
  
  Note that the
installation of the CPAN module DBD:mysql may require a prior
installation of a MySQL client on your machine
(\url{http://dev.mysql.com/downloads/}). Choose MySQL Community Server.

On Mac OSX, there seems to be problems with the mysql bundle,

- for mysql\_config issues, specify the path to mysql\_config (adapt the path to your own installation):
\begin{verbatim}
	cd /Users/yourusername/.cpan/build/DBD-mysql-version-xxxxx
	perl Makefile.pl --mysql_config=/usr/local/mysql/bin/mysql_config 
\end{verbatim}

-then adding a soft link as follows\footnote{\url{http://www.blog.bridgeutopiaweb.com/post/how-to-fix-mysql-load-issues-on-mac-os-x/}}:
\begin{verbatim}
sudo ln -s /usr/local/mysql/lib/libmysqlclient.18.dylib /usr/lib/libmysqlclient.18.dylib
\end{verbatim}
 - finish with 
\begin{verbatim}
make
make test
make install
\end{verbatim}

\item \program{Bio::Perl} is required for the Ensembl API, which in
  turn is required for handling genomes installed on the ENSEMBL
  database (\url{http://www.ensembl.org/}).

\item \program{IPC::Run} is required for the analysis of linkage
  desequilibrium with ENSEMBL API
  \footnote{\url{http://cvs.sanger.ac.uk/cgi-bin/viewvc.cgi/ensembl-variation/C\_code/README.txt?root=ensembl&view=markup}}.

\item \program{Bio::Das} is required to retrieve sequences from DAS
  servers with the program \program{bed-to-seq}.

\item \program{LWP::Simple} is used by the Web interface, to fetch
  sequences from remote servers (URL of the sequence file specified in
  a text field).

\end{enumerate}


\section{Details about \RSAT compilation}

\subsection{Compiling \RSAT programs in a different directory}

The installation directory can be changed by redefining the BIN
variable. For instance, if you have the system adminstrator
privileges, you could install the compiled programs in the standard
directory for compiled packages (\file{/usr/local/bin}).

\begin{lstlisting}
cd $RSAT;
make -f makefiles/init_rsat.mk compile_all BIN=/usr/local/bin
\end{lstlisting}

\subsection{\RSAT programs requiring compilation}

\begin{itemize}

\item \program{count-words}: an efficient algorithm for counting word
  occurrences in DNA sequences. This program is much faster than
  \program{oligo-analysis}, but it only returns the occurrences and
  frequencies, whereas oligo-analysis returns over-representation
  statistics and supports many additional
  options. \program{count-words} is routinely used to compute word
  frequencies in large genome sequences, for calibrating the Markov
  models used by \program{oligo-analysis}.

\item \program{matrix-scan-quick}: an efficient algorithm for scanning
  sequences with a position-specific scoring matrix. As its name
  indicates, \program{matrix-scan-quick} is \emph{much} faster than
  the Perl script \program{matrix-scan}, but presents reduced
  functionalities (only computes the weight, returns either a list of
  sites or the weight score distribution).

\item \program{info-gibbs}: a gibbs sampling algorithm based on
  optimization of the the information content of the motif
  \cite{Defrance:2009}.
\end{itemize}
