
%%%%%%%%%%%%%%%%%%%%%%%%%%%%%%%%%%%%%%%%%%%%%%%%%%%%%%%%%%%%%%%%
%%%% Accessing the programs
%%%%%%%%%%%%%%%%%%%%%%%%%%%%%%%%%%%%%%%%%%%%%%%%%%%%%%%%%%%%%%%%
\section{Accessing the programs}

In order to use the shell version of \RSAT, you first need an account
on a unix machine where \RSAT is installed, and you should know the
directory where \RSAT have been installed. (if you don't know, ask
assistance to your system administrator).

For this tutorial, let us assume that \RSAT is installed in the
directory \texttt{/usr/local/rsa-tools}

\begin{enumerate}

\item Open a telnet or ssh session to your account.

\item If your default shell is \textbf{tcsh}, type the following
commands.

\begin{verbatim}
set RSAT=/usr/local/rsa-tools
set path=($path $RSAT/perl-scripts)
set path=($path $RSAT/bin/)
rehash
\end{verbatim}

If you are using a different shell (e.g. bash), you might need a
slightly different command to obtain the same result. See you system
manager in case of doubt.

\item The previous step should have included all the \RSAT programs in
your path.  To check if it worked, just type:

\begin{verbatim}
random-seq -l 350
\end{verbatim}

If your configuration is correct, this command should return a random
sequence of 350 nucleotides.

\end{enumerate}

You are now able to use any program from the \RSAT package, untill you
quit your telnet session. It is however not very convenient to set the
path manually each time you open a new connection. You can modify your
defaul configuration by including the above commands in the file
\texttt{.personal-cshrc} in the root of your home directory. If you
don't know how to modify this file, see the system adiministrator.
