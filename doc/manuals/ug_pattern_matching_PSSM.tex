%%%%%%%%%%%%%%%%%%%%%%%%%%%%%%%%%%%%%%%%%%%%%%%%%%%%%%%%%%%%%%%%
%%%% MATRIX-BASED PATTERN MATCHING
%%%%%%%%%%%%%%%%%%%%%%%%%%%%%%%%%%%%%%%%%%%%%%%%%%%%%%%%%%%%%%%%

\chapter{Matrix-based pattern matching}

\section{patser (program developed by by Jerry Hertz)}

We will now see how to match a profile matrix against a sequence
set. For this, we use \textit{patser}, a program written by Jerry
Hertz. 

\subsection{Getting help}

help can be obtained with the two usual options.

{\color{Blue} \begin{footnotesize} 
\begin{verbatim}
patser -h
patser -help
\end{verbatim} \end{footnotesize}
}

\subsection{Matrix conversion}

Patser expects as input a matrix like the 4 matrices we obtained above
with \textit{consensus}. The output from \textit{consensus} can however
not be used directly because it contains several matrices, and a lot
of additional information. One possibility is to copy-paste the matrix
of interest to a separate file.

To avoid manual editing, RSAT contains a program
\textit{convert-matrix}, which automaticaly extracts a matrix from
various file formats, including consensus.

{\color{Blue} \begin{footnotesize} 
\begin{verbatim}
convert-matrix -format consensus -i PHO_consensus \
    -return counts -o PHO_matrix 

more PHO_matrix
\end{verbatim} \end{footnotesize}
}

Note that the program \textit{convert-matrix} includes many other
output options, fo example you can obtain the degenerate consensus
from a matrix with the following options.

{\color{Blue} \begin{footnotesize} 
\begin{verbatim}
convert-matrix -format consensus -i PHO_consensus \
    -return consensus
\end{verbatim} \end{footnotesize}
}

Additional information can be otbained with the on-line help for
\textit{convert-matrix}.

{\color{Blue} \begin{footnotesize} 
\begin{verbatim}
convert-matrix -h
\end{verbatim} \end{footnotesize}
}

\subsection{Detecting Pho4p sites in the PHO genes}

After having extracted the matrix, we can match it against the PHO
sequences to detect putative regulatory sites.

{\color{Blue} \begin{footnotesize} 
\begin{verbatim}
patser -m PHO_matrix -f PHO_up800.wc -A a:t c:g -c -ls 9
\end{verbatim} \end{footnotesize}
}

\subsection{Detecting Pho4p sites in all upstream regions}

We will now match our PHO matrix against the whole set of upstream
regions from the 6200 yeast genes. This should allow us to detect new
genes potentially regulated by Pho4p.

One possibility would be to use \textit{retrieve-seq} to extract all
yeast upstream regions, and save the result in a file, which will then
be used as input by \textit{patser}. To avoid occupying too much space
on the disk, we could combine both tasks in a single command, and
immediately redirect the output of \textit{retrieve-seq} as input for
\textit{patser}. This can be done with the pipe character | as
below.

\textit{patser} result can be redirected to a file with the unix
``greater than'' ($>$) symbol. We will store the result of the
genome-scale search in a file \file{PHO\_matrix\_matches\_allup.txt}.

{\color{Blue} \begin{footnotesize} 
\begin{verbatim}
retrieve-seq -type upstream -from -1 -to -800  \
    -org Saccharomyces_cerevisiae \
    -all -format wc -label gene  \
    | patser -m PHO_matrix -ls 9 -A a:t c:g \
    > PHO_matrix_matches_allup.txt

more PHO_matrix_matches_allup.txt
\end{verbatim} \end{footnotesize}
}

\section{Scanning sequences with \program{matrix-scan}}

The program \program{matrix-scan} allows to scan sequences with a
position-specific scoring matrix (\concept{PSSM}), in the same way as
patser. However, it presents some differences:

\begin{enumerate}

\item \program{matrix-scan} is much slower than \program{patser},
  because it is a perl script (whereas \program{patser} is
  compiled). However, for most tasks, we can affor dto spend a few
  minuts per genome rather than a few seconds.

\item \program{matrix-scan} does not (yet) calculate the P-value
  associated to each match. I intend to implement it in a next
  version.

\item \program{matrix-scan} supports higher-order Markov chain models,
  whereas \program{paters} only supports Bernoulli models. The markov
  models can be defined from different sequence sets: external
  sequences, input sequences, or even locally (\concept{adaptive
    background models}).

\end{enumerate}



