%%%%%%%%%%%%%%%%%%%%%%%%%%%%%%%%%%%%%%%%%%%%%%%%%%%%%%%%%%%%%%%%
%%%% PATTERN MATCHING
%%%%%%%%%%%%%%%%%%%%%%%%%%%%%%%%%%%%%%%%%%%%%%%%%%%%%%%%%%%%%%%%
\section{Pattern matching}

In a pattern matching problem, you start from one or several
predefined patterns, and you match this pattern against a sequence,
i.e. you locate all occurrences of this pattern in the sequences.

Patterns can be represented as strings (with \textit{dna-pattern}) or
position-weight matrices (with \textit{patser}). 

\subsection{dna-pattern}

\textit{dna-pattern} is a string-based pattern matching program,
specialized for searching patterns in DNA sequences. 

\begin{itemize}

\item 
This specialization mainly consists in the ability to search on both
the direct and reverse complement strands.

\item 
A single run can either search for a single pattern, or for a list
of patterns.

\item 
multi-sequence file formats (fasta, filelist, wc, ig) are supported,
allowing to match patterns against a list of sequences with a single
run of the program.

\item 
String descriptions can be refined by using the 15-letters IUPAC code
for uncompletely specified nucleotides, or by using regular
expressions.

\item 
The program can either return a list of matching positions (default
behaviour), or the count of occurrences of each pattern.

\item 
Imperfect matches can be searched by allowing
substitutions. Insertions and deletions are not supported.  The reason
is that, when a regulatory site presents variations, it is generally
in the form of a tolerance for substitution at a specific position,
rather than insertions or deletions. It is thus essential to be able
distinguishing between these types of imperfect matches.

\end{itemize}


\subsubsection{Matching a single pattern}

We will start by searching all positions of a single pattern in a
sequence set. The sequence is the set of upstream regions from the PHO
genes, that was obtained in the tutorial on sequence retrieval. We
will search all occurrences of the most conserved core of the Pho4p
medium affinity binding site (\texttt{CACGTT}) in this sequence set.

Try the following command:

\begin{verbatim}
dna-pattern -i PHO_up800.fasta -format fasta \ 
    -1str -p cacgtt -id 'Pho4p_site'
\end{verbatim}

You see a list of positions for all the occurrences of CACGTT in the sequence.

Each row represents one match, and the columns provide the following
information:
\begin{enumerate}
\item pattern identiifier
\item strand
\item pattern searched
\item sequence identifier
\item start position of the match
\item end position of the match
\item matched sequence
\item matching score
\end{enumerate}

\subsubsection{Matching on both strands}

To perform the search on both strands, type:
\begin{verbatim}
dna-pattern -i PHO_up800.fasta -format fasta \ 
    -2str -p cacgtt -id 'Pho4p_site'
\end{verbatim}

Notice that the strand colmn now contains two possible values: D for
``direct'' and R for ``reverse complement''.

\subsubsection{Allowing substitutions}

To allow one substitutions, type:
\begin{verbatim}
dna-pattern -i PHO_up800.fasta -format fasta \
    -2str -p cacgtt -id 'Pho4p_site' -subst 1
\end{verbatim}

Notice that the score column now contains 2 values: 1.00 for perfect
matches, 0.83 (=5/6) for single substitutions. This si one possible
use of the score column: when substitutions are allowed, the score
indicates the percentage of matching nucleotides.

Actually, for regulatory patterns, allowing substitutions usually
returns many false positive, and this option is usually avoided. We
will not use it further in the tutorial.

\subsubsection{Extracting flanking sequences}

The matching positions can be extracted along with their flanking nucleotides. Try:

\begin{verbatim}
dna-pattern -i PHO_up800.fasta -format fasta \
     -2str -p cacgtt \
     -id  'Pho4p_site' -N 4
\end{verbatim}

Notice the change in the matched sequence column: each matched
sequence contains the pattern CACGTT in uppercase, and 4 lowercase
letters on each side (the flanks).

\subsubsection{Changing the origin}

When working with upstream sequences, it is convenient to work with
coordinates relative to the start codon (i.e. the right side of the
sequence). Sequence matching programs (including dna-pattern) return
the positions relative to the beginning (i.e. the left side) of the
sequence. The reference (coordinate 0) can however be changed iwith
the option \texttt{-origin}. In this case, we retrieved upstream
sequences over 800bp. the start codon is thus located at position
801. Try:

\begin{verbatim}
dna-pattern -i PHO_up800.fasta -format fasta \
    -2str -p cacgtt \
    -id  'Pho4p_site' -N 4 -origin 801
\end{verbatim}

Notice the change in coordinates.

In some cases, a sequence file will contain a mixture of sequences of
different length (for example if one clipped the sequences to avoid
upstream coding sequences). The origin should thus vary from sequence
to sequence. A convenient way to circumvent the problem is to use a
egative value with the option \texttt{origin}. for example,
\texttt{-origin -100} would take as origin the 100th neucleotide
starting from the right of each sequence in the sequence file. But in
our case we want to take as origin the position immediately after the
last nucleotide. For this, there is a special convention: \texttt{-origin
-0}.

\begin{verbatim}
dna-pattern -i PHO_up800.fasta -format fasta \
    -2str -p cacgtt \
    -id  'Pho4p_site' -N 4 -origin -0
\end{verbatim}

In the current example, since all sequences have exactly 800bp length,
the result is identical to the one obtained with \texttt{-origin 801}.

\subsubsection{Matching degenerate patterns}

As we said before, there are two forms of Pho4p binding sites: the
protein has high affinity for motifs containing the core CACGTG, but
can alos bind, with a medium affinity, CACGTT sites. The IUPAC code
for partly specified nucleotides allows to represent any combination
of nucleotids by a single letter.

\begin{tabular}{lll}
\textbf{A} &  & (Adenine) \\
\textbf{C} &  & (Cytosine) \\
\textbf{G} &  & (Guanine) \\
\textbf{T} &  & (Trymine) \\
\textbf{R} & = A or G & (puRines) \\
\textbf{Y} & = C or T & (pYrimidines) \\
\textbf{W} & = A or T & (Weak hydrogen bonding) \\
\textbf{S} & = G or C & (Strong hydrogen bonding) \\
\textbf{M} & = A or C & (aMino group at common position) \\
\textbf{K} & = G or T & (Keto group at common position) \\
\textbf{H} & = A, C or T & (not G) \\
\textbf{B} & = G, C or T & (not A) \\
\textbf{V} & = G, A, C & (not T) \\
\textbf{D} & = G, A or T & (not C) \\
\textbf{N} & = G, A, C or T & (aNy) \\
\end{tabular}

Thus, we could use the string CACGT\textbf{K} to represent the Pho4p
consensus, and search both high and medium affinity sites in a single
run of the program.

\begin{verbatim}
dna-pattern -i PHO_up800.fasta -format fasta \
    -2str -p cacgtk \
    -id  'Pho4p_site' -N 4 -origin -0
\end{verbatim}

\subsubsection{Matching regular expressions}

Another way to represent partly specified strings is by using regular
expressions. This not only allows to represent combinations of letters
as we did above, but also spacings of variable width. For example, we
could search for tandem repeats of 2 Pho4p binding sites, separated by
less than 100bp. This can be represented by the following regular expression: 

\begin{verbatim}
cacgt[gt].{0,100}cacgt[gt]
\end{verbatim}

which means
\begin{itemize}
\item cacgt 
\item followed by either g or t [gt]
\item followed by 0 to 100 unspecified letters .{0,100}
\item followed by cacgt
\item followed by either g or t [gt]
\end{itemize}

Let us try to use it with dna-pattern

\begin{verbatim}
dna-pattern -i PHO_up800.fasta -format fasta \
    -2str -id 'Pho4p_pair' \
    -N 4 -origin -0 \
    -p 'cacgt[gt].{0,100}cacgt[gt]'
\end{verbatim}


Note that the pattern has to be quoted, to avoid possible conflicts
between special characters used in the regular expression and the unix
shell.


\subsubsection{Matching several patterns}

TO match a series of patterns, you first need to store these patterns
in a file. Let create a pattern file:

\begin{verbatim}
cat > test_patterns.txt
cacgtg	high
cacgtt	medium
\end{verbatim}
(then tpye Ctrl-d to close)

check the content of your pattern file.
\begin{verbatim}
more test_patterns.txt
\end{verbatim}

There are two lines, each representing a pattern. The first word of
each line contains the pattern, the second word the identifier for
that pattern. This column can be left lank, in which case the pattern
is used as identifier.

We can now use this file to search all matching psitions of both
patterns in the PHO sequences.

\begin{verbatim}
dna-pattern -i PHO_up800.fasta -format fasta \
    -2str  -N 4 -origin -0 \
    -pl test_patterns.txt
\end{verbatim}

\subsubsection{Counting pattern matches}

In the previous examples, we were interested in matching positions. It
is sometimes interesting to get a more synthetic information, in the
form of a count of matching positions for each sequences. Try:

\begin{verbatim}
dna-pattern -i PHO_up800.fasta -format fasta \ 
     -2str  -N 4 -origin -0 -c \
     -pl test_patterns.txt
\end{verbatim}

With the option \texttt{-c}, the program returns the number of
occurrences of each pattern in each sequence. The output format is
different: there is one row for each combination pattern-sequence. Te
columns indicate respectively
\begin{enumerate}
\item sequence identifier
\item pattern identifier
\item pattern sequence
\item match count
\end{enumerate}

An even more synthetic result can be obtained with the option
\texttt{-ct} (count total).

\begin{verbatim}
dna-pattern -i PHO_up800.fasta -format fasta -2str \
    -pl test_patterns.txt -N 4 -origin -0 -ct
\end{verbatim}

This time, only two rows are returned, one per pattern. 

\subsubsection{Getting a count table}

Anothe rway to display the count information is in the form of a
table, where each row represents a gene and each column a pattern.

\begin{verbatim}
dna-pattern -i PHO_up800.fasta -format fasta -2str \
    -pl test_patterns.txt -N 4 -origin -0 -table
\end{verbatim}

This representation is very convenient for aplying multivariate
statistics on the results (e.g. classificatying genes according to the
patterns found in their upstream sequences)

Last detail: we can add one column and one row for the totals per
gene and per pattern.

\begin{verbatim}
dna-pattern -i PHO_up800.fasta -format fasta -2str \
    -pl test_patterns.txt -N 4 -origin -0 -table -total
\end{verbatim}




\subsection{patser (program developed by by Jerry Hertz)}

We will now see how to match a profile matrix against a sequence
set. For this, we use \textit{patser}, a program written by Jerry
Hertz. 

\subsubsection{Getting help}

help can be obtained with the two usual options.

\begin{verbatim}
patser -h
patser -help
\end{verbatim}

\subsubsection{Matrix conversion}

Patser expects as input a matrix like the 4 matrices we obtained above
with \textit{consensus}. The output from \textit{consensus} can however
not be used directly because it contains several matrices, and a lot
of additional information. One possibility is to copy-paste the matrix
of interest to a separate file.

To avoid manual editing, RSAT contains a program
\textit{convert-matrix}, which automaticaly extacts a matrix from
various file formats, including consensus.

\begin{verbatim}
convert-matrix -format consensus -i PHO_consensus -o PHO_matrix 

more PHO_matrix
\end{verbatim}


\subsubsection{Detecting Pho4p sites in the PHO genes}

After having extracted the matrix, we can match it against the PHO
sequences to detect putative regulatory sites.

\begin{verbatim}
patser -m PHO_matrix -f PHO_up800.wc -A a:t c:g -c -ls 9
\end{verbatim}

\subsubsection{Detecting Pho4p sites in all upstream regions}

We will now match our PHO matrix against the whole set of upstream
regions from the 6200 yeast genes. This should allow us to detect new
genes potentially regulated by Pho4p.

One possibility would be to use \textit{retrieve-seq} to extract all
yeast upstream regions, and save the result in a file, which will then
be used as input by \textit{patser}. To avoid occupying too much space
on the disk, we could combine both tasks in a single command, and
immediately redirect the output of \textit{retrieve-seq} as input for
\textit{patser}. This can be done with the pipe character as below. 


\begin{verbatim}
retrieve-seq -type upstream -from -1 -to -800  \
    -org Saccharomyces_cerevisiae \
    -all -format wc -label gene  \
    | patser -m PHO_matrix -l 9 -A a:t c:g
\end{verbatim}
