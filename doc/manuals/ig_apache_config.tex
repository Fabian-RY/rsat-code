\section{Configuring and activating a local \RSAT Web server}

In order to provide web access to the Regulatory Sequence Analysis
Tools (\RSAT), you need to adapt the configuration of your web
server. This requires root privileges (can be done only by the system
administrator of the computer).


\begin{enumerate}
\item A default configuration file is provided with the \RSAT
  distribution (\file{rsat\_apache\_default.conf}). 

  Copy this template to a file named \file{rsat.conf}, which you will
  edit to replace the string \texttt{[RSAT\_PARENT\_PATH]} by the
  full path of your \file{rsat} folder.
  
\item The configuration file should then be copied to some appropriate
  place in the Apache configuration folder of your computer. This
  place depends on the operating system (Mac OSX or Linux) and on the
  distribution (Linux Ubuntu, Centos, ...).
  
  Some Usual places:
  \begin{itemize}
  \item On Centos: \file{/etc/httpd/conf.d/rsat.conf}
  \item On Ubuntu: \file{/etc/apache2/sites-enabled/rsat.conf}
  \item On Mac OSX: \file{/etc/apache2/users/rsat.conf}
  \end{itemize}
  
\item You need to restart the Web server (note: the command depends on
  your OS. Can be \program{apachectl}, \program{apache2ctl} or
  \program{httpd}.

  \begin{lstlisting}
sudo apachectl restart
  \end{lstlisting}


\item Check that all properties related to the Web site URL are
  properly defined in the \RSAT property files
  \file{\$RSAT/RSAT\_config.props} and
  \file{\$RSAT/RSAT\_config.mk}. 

  In principle you already configured these files in the beginning of
  the installation, with the command
  \begin{lstlisting}
perl perl-scripts/configure_rsat.pl
  \end{lstlisting}

  \textbf{Note:} it is important to properly define the URL fo the Web
  server (\texttt{RSAT\_WWW} and related variables). The default URL
  (http://localhost/rsat/) only works if the server and client (your
  Web browser) are on the same machine. This internal access is very
  convenient to work in places where you don't have Internet
  connections, but does not allow other computers to use your Web
  server.  If you want to enable Web queries from remote computers,
  you should specify an externally visible URL.

\end{enumerate}

