\chapter{Generating random sequences}

Th program \program{random-seq} allows to generate random sequences
wih different random models.

It supports Bernoulli models (independence between successive
residues) and Markov models of any order. Markov models are generally
more suitable to represent biological sequences. 


We will briely illustrate different ways to use this program. 

\section{Sequences with identically and independently distributed (IID) nucleotides}

\begin{small}
\begin{verbatim}
random-seq -l 200 -r 20 -o rand_L200_N20.fasta
\end{verbatim}
\end{small}

We can now check th residue cmposition of this random sequence.

\begin{small}
\begin{verbatim}
oligo-analysis -v 1 \
  -i rand_L200_N20.fasta \
  -l 1 -1str -return occ,freq \
  -o rand_L200_N20_1nt-1str.tab
\end{verbatim}
\end{small}

\section{Sequences with nucleotide-specific frequencies}

In general, the residue composition of biological sequences is
biased. We can impose residue-specific probabilities for the random
sequence generation.

\begin{small}
\begin{verbatim}
random-seq -l 200 -r 20 -a a:t 0.3 c:g 0.2 \
  -o rand_L200_N20_at30.fasta 

oligo-analysis -v 1 \
  -i rand_L200_N20_at30.fasta \
  -l 1 -1str -return occ,freq \
  -o rand_L200_N20_at30_1nt-1str.tab
\end{verbatim}
\end{small}


\section{Markov chain-based random sequences}

The random generator \program{random-seq} supports Markov chains of
any order (as far as the corresponding ferquency table has previously
been calculated). The Markov model is specified by indicating an
oligonucleotide frequency table. The table of oligonucleotides of
length $k$ is automatically converted in a transition table of order
$m=k-1$ duing the execution of \program{random-seq}. 


\begin{small}
\begin{verbatim}
random-seq -l 200 -r 20 \
  -expfreq $RSAT/data/genomes/Escherichia_coli_K12/oligo-frequencies/3nt_upstream-noorf_Escherichia_coli_K12-1str.freq.gz \
  -o rand_L200_N20_mkv2.fasta 
\end{verbatim}
\end{small}

A simpler way to obtain organism-specific Markov models is to use the
options \option{-bg} and \option{-org} of \program{random-seq}. 

\begin{small}
\begin{verbatim}
## This command generates random sequences with a Markov model of order 2,
## calibrated on all the non-coding upstream sequences of E.coli.
random-seq -l 200 -r 20 \
  -org Escherichia_coli_K12 -bg upstream-noorf -ol 3 \
  -o rand_L200_N20_mkv2.fasta 
\end{verbatim}
\end{small}



