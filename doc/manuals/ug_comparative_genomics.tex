%%%%%%%%%%%%%%%%%%%%%%%%%%%%%%%%%%%%%%%%%%%%%%%%%%%%%%%%%%%%%%%%
%%%% COMPARATIVE GENOMICS
%%%%%%%%%%%%%%%%%%%%%%%%%%%%%%%%%%%%%%%%%%%%%%%%%%%%%%%%%%%%%%%%
\chapter{Comparative genomics}

\section{Genome-wise comparison of protein sequences}
\label{genome_blast}

In this section, we explain how to use the program
\program{genome-blast}, which runs the sequence similarity search
program \program{BLAST} to detect significant similarities between all
the proteins of a set of genomes. 

This operation can take time, and the result tables occupy a
considerable amount of space on the hard disk. For this reason, the
\RSAT distribution does thus not include the complete comparison of
all genomes against all other ones, but is restricted to some model
genomes (\org{Escherchia coli K12} versus all bacteria,
\org{Saccharomyces cerevisiae} against all Fungi, ...). 

Depending on your organism of interest, you might wish to perform
additional comparisons for your own purpose. In this section, we
explain how to compute the similiraty tables between a query organism
(e.g. \org{Mycoplasma pneumoniae}) and a reference taxon (e.g. all
Bacteria).

In order to install the tables of similarities between gene products
in \RSAT, you need writing permissions in the directory
$\$RSAT/data$. If this is not the case, ask your system administrator
to do it for you.


\subsection{Applying genome-blast between two genomes}

As a first test, we will use \program{genome-blast} to compare all the
gene products (proteins) of a query organism (e.g. \org{Mycoplasma
  pneumoniae}) against all the gene products of a reference organism
(e.g. \org{Bacillus subtilis}).

This protocol assumes that the two organisms are already installed on
your \RSAT site, as explained in the installation guide.

We will perform in two steps:

\begin{enumerate}
\item Use the program \program{formatdb} (which is part of the
  \program{BLAST} distribution) to create a BLAST-formatted structure
  (the ``database'') with all proteins of the reference organism
  (\org{Bacillus subtilis}).
\item Use the program \program{blastall} (part of the \program{BLAST}
  distribution) to detect similarities between each protein of the
  query organism (\org{Mycoplasma pneumoniae}) and the reference
  organism.
\end{enumerate}

\subsubsection{Formatting the BLAST DB}

This DB formatting step is very efficient, it should be completed in a
few seconds.

\begin{footnotesize}
\begin{verbatim}
genome-blast -v 1 -task formatdb \
   -q Mycoplasma_pneumoniae \
    -db Bacillus_subtilis 
\end{verbatim}
\end{footnotesize}

The result is found in the data directory containing
\org{Bacillus subtilis}. A new directory \file{blastdb} has been
created, which contains the BLAST-formatted database with all the
proteins of the reference organism.

\begin{footnotesize}
\begin{verbatim}
ls -ltr $RSAT/data/genomes/Bacillus_subtilis/blastdb
\end{verbatim}
\end{footnotesize}

These are binary files, that you should in principle not open as such.

\subsubsection{Searching similarities}

The program \program{blastall} compares all the sequences of an input
set against all the sequences of a database (the one we just created
above). The program \program{genome-blast} generates the appropriate
\program{blastall} command to find the BLAST database directory, and
query it with the proteins of the query organism.

\begin{footnotesize}
\begin{verbatim}
genome-blast -v 1 -task blastall \
   -q Mycoplasma_pneumoniae \
    -db Bacillus_subtilis 
\end{verbatim}
\end{footnotesize}

This task takes a bit less that one minute for Pneumoniae (because we
chose a very small genomes), and can take around 10 minutes for
medium-sized bacterial genomes (~4,000 genes).

Note that the \program{blastall} command is written in the verbosity
message. If you have specific reasons to customize this command, you
can adapt it to apply different parameters.

\subsubsection{Searching reciprocal similarities}

One classical orthology criterion (which is not perfect but has
practical advantages) is to select the bidirectional best hits as
candidate orthologs.

For this, we need to run the reciprocal blast, i.e. using
\org{Bacillus subtilis} as query organism, and \org{Mycoplasma
  genitalium} as reference organism.

Note that you can run the two BLAST commands (\program{formatdb} and
\program{blastall}) in a single shot, by specifying multiple tasks for
\program{genome-blast}.

\begin{footnotesize}
\begin{verbatim}
genome-blast -v 1 -task formatdb,blastall \
    -q Bacillus_subtilis \
    -db Mycoplasma_pneumoniae
\end{verbatim}
\end{footnotesize}

We can now perform a quick test: select the bidirectional best hit
(\option{-rank 1}) for the gene \gene{NP\_109706.1}.

\begin{footnotesize}
\begin{verbatim}
get-orthologs -q NP_109706.1 -uth rank 1 -return all \
    -org Mycoplasma_pneumoniae -taxon Bacillus_subtilis
\end{verbatim}
\end{footnotesize}

\subsection{Applying genome-blast between a genome and a  taxon}

Generally, we want to compare a query organism to all the organisms of
a given taxon (the \concept{reference taxon}). This can be done with
the option \option{-dbtaxon}.

As an example, we will BLAST all the proteins of \org{Mycoplasma
  pneumoniae} against all the proteins of each species of
\org{Mollicutes}.


\begin{footnotesize}
\begin{verbatim}
genome-blast -v 1 -task formatdb,blastall \
    -q Mycoplasma_pneumoniae \
    -dbtaxon Mollicutes
\end{verbatim}
\end{footnotesize}

And now the reciprocal search: BLAST all gene products of each
bacteria of the taxon \org{Mollicutes} against those of
\org{Mycoplasma pneumoniae}.

\begin{footnotesize}
\begin{verbatim}
genome-blast -v 1 -task formatdb,blastall \
    -db Mycoplasma_pneumoniae \
    -qtaxon Mollicutes
\end{verbatim}
\end{footnotesize}

We can now retrieve the orthologs of the gene
\gene{Mycoplasma_pneumoniae} in all Mollicutes.

\begin{footnotesize}
\begin{verbatim}
get-orthologs -q NP_109706.1 -uth rank 1 -return all \
    -org Mycoplasma_pneumoniae -taxon Mollicutes
\end{verbatim}
\end{footnotesize}


\section{Getting putative homologs, orthologs and paralogs}

In this section, I will explain how to use the program
\program{get-orthologs}. This program takes as input one or several
query genes belonging to a given organism (the \concept{reference
  organism}), and return the genes whose product (peptidic sequence)
show significant similarities with the products of the query
genes. The primary usage of \program{get-orthologs} is thus to return
lists of similar genes, not specialy orthologs. Additional criteria
can be imposed to infer orthology.  In particular, one of the most
common criterion is to select \concept{bidirectional best hits
  (BBH)}. This can be achieved by imposing the rank 1 with the option
\option{-uth rank 1}.

We will illustrate the concept by retrieving the genes whose product
is similar to the protein LexA of \org{Escherichia coli K12}, in
all the Gammaproteobacteria. We will then refine the query to extract
putative orthologs.

\subsection{Getting genes by similarities}

\begin{footnotesize}
\begin{verbatim}
get-orthologs -v 1 -org Escherichia_coli_K12 \
  -taxon Gammaproteobacteria \
  -q lexA -o lexA_orthologs_Gammaproteobacteria.tab
\end{verbatim}
\end{footnotesize}

The result file is a list of all the Gammaproteobacterial genes whose
product shows some similarity with the LexA protein from E.coli K12.  

\begin{scriptsize}
\begin{verbatim}
...
#ref_id ref_org query
Sde_1787        Saccharophagus_degradans_2-40   b4043
CPS_0237        Colwellia_psychrerythraea_34H   b4043
CPS_2683        Colwellia_psychrerythraea_34H   b4043
CPS_1635        Colwellia_psychrerythraea_34H   b4043
IL0262  Idiomarina_loihiensis_L2TR      b4043
...
c5014   Escherichia_coli_CFT073 b4043
c3190   Escherichia_coli_CFT073 b4043
b4043   Escherichia_coli_K12    b4043
...
\end{verbatim}
\end{scriptsize}

Each similarity is reported by the ID of the gene, the organism to
which is belong, and the ID of the query gene. In this case, the third
column contains the same ID on all lines: b4043, which is the ID of
the gene lexA in \org{Escherichia coli K12}. It seems thus poorly
informative, but this column becomes useful when several queries are
submitted simultaneously.

\subsection{Obtaining information on the BLAST hits}

The program \program{get-orthologs} allows to return additional
information on the hits. The list of supported return fields is
obtained by calling the command with the option \option{-help}. For
example, we can ask to return the percentage of identity, the
alignment length, the E-value and the rank of each hit.

\begin{footnotesize}
\begin{verbatim}
get-orthologs -v 1 -org Escherichia_coli_K12 \
  -taxon Gammaproteobacteria \
  -q lexA -o lexA_orthologs_Gammaproteobacteria.tab \
  -return ident,ali_len,e_value,rank 
\end{verbatim}
\end{footnotesize}

Which gives the following result:

\begin{scriptsize}
\begin{verbatim}
...
#ref_id ref_org query   ident   ali_len e_value rank
Sde_1787        Saccharophagus_degradans_2-40   b4043   65.33   199     1e-68   1
CPS_0237        Colwellia_psychrerythraea_34H   b4043   65.69   204     6e-75   1
CPS_2683        Colwellia_psychrerythraea_34H   b4043   33.94   109     1e-10   2
CPS_1635        Colwellia_psychrerythraea_34H   b4043   34.12   85      1e-06   3
IL0262  Idiomarina_loihiensis_L2TR      b4043   66.83   202     1e-75   1
...
c5014   Escherichia_coli_CFT073 b4043   100.00  202     2e-111  1
c3190   Escherichia_coli_CFT073 b4043   43.33   90      2e-14   2
b4043   Escherichia_coli_K12    b4043   100.00  202     2e-111  1
...
\end{verbatim}
\end{scriptsize}

Not surprisingly, the answer includes the self-match of lexA (ID
b4043) in \org{Escherichia coli K12}, with 100\% of identify.

\subsection{Selecting bidirectional best hits}

We can see that the output contains several matches per genome. For
instance, there are 3 matches in \org{Colwellia psychrerythraea
  34H}. If we assume that these similarities reflect homologies, the
result contains thus a combination of paralogs and orthologs. 

The simplest criterion to select ortholog is that of
\concept{bidirectional best hit (BBH)}. We can select BBH by imposing
an upper threshold on the rank, with the option \option{-uth}.

\begin{footnotesize}
\begin{verbatim}
get-orthologs -v 1 -org Escherichia_coli_K12 \
  -taxon Gammaproteobacteria \
  -q lexA -o lexA_orthologs_Gammaproteobacteria_bbh.tab \
  -return ident,ali_len,e_value,rank \
  -uth rank 1 
\end{verbatim}
\end{footnotesize}

The result has now been reduced to admit at most one hit per genome.

\begin{scriptsize}
\begin{verbatim}
...
#ref_id ref_org query   ident   ali_len e_value rank
Sde_1787        Saccharophagus_degradans_2-40   b4043   65.33   199     1e-68   1
CPS_0237        Colwellia_psychrerythraea_34H   b4043   65.69   204     6e-75   1
IL0262  Idiomarina_loihiensis_L2TR      b4043   66.83   202     1e-75   1
...
c5014   Escherichia_coli_CFT073 b4043   100.00  202     2e-111  1
b4043   Escherichia_coli_K12    b4043   100.00  202     2e-111  1
...
\end{verbatim}
\end{scriptsize}

\subsection{Selecting hits with more stringent criteria}

It is well known that the sole criterion of BBH is not sufficient to
infer orthology between two genes. In particular, there is a risk to
obtain irrelevant matches, due to partial matches between a protein
and some spurious domains. To avoid this, we can add a constraint on
the percentage of identity (min 30\%), and on the alignment length
(min 50 aa). These limits are somewhat arbitrary, we use them to
illustrate the principe, and leave to each user the responsibility to
choose the criteria that she/he considers as relevant. Finally, we
will use a more stringent threshold on E-value than the default one,
by imposing an upper threshold of 1e-10.

\begin{footnotesize}
\begin{verbatim}
## Note that or this test we suppress the BBH constraint (-uth rank 1)
get-orthologs -v 1 -org Escherichia_coli_K12 \
  -taxon Gammaproteobacteria \
  -q lexA -o lexA_orthologs_Gammaproteobacteria_id30_len50_eval-10.tab \
  -return ident,ali_len,e_value,rank \
  -lth ident 30 -lth ali_len 50 -uth e_value 1e-10
\end{verbatim}
\end{footnotesize}

We can now combine the constrains above with the criterion of BBH.

\begin{footnotesize}
\begin{verbatim}
## Note that or this test we include the BBH constraint (-uth rank 1)
get-orthologs -v 1 -org Escherichia_coli_K12 \
  -taxon Gammaproteobacteria \
  -q lexA -o lexA_orthologs_Gammaproteobacteria_bbh_id30_len50_eval-10.tab \
  -return ident,ali_len,e_value,rank \
  -lth ident 30 -lth ali_len 50  -uth e_value 1e-10 \
  -uth rank 1 
\end{verbatim}
\end{footnotesize}

As expected, the number of selected hits is reduced by adding these
constraints. In Sept 2006, we obtained the following number of hits
for lexA in Gammaproteobacteria.

\begin{itemize}
\item 122 hits without any constraint;
\item 107 hits with contrains on ident,ali\_len and e\_value;
\item 69 hits with the constraint of BBH;
\item 69 hits with the combined constraint of BBH, at least 30\%
  identity and an alignment over more than 50 aminoacids, and an
  E-value <= 1.e-10.
\end{itemize}

Actually, in the particular case of \gene{lexA}, the BBH constraint
already filtered out the spurious matches, but inother cases they can
be useful.

\section{Retrieving sequences for multiple organisms}

The program \program{retrieve-seq-multigenome} can be used to retrieve
sequences for a group of genes belonging to different organisms.This
program takes as input a file with two columns. Each row of this file
specifies one query gene.

\begin{enumerate}

\item The first column contains the name or identifier of the gene
  (exactly as for the single-genome program \program{retrieve-seq}).

\item The second column indicates the organism to which the gne belongs.

\end{enumerate}

The output of \program{get-orthologs} can thus directly be used as
input for \program{retrieve-seq-multigenome}.

\begin{footnotesize}
\begin{verbatim}
retrieve-seq-multigenome -noorf \
  -i lexA_orthologs_Gammaproteobacteria_bbh_id30_len50_eval-10.tab \
  -o lexA_orthologs_Gammaproteobacteria_up-noorf.fasta
\end{footnotesize}
\end{verbatim}
\end{footnotesize}

\section{Detection of phylogenetic footprints}

\textbf{TO BE WRITTEN}

\begin{footnotesize}
\begin{verbatim}
dyad-analysis  -v 1 \
  -i lexA_orthologs_Gammaproteobacteria_up-noorf.fasta \
  -sort -2str -noov -lth occ 1 -lth occ_sig 0 \
  -return occ,freq,proba,rank \
  -l 3 -spacing 0-20 -bg monads \
  -o lexA_orthologs_Gammaproteobacteria_up-noorf_dyads-2str-noov.tab
\end{verbatim}
\end{footnotesize}

\section{Phylogenetic profiles}

The notion of \concept{phylogenetic profile} was introduced by
Pellegrini et al. (1999). They identified putative orthologs for all
the genes of \org{Escherichia coli K12} in all the complete genomes
available at that time, and built a table with one row per gene, one
column per genome. Each cell of this table indicates if an ortholog of
the considered gene (row) has been identified in the considered genome
(column). Pellegrini et al. (1999) showed that genes having similar
phylogenetic profiles are generally involved in common biological
processes. The analysis of phylogenetic profiles is thus a powerful
way to identify functional grouping in completely sequenced genomes.

The program \program{get-orthologs} can be used to obtain the
phylogenetic profiles. The principle is to submit the complete list of
protein-coding genes of the query organism. We process in two steps : 

\begin{enumerate}

\item With \program{get-orthologs}, we can identify the putative
  ortholgos for all the genes of the query organism, using the
  criterion of \concept{bidirectional best hit (BBH)}. This generate a
  large table with one row per pair of putative orthologs.

\item We then use \program{convert-classes} to convert the ortholog
  table into profiles (one row per gene, one column per genome).

\end{enumerate}

We will illustrate this by calculating the phylogenetic profiles of
all the genes from \org{Saccharomyces cerevisiae} across all the
Fungi. We use a level of verbosity of 2, in order to get information
about the progress of the calculations.

\begin{footnotesize}
\begin{verbatim}

## Identify all the putative orthologs (BBH)
get-orthologs -v 2 \
  -i $RSAT/data/genomes/Saccharomyces_cerevisiae/genome/cds.tab  \
  -org Saccharomyces_cerevisiae \
  -taxon Fungi \
  -uth rank 1 -lth ali_len 50 -lth ident 30 -uth e_value 1e-10 \
  -return e_value,bit_sc,ident,ali_len \
  -o Saccharomyces_cerevisiae_vs_Fungi_bbh.tab

## Convert ortholog table into a profile table
## with the IDs of the putative orthologs
convert-classes -v 2 \
  -i Saccharomyces_cerevisiae_vs_Fungi_bbh.tab  \
  -from tab -to profiles \
  -ccol 2 -mcol 3 -scol 1 -null "<NA>" \
  -o Saccharomyces_cerevisiae_vs_Fungi_phyloprofiles_ids.tab

\end{verbatim}
\end{footnotesize}


The resulting table indicates the identifier of the ortholog
genes. The option \option{-null} was used to specify that the string
\texttt{<NA>} should be used to indicate the absence of putative
orhtolog.

Another option would be to obtain a ``quantitative'' profile, where
each cell indicates the E-value of the match between the two
orthologs. This can be done by specifying a different score column
with the option \option{-scol} of \program{convert-classes}.

\begin{footnotesize}
\begin{verbatim}
## Convert ortholog table into a profile table
## with the E-value of the putative orthologs
convert-classes -v 2 \
  -i Saccharomyces_cerevisiae_vs_Fungi_bbh.tab  \
  -from tab -to profiles \
  -ccol 2 -mcol 3 -scol 4 -null "<NA>" \
  -o Saccharomyces_cerevisiae_vs_Fungi_phyloprofiles_evalue.tab
\end{verbatim}
\end{footnotesize}

\section{Detecting pairs of genes with similar phylogenetic profiles}

In the previous section, we generated tables indicating the
phylogenetic profiles of each gene from \org{Saccharomyces
  cerevisiae}. This table contains one row per gene, and one column
per fungal genome. 

We will now use the program \program{compare-profiles} to compare each
gene profile to each other, to select the pairs of genes with
significantly similar profiles. The problem is of course to choose our
criterion of similarity between two gene profiles.

\subsection{Comparing binary profiles with \program{compare-profiles}}

For the binary profiles, the most relevant statistics is the
\concept{hypergeometric significance}.

\begin{footnotesize}
\begin{verbatim}
## Compare the binary phylogenetic profiles 
## using the hypergeometric significance
compare-profiles -v 2 \
  -i Saccharomyces_cerevisiae_vs_Fungi_phyloprofiles_evalue.tab \
  -lth AB 1 -lth sig 0 \
  -return counts,jaccard,hyper,entropy \
  -o Saccharomyces_cerevisiae_vs_Fungi_phyloprof_gene_pairs.tab
\end{verbatim}
\end{footnotesize}

In the previous commands, we set the verbosity to 2, in order to keep
track the progress of the task. Actually, the processing can take a
few minuts, it is probably the good moment for a coffee break.

\subsection{Comparing binary profiles with \program{compare-classes}}

Another way to compare the phylogenetic profiles is to directly
analyze with \program{compare-classes} the table of orthology
(previously obtained from \program{get-orthologs}).

This is just another way of considering the same problem: in order to
compare genes $A$ and $B$, we will consider as a first class ($Q$) the
set of genomes in which gene $A$ is present, and as a second class
($R$) the set of genomes in which gene $B$ is present. We will then
calculate the intersection between these two classes, and assess the
significance of this intersection, given the total number of
genomes.

Thus, \program{compare-classes} will calculate the hypergeometric
statistics, exactly in the same way as \program{compare-profiles}. 

\begin{footnotesize}
\begin{verbatim}
## Convert the orthology into "classes", where each class (second column)
## corresponds to a gene from Saccharomyces cerevisiae, and indicates 
## the set of genomes (first column) in which this gene is present. 
convert-classes -from tab -to tab -mcol 2 -ccol 3 -scol 5 \
  -i Saccharomyces_cerevisiae_vs_Fungi_bbh.tab \
  -o Saccharomyces_cerevisiae_vs_Fungi_bbh_classes.tab

## Compare the classes to detect significant overlaps
compare-classes -v 3 \
  -i Saccharomyces_cerevisiae_vs_Fungi_bbh_classes.tab \
  -lth QR 1 -lth sig 0 -sort sig -sc 3 \
  -return occ,proba,dotprod,jac_sim,rank \
  -o phyloprof_gene_pairs.tab
\end{verbatim}
\end{footnotesize}
