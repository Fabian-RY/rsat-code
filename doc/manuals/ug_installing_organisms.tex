%%%%%%%%%%%%%%%%%%%%%%%%%%%%%%%%%%%%%%%%%%%%%%%%%%%%%%%%%%%%%%%%
%%%% Parsing and installing new organisms
%%%%%%%%%%%%%%%%%%%%%%%%%%%%%%%%%%%%%%%%%%%%%%%%%%%%%%%%%%%%%%%%


\chapter{Installing additional genomes on your machine}

The easiest way to install genomes on your machine is to download them
from the main \RSAT server, as indicated in the Chapter ``Downloading
genomes'' (Chap.~\ref{downloading_genomes} of the installation guide).

In some cases, you may however wish to install a genome by yourself,
because this genome is not supported on the main \RSAT server. For
this, you can use the programs that we use to install new genomes on
the main \RSAT server.

\section{Installing genomes from  NCBI/Genbank files}

In the section \ref{downloading_genomes}, we saw that the genomes
installed on the main \RSAT server can easily be installed on your
local site. In some cases, you would like to install additional
genomes, which are not published yet, or which are not supported on
the main \RSAT server.

If your genomes are available in Genbank (files .gbk) or EMBL (files
.embl) format, this can be done without too much effort, using the
installation tools of \RSAT. 

The parsing of genomes from their original data sources is however
more tricky than the synchronization from the \RSAT server, so this
procedure should be used only if you need to install a genome that is
not yet supported. 

If this is not your case, you can skip the rest of this section.

\subsection{Organization of the genome files}

In order for a genome to be supported, \RSAT needs to find at least
the following files.

\begin{enumerate}
\item organism description
\item genome sequences
\item feature tables (CDS, mRNA, \ldots)
\item lists of names/synonyms
\end{enumerate}

From these files, a set of additional installation steps will be done
by \RSAT programs in order to compute the frequencies of
oligonucleotides and dyads in upstream sequences.

If you installed \RSAT as specified above, you can have a look at the
organization of a supported genome, for example the yeast
\org{Saccharomyces cerevisiae}.

\begin{footnotesize}
\begin{verbatim}
cd ${RSAT}/public_html/data/genomes/Saccharomyces_cerevisiae/genome
ls -l
\end{verbatim}
\end{footnotesize}

As you see, the folder \file{genome} contains the sequence files and
the tables describing the organism and its features (CDSs, mRNAs,
\ldots). The \RSAT parser exports tables for all the feature types
found in the original genbank file. There are thus a lot of distinct
files, but you should not worry too much, for the two following
reasons:
\begin{enumerate}
\item \RSAT only requires a subset of these files (basically, those
  describing organisms, CDSs, mRNAs, rRNAs and tRNAs).
\item All these files can be generated automatically by \RSAT parsers.
\end{enumerate}

\subsubsection{Organism description} 

The description of the organism is given in two separate files.


\begin{footnotesize}
\begin{verbatim}
cd ${RSAT}/public_html/data/genomes/Saccharomyces_cerevisiae/genome
ls -l organism*.tab

more organism.tab

more organism_names.tab
\end{verbatim}
\end{footnotesize}

\begin{enumerate}
\item \file{organism.tab} specifies the ID of the organism and its
  taxonomy. The ID of an organism is the TAXID defined by the NCBI
  taxonomical database, and its taxonomy is usually parsed from the
  .gbk files (but yo may need to specify it yourself in case it would
  be missing in your own data files).

\item \file{organism\_name.tab} indicates the name of the organism.
\end{enumerate}


\subsubsection{Genome sequence} 

A genome sequence is composed of one or more contigs. A contig is a
contigous sequence, resulting from the assembly of short sequence
fragments obtained during the sequencing. When a genome is completely
sequenced and assembled, each chomosome comes as a single contig. 

In \RSAT, the genome sequence is specified as one separate file per
contig (chromosome) sequence. Each sequence file must be in raw format
(i.e. a text file containing the sequence without any space or
carriage return). 

In addition, the genome directory contains one file indicating the
list of the contig (chromosome) files.

\begin{footnotesize}
\begin{verbatim}
cd $RSAT/data/genomes/Saccharomyces_cerevisiae/genome/

## The list of sequence files
cat contigs.txt

## The sequence files
ls -l *.raw

\end{verbatim}
\end{footnotesize}

\subsubsection{Feature table}

The \file{genome} directory also contains a set of feature tables
giving the basic information about gene locations. Several feature
types (CDS, mRNA, tRNA, rRNA) can be specified in separate files
(\file{cds.tab}, \file{mrna.tab}, \file{trna.tab}, \file{rrna.tab}).

Each feature table is a tab-delimited text file, with one row per
feature (cds, mrna, \ldots) and one column per parameter. The
following information is expected to be found.

\begin{enumerate}

\item Identifier

\item Feature type (e.g. ORF, tRNA, ...)

\item Name

\item Chromosome. This must correspond to one of the sequence
identifiers from the fasta file.

\item Left limit

\item Right limit

\item Strand (D for direct, R for reverse complemet)

\item Description. A one-sentence description of the gene function.

\end{enumerate}


\begin{footnotesize}
\begin{verbatim}
## The feature table
head -30 cds.tab
\end{verbatim}
\end{footnotesize}


\subsubsection{Feature names/synonyms}

Some genes can have several names (synonyms), which are specified in
separate tables.

\begin{enumerate}
\item ID. This must be one identifier found in the feature table
\item Synonym
\item Name priority (primary or alternate)
\end{enumerate}


\begin{footnotesize}
\begin{verbatim}
## View the first row of the file specifying gene names/synonyms
head -30 cds_names.tab
\end{verbatim}
\end{footnotesize}


Multiple synonyms can be given for a gene, by adding several lines with
the same ID in the first column.

\begin{footnotesize}
\begin{verbatim}
## An example of yeast genes with multiple names
grep YFL021W cds_names.tab 
\end{verbatim}
\end{footnotesize}



\subsection{Downloading genomes from NCBI/Genbank}

The normal way to install an organism in \RSAT is to download the
complete genome files from the NCBI
\urlref{ftp://ftp.ncbi.nih.gov/genomes/}, and to parse it with the
program \program{parse-genbank.pl}.

However, rather than downloading genomes directly from the NCBI site,
we will obtain them from a mirror
\urlref{bio-mirror.net/biomirror/ncbigenomes/} which presents two
advantages?

\begin{itemize}
\item Genome files are compressed (gzipped), which strongly reduces
  the transfer and storage volume. 
\item This mirror can be queried by \program{rsync}, which facilitates
  the updates (with the appropriate options, \program{rsync} will only
  download the files which are newer on the server than on your
  computer).
\end{itemize}

\RSAT includes a makefile to download genomes from different sources.
We provide hereafter a protocol to create a download directory in your
account, and download genomes in this directory. Beware, genomes
require a lot of disk space, especially for those of higher
organisms. To avoid filling up your hard drive, we illustrate the protocol
with the smallest procaryote genome to date: \textit{Mycoplasma
  genitamlium}.

\begin{footnotesize}
\begin{verbatim}
## Creating a directory for downloading genomes
cd $RSAT
mkdir -p downloads
cd downloads

## Creating a link to the makefile which allows you to dowload genomes
ln -s $RSAT/makefiles/downloads.mk ./makefile
\end{verbatim}
\end{footnotesize}

We will now download a small genome from NCBI/Genbank. 

\begin{footnotesize}
\begin{verbatim}
## Downloading one directory from NCBI Genbank
cd $RSAT/downloads/
make one_genbank_dir NCBI_DIR=Bacteria/Mycoplasma_genitalium_G37_uid57707
\end{verbatim}
\end{footnotesize}

We can now check the list of files that have been downloaded.

\begin{footnotesize}
\begin{verbatim}
## Downloading one directory from NCBI Genbank
cd $RSAT/downloads/
ls -l ftp.ncbi.nih.gov/genomes/Bacteria/Mycoplasma_genitalium_G37_uid57707
\end{verbatim}
\end{footnotesize}

\RSAT parsers only use the files with extension \file{.gbk.gz}.

You can also adapt the commande to download (for example) all the
Fungal genomes in a single run.

\begin{footnotesize}
\begin{verbatim}
## Downloading one directory from NCBI Genbank
cd $RSAT/downloads/
make one_ncbi_dir NCBI_DIR=Fungi
\end{verbatim}
\end{footnotesize}

You can do the same for Bacteria, of for the whole NCBI genome
repository, but this requires sveral Gb of free disck space.

\subsection{Parsing a genome from NCBI/Genbank}

The program \program{parse-genbank.pl} extract genome information
(sequence, gene location, ...) from Genbank flat files, and exports
the result in a set of tab-delimited files.

\begin{footnotesize}
\begin{verbatim}
parse-genbank.pl -v 1 \
    -i $RSAT/downloads/ftp.ncbi.nih.gov/genomes/Bacteria/Mycoplasma_genitalium 
\end{verbatim}
\end{footnotesize}

\subsection{Parsing a genome from the Broad institute (MIT)}

The website \url{http://www.broad.mit.edu/} offers a large collection of
genomes that are not available on the NCBI website. We wrote a specific parser 
for the Broad files.

To this, download the following files for the organism of interest : the supercontig file, the protein sequences and the annotation file in the GTF format.

These files contain sometimes too much information that shoud be removed. 

This is an example of the beginning of the fasta file containing the protein traduction. In this file, we should remove 
everything that follows the protein name.
\begin{footnotesize}
\begin{verbatim}
>LELG_00001 | Lodderomyces elongisporus hypothetical protein (translation) (1085 aa)
MKYDTAAQLSLINPQTLKGLPIKPFPLSQPVFVQGVNNDTKAITQGVFLDVTVHFISLPA
ILYLHEQIPVGQVLLGLPFQDAHKLSIGFTDDGDKRELRFRANGNIHKFPIRYDGDSNYH
IDSFPTVQVSQTVVIPPLSEMLRPAFTGSRASEDDIRYFVDQCAEVSDVFYIKGGDPGRL
\end{verbatim}
\end{footnotesize}
This is an example of the beginning of the fasta file containing the contigs. In this file, we should remove 
everything that follows the name of the contig. 
\begin{footnotesize}
\begin{verbatim}
>supercontig_1.1 of Lodderomyces elongisporus
AAGAGCATCGGGCAAATGATGTTTTTCAGTCCATCAATGTGATGGATCTGATAGTTGAAG
GTCCTGATGAAGTTCAACCATTTGTAAACCCGATTTACAAAGTGTGAATTATCGAGTGGT
TTATTCATCACAAGGACAAGAGCTTTGTTGGTTGACAGAGATGTTTTGCAGAAAGCCCTT
AAGGATGGTATTGCCTTGTTCAAGAAGAAACCAGTTGTTACTGAAGTAAATCTGACGACC
\end{verbatim}
\end{footnotesize}


This is an example of the beginning of the GTF file containing the contigs annotation. We should rename the contig
name so that it corresponds to the fasta file of contig. To this, we will remove the text in the name of the contig (only keep the supercontig number)
and add a prefix.
\begin{footnotesize}
\begin{verbatim}
supercont1.1%20of%20Lodderomyces%20elongisporus	LE1_FINAL_GENECALL	start_codon	322	324	.	+	0	gene_id "LELG_00001"; transcript_id "LELT_00001";
supercont1.1%20of%20Lodderomyces%20elongisporus	LE1_FINAL_GENECALL	stop_codon	3574	3576	.	+	0	gene_id "LELG_00001"; transcript_id "LELT_00001";
supercont1.1%20of%20Lodderomyces%20elongisporus	LE1_FINAL_GENECALL	exon	322	3576	.	+	.	gene_id "LELG_00001"; transcript_id "LELT_00001";
supercont1.1%20of%20Lodderomyces%20elongisporus	LE1_FINAL_GENECALL	CDS	322	3573	.	+	0	gene_id "LELG_00001"; transcript_id "LELT_00001";
\end{verbatim}
\end{footnotesize}
We use the parse \program{parse-broad-mit}.
\begin{footnotesize}
\begin{verbatim}
parse-broad-mit.pl -taxid 36914 -org Lodderomyces_elongisporus \
		   -nuc_seq lodderomyces_elongisporus_1_supercontigs.fasta \
		   -gtf lodderomyces_elongisporus_1_transcripts.gtf \
		   -gtf_remove 'supercont' \
		   -gtf_remove '%20of%20Lodderomyces%20elongisporus' \
		   -contig_prefix LELG_ -nuc_remove supercontig_ \
		   -nuc_remove ' of Lodderomyces elongisporus' \
		   -aa lodderomyces_elongisporus_1_proteins.fasta -aa_remove ' .*'
\end{verbatim}
\end{footnotesize}

This will create the raw files, the feature files and the protein sequence file.

\subsection{Updating the configuration file}

After having parsed the genome, you need to perform one additional
operation in order for \RSAT to be aware of the new organism: update
the configuration file.

\begin{footnotesize}
\begin{verbatim}
install-organism -v 1 -org Mycoplasma_genitalium -task config

## Check the last lines of the configuration file
tail -15 $RSAT/data/supported_organisms.pl
\end{verbatim}
\end{footnotesize}

From now on, the genome is considered as supported on your local \RSAT
site. You can check this with the command \program{supported-organisms}. 

\subsection{Checking the start and stop codon composition}

Once the organism is found in your configuration, you need to check
whether sequences are retrieved properly. A good test for this is to
retrieve all the start codons, and check whether they are made of the
expected codons (mainly ATG, plus some alternative start codons like
GTG or TTG for bacteria).

The script \program{install-organism} allows you to perform some
additional steps for checking the conformity of the newly installed
genome. For example, we will compute the frequencies of all the start
and stop codons, i order to check that gene locations were corectly
parsed.

\begin{footnotesize}
\begin{verbatim}
install-organism -v 1 -org Mycoplasma_genitalium -task start_stop

ls -l $RSAT/data/genomes/Mycoplasma_genitalium/genome/*start*

ls -l $RSAT/data/genomes/Mycoplasma_genitalium/genome/*stop*
\end{verbatim}
\end{footnotesize}


The stop codons should be TAA, TAG or TGA, for any organism. For
eucaryotes, all start codons should be ATG. For some procaryotes,
alternative start codons (GTG, TGG) are found with some
genome-specific frequency.

\begin{footnotesize}
\begin{verbatim}
cd $RSAT/data/genomes/Mycoplasma_genitalium/genome/

## A file containing all the start codons
more Mycoplasma_genitalium_start_codons.wc

## A file with trinucleotide frequencies in all start codons
more Mycoplasma_genitalium_start_codon_frequencies

## A file containing all the stop codons
more Mycoplasma_genitalium_stop_codons.wc

## A file with trinucleotide frequencies in all stop codons
more Mycoplasma_genitalium_stop_codon_frequencies
\end{verbatim}
\end{footnotesize}

\subsection{Calibrating oligonucleotide and dyad frequencies with \program{install-organisms}}

The programs \program{oligo-analysis} and \program{dyad-analysis}
require calibrated frequencies for the background models. These
frequencies are calculated automatically with
\program{install-organism}.

\begin{footnotesize}
\begin{verbatim}
install-organism -v 1 -org Debaryomyces_hansenii \
    -task allup,oligos,dyads,upstream_freq,protein_freq
\end{verbatim}
\end{footnotesize}

\textbf{Warning: } this task may require several hours of computation,
depending on the genome size. For the \RSAT server, we use a PC
cluster to regularly install and update genomes. As the task
\textit{allup}, is a prerequisite for the computation of all
oligonucleotide and dyad frequencies, it should be run directly on the
main server before computing oligo and dyad frequencies on the nodes
of the cluster. We will thus proceed in two steps. Note that this
requires a PC cluster and a proper configuration of the batch
management program.

\begin{footnotesize}
\begin{verbatim}
## Retrieve all upstream sequences
## Executed directly on the server
install-organism -v 1 -org Debaryomyces_hansenii \
    -task allup

## Launch a batch queue for computing all oligo and dyad frequencies
## Executed on the nodes of a cluster
install-organism -v 1 -org Debaryomyces_hansenii \
    -task oligos,dyads,upstream_freq,protein_freq -batch
\end{verbatim}
\end{footnotesize}

\subsection{Installing a genome in your own account}

In some cases, you might want to install a genome in your own account
rather than in the \RSAT folder, in order to be able to analyze this
genome before putting it in public access.


In this chapter, we explain how to add support for an organism on your
local configuration of \RSAT. This assumes that you have the complete
sequence of a genome, and a table describing the predicted location of
genes.

First, prepare a directory where you will store the data for your
organism. For example:

\begin{footnotesize}
\begin{verbatim}
mkdir -p $HOME/rsat-add/data/Mygenus_myspecies/
\end{verbatim}
\end{footnotesize}


One you have this information, start the program
\program{install-organism}. You will be asked to enter the information
needed for genome installation.

\subsubsection{Updating your local configuration}


\begin{itemize}
\item Modify the local config file

\item You need to define an environment variable called
  RSA\_LOCAL\_CONFIG, containing the full path of the local config
  file.

\end{itemize}


\section{Installing genomes from EMBL files}

\RSAT also includes a script \program{parse-embl.pl} to parse genomes
from EMBL files. However, for practicaly reasons we generally parse
genomes from the NCBI genome repository. Thus, unless you have a
specific reason to parse EMBL files, you can skip this section.

The program \program{parse-embl.pl} reads flat files in EMBL format,
and exports genome sequences and features (CDS, tRNA, ...) in
different files.

As an example, we can parse a yeast genome sequenced by the
``Genolevures'' project
\urlref{http://natchaug.labri.u-bordeaux.fr/Genolevures/download.php}.

Let us assume that you want to parse the genome of the species
\textit{Debaryomyces hansenii}.

Before parsing, you need to download the files in your account, 

\begin{itemize}
\item Create a directory for storing the EMBL files. The last level of
  the directory should be the name of the organism, where spaces are
  replaced by underscores. Let us assume that you store them in
  the directory \file{\$RSAT/downloads/Debaryomyces\_hansenii}.

\item Download all the EMBL file for the selected organism. Save each
  name under its original name (the contig ID), followed by the
  extension \texttt{.embl})

\end{itemize}

We will check the content of this directory.

\begin{footnotesize}
\begin{verbatim}
ls -1 $RSAT/downloads/Debaryomyces_hansenii
\end{verbatim}
\end{footnotesize}

On my computer, it gives the following result

\begin{footnotesize}
\begin{verbatim}
CR382133.embl
CR382134.embl
CR382135.embl
CR382136.embl
CR382137.embl
CR382138.embl
CR382139.embl
\end{verbatim}
\end{footnotesize}

The following instruction will parse this genome.

\begin{footnotesize}
\begin{verbatim}
parse-embl.pl -v 1 -i  $RSAT/downloads/Debaryomyces_hansenii
\end{verbatim}
\end{footnotesize}

If you do not specify the output directory, a directory is
automatically created by combining the current date and the organism
name.  The verbose messages will indicate you the path of this
directory, something like
\file{\$HOME/parsed\_data/embl/20050309/Debaryomyces\_hanseni}.

You can now perform all the steps above (updating the config file,
installing oligo- and dyad frequencies, \ldots) as for genomes parsed
from NCBI.



\subsubsection{Installing a genome in the main \RSAT directory}

Once the genome has been parsed, the simplest way to make it available
 for all the users is to install it in the \RSAT genome directory. You
 can already check the genomes installed in this directory.

\begin{footnotesize}
\begin{verbatim}
ls -1 $RSAT/data/genomes/
\end{verbatim}
\end{footnotesize}

There is one subdirectory per organism. For example, the yeast data is
 in \file{\$RSAT/data/genomes/Saccharomyces\_cerevisiae/}. This
 directory is further subdivided in folders: \file{genome} and
 \file{oligo-frequencies}.

We will now create a directory to store data about
 Debaryomyces\_hansenii, and transfer the newly parsed genome in this
 directory.

\begin{footnotesize}
\begin{verbatim}
## Create the directory
mkdir -p $RSAT/data/genomes/Debaryomyces_hansenii/genome

## Transfer the data in this directory
mv $HOME/parsed_data/embl/20050309/Debaryomyces_hanseni/* \
  $RSAT/data/genomes/Debaryomyces_hansenii/genome

## Check the transfer
ls -ltr $RSAT/data/genomes/Debaryomyces_hansenii/genome
\end{verbatim}
\end{footnotesize}
