%%%%%%%%%%%%%%%%%%%%%%%%%%%%%%%%%%%%%%%%%%%%%%%%%%%%%%%%%%%%%%%%
%% Running web services with RSAT

\section{Using RSAT web services}

\subsection{Introduction}

\RSAT facilities can be used as web services (\concept{WS}), i.e. external developers
(you) can use \RSAT facilities in their own code. One important
advantage of web services is that they are using a standard
communication interface between client and server, and the libraries
exist in various languages (Perl, Python, java).

We explain below how to implement a WS client in Perl. 

\subsection{Requirements}

Before using a WS client, You need to install the Perl module
\program{SOAP::Lite}. Perl modules can be installed with the program
\program{cpan}, but for this you need root privileges. If this is not
your case, please ask your system administrato to install them for
you.

\subsection{Examples of WS client in Perl}

We show hereafter some simple examples of 

\subsubsection{Retrieving sequences from RSATWS}


\subsubsection{Work flow using RSATWS}


\subsection{Examples of WS client in java}


\subsection{Full documentation of the RSATWS interface}







