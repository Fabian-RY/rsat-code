\section{Generating random sequences}

Th program \program{random-seq} allows to generate random sequences
wih different random models.

It supports Bernoulli models (independence between successive
residues) and Markov models of any order. Markov models are generally
more suitable to represent biological sequences. 


We will briely illustrate different ways to use this program. 

\subsection{Sequences with identically and independently distributed (IID) nucleotides}

\begin{small}
\begin{verbatim}
random-seq -l 200 -r 20 -o rand_L200_N20.fasta
\end{verbatim}
\end{small}

We can now check th residue cmposition of this random sequence.

\begin{small}
\begin{verbatim}
oligo-analysis -v 1 \
  -i rand_L200_N20.fasta \
  -l 1 -1str -return occ,freq \
  -o rand_L200_N20_1nt-1str.tab
\end{verbatim}
\end{small}

\subsection{Sequences with nucleotide-specific frequencies}

In general, the residue composition of biological sequences is
biased. We can impose residue-specific probabilities for the random
sequence generation.

\begin{small}
\begin{verbatim}
random-seq -l 200 -r 20 -a a:t 0.3 c:g 0.2 \
  -o rand_L200_N20_at30.fasta 

oligo-analysis -v 1 \
  -i rand_L200_N20_at30.fasta \
  -l 1 -1str -return occ,freq \
  -o rand_L200_N20_at30_1nt-1str.tab
\end{verbatim}
\end{small}


\subsection{Introduction to Markov order}

Markov chain models involve local dependencies between successive
residues.

A Markov model of order $m$ means that the
probability of the residue at each position $i$ of the sequence
depends on the $m$ preceding residues.

\RSAT includes organism-specific Markov models, stored in the form of
oligonucleotide frequency tables (see chapter on pattern
discovery). for example, the calibration tables for Escherichia coli
K12 can be found in the following directory.

\begin{small}
\begin{verbatim}
ls -ltr $RSAT/data/genomes/Escherichia\_coli\_K12/oligo-frequencies
\end{verbatim}
\end{small}

For example, the file
\file{4nt\_upstream-noorf\_Escherichia\_coli\_K12-1str.freq.gz}
indicates the tetranucleotide frequencies for all the upstream
sequences of \organism{E.coli}.

\begin{small}
\begin{verbatim}
cd $RSAT/data/genomes/Escherichia_coli_K12/oligo-frequencies/

## Have a look at the content of the 4nt frequency file
zless zless 4nt_upstream-noorf_Escherichia_coli_K12-1str.freq.gz

\end{verbatim}
\end{small}

However, Markov models are described by transition frequencies
$P(r|W_m)$, i.e. the probability to osberve a given residue $r$ given
the preceding word $W_m$ of size $m$.

Transition frequencies are automatically derived from the table of
oligonucleotide frequencies, but one should take care of the fact
that, order to generate a random sequence with a Markov model of order
$m$, we need to use the frequency tables for oligonucleotides of size
$m+1$.

We can illustrate this convertion by converting the table of
dinucleotide frequencies into a transition matrix of first order. each
row of the matrix indicates the prefix $W_m$, and each column the
following residue $r$. For a Markov model of order 1, the prefixes are
single residues.

\begin{small}
\begin{verbatim}
convert-background-model \
  -i 2nt_upstream-noorf_Escherichia_coli_K12-1str.freq.gz  \
  -from oligo-analysis -to tab
\end{verbatim}
\end{small}

We can now obtain a Markov model of 2nd order, from the table of
trinucleotide frequencies.

\begin{small}
\begin{verbatim}
convert-background-model \
  -i 3nt_upstream-noorf_Escherichia_coli_K12-1str.freq.gz  \
  -from oligo-analysis -to tab
\end{verbatim}
\end{small}

The same operation can be extended to higher order markov models.

\subsection{Markov chain-based random sequences}

The random generator \program{random-seq} supports Markov chains of
any order (as far as the corresponding ferquency table has previously
been calculated). The Markov model is specified by indicating an
oligonucleotide frequency table. The table of oligonucleotides of
length $k$ is automatically converted in a transition table of order
$m=k-1$ duing the execution of \program{random-seq}. 


\begin{small}
\begin{verbatim}
random-seq -l 200 -r 20 \
  -expfreq $RSAT/data/genomes/Escherichia_coli_K12/oligo-frequencies/3nt_upstream-noorf_Escherichia_coli_K12-1str.freq.gz \
  -o rand_L200_N20_mkv2.fasta 
\end{verbatim}
\end{small}

A simpler way to obtain organism-specific Markov models is to use the
options \option{-bg} and \option{-org} of \program{random-seq}. 

\begin{small}
\begin{verbatim}
## This command generates random sequences with a Markov model of order 2,
## calibrated on all the non-coding upstream sequences of E.coli.
random-seq -l 200 -r 20 \
  -org Escherichia_coli_K12 -bg upstream-noorf -ol 3 \
  -o rand_L200_N20_mkv2.fasta 
\end{verbatim}
\end{small}



