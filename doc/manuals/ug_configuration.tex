\section{Configuring \RSAT}


In order to use the command-line version of \RSAT, you first need an
account on a Unix machine where \RSAT has been installed, and you
should know the directory where the tools have been installed (if you
don't know, ask assistance to your system administrator).

In the following instruction, we will assume that \RSAT is installed
in the directory \texttt{/home/rsat/rsa-tools}. This path has to be
replaced by the actual path where \RSAT has been installed on your
computer.

%%%%%%%%%%%%%%%%%%%%%%%%%%%%%%%%%%%%%%%%%%%%%%%%%%%%%%%%%%%%%%%%
%%%% Accessing the programs
%%%%%%%%%%%%%%%%%%%%%%%%%%%%%%%%%%%%%%%%%%%%%%%%%%%%%%%%%%%%%%%%
\subsection{Adding \RSAT  to your path}

Before starting to use the tools, you need to define an environment
variable (\texttt{RSAT}), and to add some directories to your path.

\begin{enumerate}

\item Create an environment variable named \textit{RSAT} and
  containing the path of rsa-tools.

  The way to create an environment variable depends on your shell. To
  know you shell, you can type

\lstset{language=csh}

\begin{lstlisting}
echo $SHELL
\end{lstlisting}
%$

The answer should be something like \result{/sbin/bash} or
\result{/bin/tcsh}.

Now, if we assume that \RSAT have been installed in the following
directory.

\begin{lstlisting}
/home/rsat/rsa-tools
\end{lstlisting}


We will declare the \RSAT path to your shell by defining an
environment variable named \texttt{RSAT}.  We will then add the path
of the \RSAT \ perl scripts, python scripts and binaries to your
path. In addition, add java jar files to your classpath.

\item If your default shell is \textbf{tcsh} or \textbf{csh}, type the
  following commands (you probably need to update the first command to
  specify the RSAT path of your machine.

\begin{lstlisting}
setenv RSAT /home/rsat/rsa-tools
set path=($path $RSAT/bin)
set path=($path $RSAT/perl-scripts)
set path=($path $RSAT/python-scripts)
set classpath=($classpath $RSAT/java/lib/NeAT_javatools.jar)
rehash
\end{lstlisting}



If your shell is bash, you should type the following command:

\begin{lstlisting}
export RSAT=/home/rsat/rsa-tools
export PATH=${PATH}:${RSAT}/bin
export PATH=${PATH}:${RSAT}/perl-scripts
export PATH=${PATH}:${RSAT}/python-scripts
export CLASSPATH=${CLASSPATH}:${RSAT}/java/lib/NeAT_javatools.jar
\end{lstlisting}



(the \texttt{rehash} command updates the list of executable programs)

\end{enumerate}

If you are using a different shell than bash, csh or tcsh, the
specification of environment variables might differ from the syntax
above.  In case of doubt, ask your system administrator how to
configure your environment variables and your path.

The specification of the environment variables and paths are required
each time you want to use \RSAT. You can add these specification to
your personal profile.  This file is normally found at the root of
your personal account, in the file \file{.bashrc} if your shell is
bash, or \file{.cshrc} if your shell is csh or tcsh. If you don't know
how to proceed, ask your system administrator.



\subsection{Checking the RSAT path}

 The previous step should have included all the \RSAT programs in
your path.  To check if it worked, just type:

\begin{lstlisting}
random-seq -l 350
\end{lstlisting}

If your configuration is correct, this command should return a random
sequence of 350 nucleotides.

You are now able to use any program from the \RSAT package, untill you
quit your session. It is however not very convenient to set the path
manually each time you open a new connection. You can modify your
default configuration by including the above commands in the file
\file{.cshrc} (in tcsh) or \file{.bashrc} (in bash) which should be
found at the root of your home directory. If you don't know how to
modify this file, see the system administrator.

