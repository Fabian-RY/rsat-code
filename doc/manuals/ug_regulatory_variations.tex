%%%%%%%%%%%%%%%%%%%%%%%%%%%%%%%%%%%%%%%%%%%%%%%%%%%%%%%%%%%%%%%%
%% Analysis of the impact of genetic polymorphism (inter-individual
%% variations) on transcription factor binding sites.


\chapter{Regulatory variations (rSNPs and insertion/deletions)}


\authors{J\'er\'emy Delerce and Jacques van Helden}

This chapter presents a series of tools to analyze the impact of
genetic polymorphism (inter-individual variations) on the binding of
transcription factors.

It combines a series of tools to 
\begin{enumerate}
\item download genomes, features, and variations from the \ensembl
  database and install them on the local \RSAT server;
\item obtain sequences of variants, exended on both sides;
\item scan these variant sequences with position-specific scoring
  matrices (PSSM) in order to detect variations that may affect
  transcription factor binding.
\end{enumerate}

\section{Requirements}

The functionalities described below require a connection to the
\ensembl database, via their Perl API\footnote{API: application
  programmatic interface}. 

In principle, the Ensembl Perl modules should have been installed
together with \RSAT (see \RSAT installation guide for details).

% \section{Installing genomes and variations from \ensembl}

% \subsection{\program{install-ensembl-genome}}

% The program \program{install-ensembl-genome} downloads the complete
% genomic sequence of a given organism from the \ensembl Web site, and
% installs it on the local \RSAT site. It also installs the descriptions
% of genomic features (genes, CDS, mRNAs, ...), and the variations.

% As usually, the complete help message can be obtained with the option
% \option{-help}.

% \begin{lstlisting}
% ## Get the description of the program + all options
% install-ensembl-genome -help
% \end{lstlisting}

% Before installing a genome, it is generally a good idea to know which
% genomes are available. For this, use the option
% \option{-available\_species}.

% \begin{lstlisting}
% ## Retrieve the list of supported species on EnsEMBL
% install-ensembl-genome -v 1  -available_species
% \end{lstlisting}

% \emph{Note:} inter-individual variations are available for a subset
% only of the genomes available in \ensembl. The option
% \option{-available\_species} indicates, for each species, the
% availability (genome, features, variations). Obviously, the analysis
% of regulatory variations only makes sense for the genomes documented
% with variations.

% We can now download the complete genomic sequence for the species of
% our choice. For the sake of space and time economy, we will use a
% small genome for this manual: the budding yeast \org{Saccharomyces
%   cerevisiae}. 

% \emph{Beware}: some installation steps take a lot of time, in
% particular the installation of genomic features, because the \ensembl
% interface requires to send individual queries for each gene
% separately. The full installation can thus take several hours. This is
% not a big issue, since installing a genome is not a daily task, but it
% is worth knowing that the whole process requires a continuous
% connection during several hours.

% \begin{lstlisting}
% ## Install the genome sequences for a selected organism
% install-ensembl-genome -v 2 -species Saccharomyces_cerevisiae
% \end{lstlisting}

% The download time depends on genome size, on the speed of your
% internet connection, and on the number of genes.

% \subsection{Installing genomes from Ensembl genomes}

% The historical \ensembl project \urlref{http://www.ensembl.org/}
% was focused on vertebrate genomes + a few model organisms
% (\org{Saccharomyces cerevisiae}, \org{Drosophila melanogaster},
% \ldots).

% A more recent project called \ensemblgenomes
% \urlref{http://ensemblgenomes.org/} extends the \ensembl project to a
% wider taxonomic space.

% The program \program{install-ensembl-genome} supports the installation
% of genomes from \ensembl as well as \ensemblgenomes. By default, it
% opens a connection to the historical \ensembl database, but the option
% \option{-ensembl\_genomes} enables to install genomes from the new
% project \ensemblgenomes.

% \begin{lstlisting}
% ## Get the list of available species from the extended project
% ## EnsemblGenomes
% install-ensembl-genome -v 2 -available_species -db ensemblgenomes \
%    -o available_species_at_EnsemblGenome.txt
% \end{lstlisting}

% You can then locate your genome of interest in the file
% \file{available\_species\_at\_EnsemblGenome.txt}, and start the
% installation (don't forget the option \option{-ensembl\_genomes}.

% \begin{lstlisting}
% ## Install Escherichia coli (strain K12 MG1665) from EnsemblGenomes
% install-ensembl-genome -v 2 -db ensemblgenomes \
%    -species Escherichia_coli_str_k_12_substr_mg1655
% \end{lstlisting}


% \subsection{Downloading variations}

% The program \program{download-ensembl-variations} downloads variations
% from the \ensembl Web site, and installs it on the local \RSAT
% site. 

% This program relies on \program{wget}, which must be installed
% beforehand on your computer.

% \begin{lstlisting}
% ## Retrieve the list of supported species in the EnsEMBL variation database
% download-ensembl-variations -v 1  -available_species
% \end{lstlisting}

% We can now download all the variations available for the yeast.

% \begin{lstlisting}
% ## Retrieve the list of supported species in the EnsEMBL variation database
% download-ensembl-variations -v 1 -species Saccharomyces_cerevisiae
% \end{lstlisting}


\section{Detecting regulatory variations}

\subsection{Scanning a selected variation with selected matrices}

\tbw

\subsection{Obtaining a list of disease-associated variation IDs}

In order to test the retrieval of multiple variations, we will first
select a realistic case, by getting a list of Human variations
associated to some disease (for instance diabetes).

Open a connection to the dbSNP database
\urlref{http://www.ncbi.nlm.nih.gov/SNP/}, and paste the following
test in the query box:

\begin{lstlisting}

("Homo sapiens"[ORG]) AND (diabetes[Text Word]) AND ("false"[Not Reference Assembly])
\end{lstlisting}

\tbw


\subsection{Scanning a list of selected variations with a list of matrices}

\tbw

\subsection{Scanning all variations with a selected matrix}

\tbw



