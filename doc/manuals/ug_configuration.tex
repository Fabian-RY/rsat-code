\section{Configuring \RSAT}


In order to use the command-line version of \RSAT, you first need an
account on a Unix machine where \RSAT has been installed, and you
should know the directory where the tools have been installed (if you
don't know, ask assistance to your system administrator).

In the following instruction, we will assume that \RSAT is installed
in the directory 

\texttt{/home/fred/rsat}. 

This path has to be
replaced by the actual path where \RSAT has been installed on your
computer.

%%%%%%%%%%%%%%%%%%%%%%%%%%%%%%%%%%%%%%%%%%%%%%%%%%%%%%%%%%%%%%%%
%%%% Accessing the programs
%%%%%%%%%%%%%%%%%%%%%%%%%%%%%%%%%%%%%%%%%%%%%%%%%%%%%%%%%%%%%%%%
\subsection{Adding \RSAT  to your path}

Before starting to use the tools, you need to define an environment
variable (\texttt{RSAT}), and to add some directories to your path.

For the following instructions, we will denote as \file{[RSAT\_PARENT\_PATH]}
the path of the parent folder in which the \RSAT suite has been
downloaded.

For example, if the \file{rsat} folder has been installed in your
home directory (it is thus found at \file{/home/fred/rsat}), you
should replace \file{[RSAT\_PARENT\_PATH]} by \file{/home/fred} in all the
following instructions. 


\begin{enumerate}

\item Identify your \texttt{SHELL}.

  The way to execute the following instructions depends on your
  ``shell'' environment. If you don't know which is your default
  shell, type

\lstset{language=csh}

\begin{lstlisting}
echo $SHELL
\end{lstlisting}
%$

The answer should be something like \result{/sbin/bash} or
\result{/bin/tcsh}

\item Declare the \RSAT environment variables.

  We will define configuration parameters necessary for \RSAT.  This
  includes an environment variable named \texttt{RSAT}.  We will then
  add the path of the \RSAT \ perl scripts, python scripts and
  binaries to your path.

\begin{itemize}

\item \emph{If your default shell is \textbf{tcsh} or \textbf{csh}},
  copy the following lines in a file named \file{.chsrc} that should
  be found at root of your account.

  \emph{Note}: replace [PARENT\_PATH] by the full path of the directory
  in which the rsat folder has been created.

\begin{lstlisting}
################################################################
## Configuration for Regulatory Sequence Analysis Tools (RSAT)
setenv RSAT [PARENT_PATH]/rsat
set path=($RSAT/bin $path)
set path=($RSAT/perl-scripts $path)
set path=($RSAT/python-scripts $path)
\end{lstlisting}
%% closing $

\item \emph{If your shell is bash}, you should copy the following
  lines in a file named \file{.bash\_profile} at the root of your
  account (depending on the Unix distribution, the bash custom
  parameters may be declered in a file named \file{.bash\_profile} or)
  \file{.bashrc}, in case of doubt ask your system administrator).

  \emph{Note}: replace [PARENT\_PATH] by the full path of the directory
  in which the rsat folder has been created.

\begin{lstlisting}
################################################################
## Configuration for Regulatory Sequence Analysis Tools (RSAT)
export RSAT=[PARENT_PATH]/rsat
export PATH=${RSAT}/bin:${PATH}
export PATH=${RSAT}/perl-scripts:${PATH}
export PATH=${RSAT}/python-scripts:${PATH}
\end{lstlisting}
%% closing $

\item If you are using a different shell than bash, csh or tcsh, the
specification of environment variables might differ from the syntax
above.  In case of doubt, ask your system administrator how to
configure your environment variables and your path.

\end{itemize}

\item In order for the variables to be taken in consideration, you
  need to log out and open a new terminal session. To check that the
  variables are correctly defined, type.

\begin{lstlisting}
  echo $RSAT
\end{lstlisting}
%%$

In the example above, this command should return

\begin{lstlisting}
/home/fred/rsat
\end{lstlisting}


\end{enumerate}

