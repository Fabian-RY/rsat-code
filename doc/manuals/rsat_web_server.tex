
%\documentclass{book}
\documentclass[12pt,a4paper, twoside]{scrreprt} % KOMA-class Neukam and Kohm, scrbook alternatively
%\documentstyle[makeidx]{book}
\makeindex
%\usepackage{color}
\usepackage[usenames]{color}
\usepackage{times}
\usepackage{graphics}
\usepackage{latexsym}
\usepackage{makeidx}


%%%%%%%%%%%%%%%%%%%%%%%%%%%%%%%%%%%%%%%%%%%%%%%%%%%%%%%%%%%%%%%%
%%%%%%%%%%%%%%%%%%%%%%%%%%% commands %%%%%%%%%%%%%%%%%%%%%%%%%%%
\newcommand{\tbw}{\textbf{TO BE WRITTEN}}
\newcommand{\RSAT}{\textbf{\textit{RSAT}}}
\newcommand{\file}[1]{\textit{#1}}
\newcommand{\concept}[1]{\index{#1}\textsl{#1}}
\newcommand{\command}[1]{\begin{footnotesize}\begin{quote}\textcolor{Blue}{\texttt{#1}}\end{quote}\end{footnotesize}}
\newcommand{\result}[1]{\begin{footnotesize}\begin{quote}\textcolor{OliveGreen}{\texttt{#1}}\end{quote}\end{footnotesize}}
\newcommand{\program}[1]{\textbf{\textsl{#1}}}
\newcommand{\option}[1]{\texttt{#1}}
\newcommand{\email}[1]{\textit{#1}}

\newcommand{\address}[1]{\small{#1}}
\newcommand{\org}[1]{\textit{#1}}
\newcommand{\gene}[1]{\textit{#1}}
\newcommand{\seq}[1]{\texttt{#1}}

\newcommand{\url}[1]{\textit{#1}}
\newcommand{\urlref}[1]{\footnote{\textit{#1}}}

\newcommand{\scmbb}{
	Service de Conformation des Macromol\'{e}cules Biologiques et de Bioinformatique, \\
	Universit\'{e} Libre de Bruxelles, \\
	Campus Plaine, CP 263, Boulevard du Triomphe, B-1050 Bruxelles, Belgium. \\
	Tel: +32 2 650 2013 - Fax: +32 2 650 5425
}

%%%%%%%%%%%%%%%%%%%%%%%%%%%%%%%%%%%%%%%%%%%%%%%%%%%%%%%%%%%%%%%%
%%%%%%%%%%%%%%%%%%%%%%%%% environments %%%%%%%%%%%%%%%%%%%%%%%%%

\def\ex@mple{example}
\def\ex@rcise{exercise}
\def\theoremfont{\ifx\@currenvir\ex@mple
	\def\@thmfont{\rm} \else
	\ifx\@currenvir\ex@rcise
	\def\@thmfont{\rm} \else
	\def\@thmfont{\it}\fi\fi}
\newtheorem{exercise}{Exercise}[chapter]
\newtheorem{example}{Example}[chapter]




\begin{document}

\RSATtitlePage{Web server configuration for \RSAT}

% \title{Regulatory Sequence Analysis Tools \\
% Web server installation}
% \author{
% 	Jacques van Helden \\
% 	\email{jvhelden@ulb.ac.be} \\
% 	\bigre 
% }
% \maketitle

\newpage
\tableofcontents
\newpage

\chapter{Web server configuration for \RSAT}

\section{Description}

This documents describes the installation procedure for the web server
of the \textbf{Regulatory Sequence Analysis Tools} (\RSAT).

It assumes that you already installed the perl scripts and the
genomes, as described in the \RSAT installation guide.

%%%%%%%%%%%%%%%%%%%%%%%%%%%%%%%%%%%%%%%%%%%%%%%%%%%%%%%%%%%%%%%%
% Web server installation

\section{Installing a local web server}

The Regulatory Sequence Analysis Tools include a web server, which
offers a user-friendly interface for biologists. The main server is
available for academic users at \url{http://rsat.ulb.ac.be/rsat/}. A
few additional mirrors have been installed in different countries.

\subsection{Web server pages}

The web pages are located in the directory
\file{rsa-tools/public\_html}. This directory contains both the HTML
pages, and the CGI scripts.

\subsection{Apache modules}

The RSAT interface relies on CGI (for the earlier tools) and PHP (for
the most recent tools). These modules should be installed on the web
server, and activated in the Apache configuration files. 

\subsubsection{PHP module for Mac OSX}

If your server is running under Mac OSX, you need to install a recent
version (at least v5) of the php module, which can be found at the following site. 

\url{href=http://www.entropy.ch/software/macosx/php/}

\subsection{Configuration of the Apache server}

In order to provide web access to the rsa-tools, you need to adapt the
configuration of your web server. This requires root privileges
(system administrator). 


The detailed settings depend on your web server program. We provide
here an example of typical settings for the Apache server (the most
widely used web server).

In summary, the configuration includes the following steps.

\begin{enumerate}
\item Open the apache configuration file \file{/etc/httpd/httpd.conf}

\item Specify an alias for rsa-tools in the appropriate section.

\begin{lstlisting}
Alias /rsa-tools /home/myaccount/rsa-tools/public_html/
\end{lstlisting}

  (you need of course to adapt this command according to your own
  \RSAT directory)

\item Associate .cgi extension to CGI scripts

Make sure the following line is present in the config file. If the
server has not yet been configured, the line is commented, and you
need to remove the \texttt{\#} character before it.

\begin{lstlisting}
AddHandler cgi-script .cgi
\end{lstlisting}

\item Give authorization to execute CGI scripts in the rsa-tools directory

\begin{lstlisting}
ScriptAlias /rsa-tools/ /home/myaccount/rsa-tools/public_html/
\end{lstlisting}

\item Specify the access options for the rsa-tools directory.

\begin{lstlisting}
<Directory "/home/myaccount/rsa-tools/public_html/">
   AllowOverride None
   AddHandler cgi-script .cgi
   Options ExecCGI Indexes
   Order allow,deny
   Allow from all
</Directory>
\end{lstlisting}

\end{enumerate}

These are the basic steps to configure the web access to
\RSAT. Depending on your operating system, you probably need to
specify some additional settings. For example, on the Max OSX version
of Apache server allows to define a user-specific configuration in the
directory \file{/etc/httpd/users}.

Note that you need to restart the web server for these changes to take
effect. The command to restart the Apache server depends on the
version installed on your computer. 

Note that you need system administration privileges to restart the
Apache server.

You can try the following commands, and if they don'r work check the
documentation of your web server.

\begin{lstlisting}
sudo httpd restart

sudo apachectl restart
\end{lstlisting}

\subsection{Editing the configuration file for the \RSAT server}

If you want to install a web server, you need to edit two variables on
the file \file{RSA.config}. Open this file with a text editor, and
specify the variables \texttt{\$config\_site} and \texttt{\$WWW\_RSA}
according to your local configuration.


\subsection{Configuring the RSAT options for the web server}

You need to adapt the RSAT configuration file to indicate the
configuration of your server. For this, edit the file
\texttt{\$RSAT/RSA.config}.

\subsubsection{IP address of your machine}

By default, the RSAT web server is configured to be used from the
machine on which it is installed. 

\begin{lstlisting}
$WWW_RSA = 	"http://localhost/rsa-tools/";
\end{lstlisting}

If you want to give access from external machines, you need to adapt
the following line and replace ``localhost'' by the IP address of your
machine. Let us assume that your mahine has the address
www.myserver. You should then edit the row as follows.

\begin{lstlisting}
$WWW_RSA = 	"http://www.myserver/rsa-tools/";
\end{lstlisting}

\subsubsection{Setting the environment variable RSAT}

For the basic RSAT configuration, you had to define an environment
variable RSAT. This is necessary for the perl program to know the
location of your programs. If this variable is not specified, the
scripts are stopped with an error message.

\begin{lstlisting}
unless ($ENV{RSAT}) {
    die "Error: the environment variable RSAT needs to be defined\n";
}
\end{lstlisting}

Since the web server is ran by another user, you need to define this
environment variable for this user as well. For this, edit the
above line in the following way.

\begin{lstlisting}
unless ($ENV{RSAT}) {
    $ENV{RSAT} = "/home/myaccount/rsa-tools";
}
\end{lstlisting}


\subsection{Testing the web server}

To test the werb server, open a web browser and connect your \RSAT server

\url{http://www.myserver/rsa-tools/} 

Of course you  need to adapt the URL according to your IP address.

If the connection works, try to execute the demonstration of the
following pages.

\begin{description}
\item[\program{retrieve-seq}] to test the correct installation of genomes. 

\item[\program{oligo-analysis}] to test the correct installation of
background oligonucleotide frequencies.

\item[\program{feature-map}] to test the correct installation of the
graphical librairies.

\end{description}

\section{Managing a local web server}

\subsection{Access logs}

Each time a script is executed via the \RSAT server, some basic
information is stored in a log file. This information is minimal: it
is restricted to the time, name of the script executed, and the IP
address of the client machine. We do not want to store any additional
information (e.g. selected organism, lists of genes), for obvious
confidentiality reasons.

The log files are saved in the directory \file{rsa-tools/logs}. There
is one file per month.


\subsection{Cleaning the temporary directory}

The web server stores result files in a temporary directory
\file{\$RSAT/public\_html/tmp/}. These files should remain 3 days on
the server, in order to allow users to consult their results.

\subsubsection{Manual cleaning}

The \RSAT package includes a make script to clean old files in the
temporary directory.

\begin{lstlisting}
cd rsa-tools
make -f makefiles/server.mk clean_tmp
\end{lstlisting}

This command cleans all the files older than 3 days. You can clean
more recent files by modifying the variable CLEAN\_DATE.

\begin{lstlisting}
make -f makefiles/server.mk clean_tmp  CLEAN_DATE=1
\end{lstlisting}

This will clean all files older than 1 day.

\subsubsection{Automatic cleaning}

The automatic management of the temporary directory can be greatly
facilitated the \program{crontab} command. For this, you need to add a
command to your personal \texttt{crontab} configuration file.

\begin{enumerate}
\item Start to edit the crontab command file

\begin{lstlisting}
crontab -e
\end{lstlisting}

This will open your \file{crontab} file with your default text editor
(this default editor can be specified with the environment variable
EDITOR or VISUAL).

\item Add the following line to the \file{crontab} file. 

\begin{lstlisting}
02 04 * * * make -f /home/myaccount/rsa-tools/makefiles/server.mk clean_tmp
\end{lstlisting}

This will execute the make script \file{server.mk}, with the target
\texttt{clean\_tmp}, every day, at 04:02 AM. 

\item Save the modified crontab file and close your text editor.

\end{enumerate}

In principle, you will receive an email from \program{crontab} each
time the command is executed.

Note that the command \program{crontab} takes effect only if the
system administrator has activated the command \command{cron}. If you
notice that the temporary files are not properly cleaned, please
contact your system administrator to check the cron command.

\end{document}
