\chapter{Pathway inference}

\section{Introduction}
The idea of pathway inference is to connect a given set of seed nodes
in the network and thereby extracting a sub-network that is optimal
according to certain criteria (e.g. minimal weight or maximal relevance).\\
In the context of biological networks, the goal is to obtain a valid pathway for a set
of biological entities of interest, e.g. genes from microarray data or compounds from metabolomic data.
For instance, genes whose products participate in the same metabolic pathway are often co-expressed or grouped
together in operons or regulons. We may try to reconstruct this metabolic pathway by associating
the gene products to relevant reactions and connecting these reactions in a metabolic network.
The resulting sub-network may be a known metabolic pathway or an unknown pathway consisting of known
pathways or known reactions and compounds. In the context of microarray data, pathway inference from a
set of co-expressed genes may predict which pathways are up- or down-regulated.\\

\section{Inferring a pathway for a set of co-expressed genes}

As an example, we take the case study discussed in \cite{vanHelden01}.
In this case study, a pathway is assembled from genes in the cell-cycle regulated MET cluster \cite{Spellman98}.
Results described in this tutorial have been obtained with KEGG RPAIR version 49.0.

\subsection{Protocol for the web server}

\begin{enumerate}

\item In the \neat  menu, select the entry \program{Pathwayinference}.

\item Copy-paste the gene names below in the seed nodes text field:
<pre>
Met3
Met14
Met16
Met5
Met10
Met17
Met6
<\pre>

\item Select "Genes/enzymes" as identifier type.

\item In the text field "Genes/enzymes are from organism" type sce, the KEGG abbreviation for \textit{Saccharomyces cerevisiae}.

\item Push the \option{GO} button.

\end{enumerate}

The result of the mapping of the given genes to KEGG RPAIRS (reactant pairs, \cite{Kotera2004}) is displayed.
Since more than one reactant pair is associated to each gene, we end up with a group of reactant pair groups.
Note that each gene (except for Met5) is associated to one or more EC numbers, each of which has been mapped
to its corresponding reactions in KEGG, which have in turn be mapped to their corresponding reactant pairs.\\

You can now select how to deal with the groups. This is a sensitive choice that strongly
affects the inferred pathway and which depends on your data.
In case of reactant pairs, it is usually not recommended to keep the groups, because
they might contain mutually exclusive reactant pairs (that is reactant pairs belonging to the same reaction).
We prevent pathwayinference in these cases, to avoid the same reaction occurring more than once in the resulting sub-network.
Instead, the groups can be re-arranged such that mutually exclusive reactant pairs are joined.
The original groups in this case are replaced by groups that reflect
"possible catalytic mechanisms" enabled by the given enzymes.\\
In general, if you keep the original groups, you assume implicitely that only
a subset of the reactions associated to the given gene will be active in the pathway.
If you think that all reactions associated to a gene might be active, choose "Treat each group member as a separate group." or,
in case of reactant pairs, choose "Treat each group member as a separate group, but ignore group members that occur more than once and merge exclusive reactant pairs into groups.".\\

For the study case, we recommend you to choose "Treat each group member as a separate group, but ignore group members that occur more than once and merge exclusive reactant pairs into groups.".

Push \option{GO}. In a few minutes, the result page will be displayed.



\subsection{Protocol for the command-line tools}

This section assumes that you have installed the RSAT/NeAT command line tools.

Pathwayinference is a web application that calls the pathwayinference web service.
You can use the Pathwayinference command line tool on the networks provided in the
network repository (check the Pathwayinference Manual for this) to reproduce
results obtained with the web application on command line. Note that the mapping of genes to reactions
and re-grouping of seeds can only be done via the web application.

Type the following command in one line:
{\color{Blue} \begin{footnotesize}
		\begin{verbatim}
java -Xmx800m graphtools.algorithms.Pathwayinference -g RPAIRGraph_allRPAIRs_undirected.txt -f flat
	     -s 'RP00016#RP00182/RP00647/RP00561/RP00143#RP00960#RP04049/RP00096#RP00168#RP04532/RP00003/RP00446/RP00946#RP00857/RP04474/RP00050#RP04533'
	     -b -y con -P -u -x 0.05
	\end{verbatim} \end{footnotesize}
	}

\subsection{Interpretation of the results}

The resulting sub-network contains the pathway given in \cite{vanHelden01}. The reaction catalysed by
the MET5 enzyme (which was not found in KEGG) has been correctly inferred.

The pathway described in the example is:\\
Sulfate 2.7.7.4 Adenylyl sulfate 2.7.1.25 3'phosphoadenylylsulfate 1.8.99.4 sulfite 1.8.1.2 sulfide (alias hydrogen sulfide) 4.2.99.10 Homocysteine 2.1.1.14 L-Methionine\\

It unites the sulfur assimilation and methionine biosynthesis pathways.

The matching part of the inferred pathway is:\\
\textbf{RP00960} Adenylyl sulfate RP05930 Sulfite \textbf{RP00168} Hydrogen sulfide RP01406 L-Homocysteine \textbf{RP0096}\\
Seeds are printed in bold.

In addition, the inferred pathway contains a branch that leads
from Adenylylselenate via 3'-Phosphoadenylylselenate, Selenite and hydrogen selenide to selenocysteine.
This branch mirrors sulfur incorporation, but instead of sulfur, selenium is incorporated into Homocysteine.\\

The presence of both the selenium and sulfur incorporation pathways in the inferred sub-network
reflects the well-known fact that selenium might replace sulfur in metabolism.\\

This example demonstrated that given a set of differentially expressed genes from
micro-array data and a metabolic network, it is possible to infer a metabolic pathway
that might be affected by altered expression of the query genes.

\subsection{Summary}

Pathwayinference is a very flexible tool that can work on any kind of biological network


\subsection{Strengths and Weaknesses of the approach}


\subsubsection{Strengths}

\subsubsection{Weaknesses}

\subsection{Troubleshooting}