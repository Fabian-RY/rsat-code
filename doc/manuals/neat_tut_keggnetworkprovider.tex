\chapter{KEGG network provider}

\section{Introduction}

% what can you do with the tool
KEGG network provider allows you to extract metabolic networks from KEGG that are specific to a set of organisms.
In addition, you can exclude certain compounds or reactions from these networks.

% competing tools
A range of tools works with KGML files. Click on ``Manual -> Related tools'' to see a selection of them.
KEGG network provider differs from these tools by allowing also the extraction of RPAIR networks and by supporting
filtering of compounds, reactions and RPAIR classes.

KEGG network provider itself has no network analysis or visualization functions,
but you can use a NeAT tool (a choice of them will appear upon termination of network construction)
or any other graph analysis tool that reads gml, VisML or dot format for these purposes.

For visualization of KEGG networks, you can use iPATH, KGML-ED or metaSHARK.
Yanasquare and Pathway Hunter Tool offer organism-specific KEGG network construction in combination
with analysis functions.

% side compounds
It should be noted that KEGG annotators omitted side compounds in the KGML files. Thus, certain molecules
(such as CO2, ATP or ADP) might be absent from the metabolic networks extracted from these files.

\section{Yeast metabolic network construction}

\subsection{Study case}

Our study case consists in the construction of a metabolic network from several yeast species. We will construct a filtered
and an unfiltered version of this network and compare metabolic path finding results.

\subsection{Protocol for the web server}

\begin{enumerate}

\item In the \neat  menu, select the entry \program{Download organism-specific networks from KEGG}.

  In the right panel, you should now see a form entitled
  ``KEGG network provider''.

\item Click on the button \option{DEMO} located at the bottom of the form.

  The KEGG network provider form has now loaded the organism identifiers of five yeast species.

\item Click the checkbox \option{directed network} to construct a directed metabolic network.

\item Click on the button \option{GO}.

  The network extraction should not take more than one minute.
  Then, a link to the extracted network is displayed.
  In addition (for formats \textit{tab-delimited} and \textit{gml}), the Next step panel should appear.

\item Click on the button ``Find metabolic paths in this graph'' in the Next step panel. This button opens the
Metabolic pathfinder with the yeast network pre-loaded.

\item Enter C00025 (L-Glutamate) as source node and C00148 (L-Proline) as target node.

\end{enumerate}

To see how results change with modified weight, you can repeat steps 1 and 2.
In the metabolic path finding form, select Reaction graph instead of Subreaction graph (which is selected by default) and
follow step 3 to 5. You will notice in the seed node selection form that the reaction identifiers are no longer mapped to
reactant pairs.

\subsection{Protocol for the command-line tools}

You may download a command line version of the KEGG network provider, which also includes the path finding tool.
Check the manual of the KEGG network provider (). \\

\subsection{Interpretation of the results}




\subsection{Summary}


\subsection{Troubleshooting}

\begin{enumerate}

\item An empty graph (with zero nodes and edges) is returned.
Make sure that the entered organism identifiers are valid in KEGG.
They should consist of three to four letters only. If in doubt,
check in the provided KEGG organism list.

\end{enumerate}
