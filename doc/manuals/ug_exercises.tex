%%%%%%%%%%%%%%%%%%%%%%%%%%%%%%%%%%%%%%%%%%%%%%%%%%%%%%%%%%%%%%%%
%%%% EXERCISES
%%%%%%%%%%%%%%%%%%%%%%%%%%%%%%%%%%%%%%%%%%%%%%%%%%%%%%%%%%%%%%%%

\chapter{Exercises}

As an exercise, we will now combine the different tools described
above to analyse the full set of promoters from \organism{Arabidopsis
thaliana}. We define ourselves the following goals :
\begin{enumerate}
\item Discover motifs which are over-represented in the complete set
of upstream sequences for the selected organism. 
\item Try different parameters for this pattern discovery, and compare
the results.
\item Use these over-represented patterns to scan full chromosomes
with a sliding window, in order to evaluate if we can predict promoter
locations on the sole basis of pattern occurrences. Find optimal
parameters for the prediction of promoter locations.
\end{enumerate}

\section{Some hints}

\subsection{Sequence retrieval}

The first step will be to retrieve the full complement of upstream
sequences. Since we have no precise idea about the best sequence size,
we will try several reasonable ranges, each roughly corresponding to a
given functionality.

\begin{description}
\item[from -1 to -200] this regions is likey to contain mostly 5'UTR.
\item[from -1 to -400] this region is likely to contain the 5' UTR and
the proximal promoter.
\item[from -1 to -1000] this region is likely to include the 5'UTR,
as well as the proxima and distal promoters. 
\item[from -1 to -2000] an even larger range, which probably contains
most of the upstream cis-acting elements in \organism{A. thaliana}.
\end{description}

In all cases, we will clip upstream ORFs, because they would bias the
oligonucleotide composition.

Write the commands which will retrieve all upstream sequences over the
specified range. Beware, the sequence files may occupy a large space
on the disk, it is probably wise to directly compress them by adding
the extension \texttt{.gz} to the output file.

\subsection{Detection of over-represented motifs}

In a first step, we will restrict our analysis to
hexanucleotides. Once all the subsequent steps (full chromosome
scanning) will be accomplished, we will redo the complete analysis
with different oligonucleotide lengths, and compare the efficiency of
promoter prediction.

Detect over-represented oligo-nucleotides with different estimators of
expected frequencies: Markov chains of different orders, non-coding
frequencies. 

Do not forget to prevent counting self-overlapping matches.

