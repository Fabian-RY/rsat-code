\chapter{Comparisons between networks}

\section{Introduction}

Protein interaction networks have deserved a special attention for
molecular biologists, and several high-throughput methods have been
developed during the last years, to reveal either pairwise
interactions between proteins (two-hybrid technology) or protein
complexes (methods relying on mass-spectrometry). The term
\concept{interactome} has been defined to denote the complete set of
interactions between proteins of a given organism.

Interactome data is typically represented by an un-directed graph,
where each node represents a polypeptide, and each edge an interaction
between two polypeptides.

The yeast interactome was characterized by the two-hybrid method by
two independent groups, Uetz and co-workers \cite{Uetz:2000}, and Ito
and co-workers \cite{Ito:2001}, respectively. Surprisingly, the two
graphs resulting from these experiments showed a very small
intersection.

In this tutorial, we will use the program \program{compare-graphs} to
analyze the interactome graphs published by from Uetz and Ito,
respectively.

We will first perform a detailed comparison, by merging the two
graphs, and labelling each node according to the fact that it was
found in Ito's network, in Uetz' network, or in both. We will then
compute some statistics to estimate the significance of the intersection
between the two interactome graphs.

\section{Computing the intersection, union and differences between two graphs}

\subsection{Study case}

In this demonstration, we will compare the networks resulting from the
two first publications reporting a complete characterization of the
yeast interactome, obtained using the two-hybrid method.The first
network \cite{Uetz:2000} contains 865 interactions between 926
proteins.The second network \cite{Ito:2001} contains 786 interactions
between 779 proteins. We will merge the two networks (i.e. compute
their union), and label each edge according to the fact that it is
found in Ito's network, Uetz' network, or both. We will also compute
the statistical significance of the intersection between the two
networks.


\subsection{Protocol for the web server}

\begin{enumerate}

\item In the \neat menu, select the command \program{network
    comparison}. 

  In the right panel, you should now see a form entitled
  ``compare-graphs''.

\item Click on the button \option{DEMO}. 

  The form is now filled with two graphs, and the parameters have been
  set up to their appropriate value for the demonstration. At the top
  of the form, you can read some information about the goal of the
  demo, and the source of the data.

\item Click on the button \option{GO}. 

  The computation should take a few seconds only. The result page
  shows you some statistics about the comparison (see interpretation
  below), and a link pointing to the full result file. 

\item Click on the link to see the full result file. 


\end{enumerate}

\subsection{Protocol for the command-line tools}

If you have installed a stand-alone version of the NeAT distribution,
you can use the program \program{compare-graphs} on the
command-line. This requires to be familiar with the Unix shell
interface. If you don't have the stand-alone tools, you can skip this
section and read the next section (Interpretation of the results).


\subsection{Interpretation of the result}

The program \program{compare-graphs} uses symbols $R$ and $Q$
respectively, to denote the two graphs to be compared. Usually, $R$
stands for reference, and $Q$ for query. 

In our case, $R$ indicates Ito's network, whereas $Q$ indicates Uetz'
network. The two input graphs are considered equivalent, there is no
reason to consider one of them as reference, but this does not really
matter, because the statistics used for the comparison are
symmetrical,as we will see below.

\subsubsection{Union, intersection and differences}

The result file contains the union graph, in tab-delimited
format. This format is very convenient for inspecting the result, and
for importing it into statistical packages (R, Excel, \ldots).

The rows starting with a semicolon (;) are comment lines. They provide
you with some information (e.g. statistics about the intersection),
but they will be ignored by graph-reading programs. The description of
the result graph comes immediately after these comment lines.

Each row corresponds to one arc, and each column specifies one
attribute of the arc.

\begin{enumerate}
\item \textbf{source}: the ID of the source node
  
\item \textbf{target}: the ID of the target node
  
\item \textbf{label}: the label of the arc. As labels, we selected
  the option ``Weights on the query and reference''. Since the
  input graphs were un-weighted, edge labels will be used instead
  of weights. The label \texttt{<NULL>} indicates that an edge is
  absent from one input network.
  
\item \textbf{color} and \textbf{status}: the status of the arc
  indicates whether it is found at the intersection, or in one graph
  only.  A color code reflects this status, as indicated below.
  \begin{itemize}
  \item \concept{R.and.Q}: arcs found at the intersection between graphs $R$
    and $Q$. Default color: green.
  \item \concept{R.not.Q}: arcs found in graph $R$ but not in graph
    $Q$. Default color: violet.
  \item \concept{Q.not.R}: arcs found in graph $Q$ but not in graph
    $R$. Default color: red.
  \end{itemize}

  The result file contains several thousands of arcs, and we will of
  course not inspect them by reading each row of this file. Instead,
  we can generate a drawing in order to obtain an intuitive perception
  of the graph.

\subsubsection{Sizes of the union, intersection and differences}

The beginning of the result file gives us some information about the
size of the two input files, their union, intersection, and
differences.

\begin{footnotesize}
\begin{verbatim}
; Counts of nodes and arcs
;    Graph    Nodes    Arcs     Description
;    R        779      786      Reference graph
;    Q        926      865      Query graph
;    QvR      1359     1529     Union
;    Q^R      346      122      Intersection
;    Q!R      580      743      Query not reference
;    R!Q      433      664      Reference not query
\end{verbatim}
\end{footnotesize}

\subsubsection{Statistical significance of the intersection between two graphs}

The next lines of the result file give some statistics about the
intersection between the two graphs. These statistics are computed in
terms of arcs.

\begin{footnotesize}
\begin{verbatim}
; Significance of the number of arcs at the intersection
;  Symbol   Value     Description                        Formula
;  N        1359      Nodes in the union                    
;  M        922761    Max number of arcs in the union    M = N*(N-1)/2
;  E(Q^R)   0.74      Expected arcs in the intersection  E(Q^R) = Q*R/M
;  Q^R      122       Observed arcs in the intersection
;  perc_Q   14.10     Percentage of query arcs           perc_Q = 100*Q^R/Q
;  perc_R   15.52     Percentage of reference arcs       perc_R = 100*Q^R/R
;  Jac_sim  0.0798    Jaccard coefficient of similarity  Jac_sim = Q^R/(QvR)
;  Pval     2.5e-228  P-value of the intersection        Pval=P(X >= Q^R)
\end{verbatim}
\end{footnotesize}

A first interesting point is the maximal number of arcs ($M$) that can
be traced between any two nodes of the union graph. In our study case,
the graph obtained by merging Ito's and Uetz' data contains $N=1359$
nodes. This graph is un-directed, and there are no self-loops. The
maximal number of arcs is thus $M=N*(N-1)/2=922,761$. This number
seems huge, compared to the number of arcs observed in either Uetz'
($A_Q=865$) or Ito's ($A_R=786$) graphs. This means that these two
graphs are sparse: only a very small fraction of the node pairs are
linked by an arc.

The next question is to evaluate the statistical significance of the
intersection between the two graphs. For this, we can already compute
the size that would be expected if we select two random sets of arcs
of the same sizes as above ($A_Q=865$, $A_R=4,038$). 

If the same numbers of arcs were picked up at random in the union
graph, we could estimate the probability for an arc to be found in the
network $R$ as follows: $P(R) = A_R/M = 0.000852$. Similarly, the
probability for an arc of the union graph to be part of the network
$Q$ is $P(Q) = A_Q/M = 0.000937$.  The probability for an arc to be
found independently in two random networks of the same sizes as $R$
and $Q$ is the product of these probabilities.

\[P(QR) = P(Q)*P(R) = A_R/M \cdot A_Q/M = 7.98e-07\]

The number of arcs expected by chance in the intersection is the
probability multiplied by the maximal number of arcs.

\begin{eqnarray*}
E(QR) & = & P(QR) \cdot M  \\
 & = & (A_Q \cdot A_R)/M \\
 & = &  7.98e-07 \cdot 922761 = 0.74
\end{eqnarray*}

Thus, at the intersection between two random sets of interaction, we
woudl expect on the average a bit less than one interaction. It seems
thus clear that the 122 interactions found at the intersection between
he two published experiments is much higher than the random
expectation.

We can even go one step further, and compute the \concept{P-value}
of this intersection, i.e. the probability to select at least that
many interactions by chance. 

The probability to observe \textit{exactly} $x$ arcs at the
intersection is given by the hypergeometrical distribution.

\begin{equation}
\label{eq:hypergeometric_density}
P(QR=x)=\frac{C^{x}_{R}C^{Q-x}_{M-R}}{C^Q_{M}}
\end{equation}

where 
\begin{itemize}
\item[$R$] is the number of arcs in the reference graph;
\item[$Q$] i the number of arcs in the query graph;
\item[$M$] is the maximal number of arcs;
\item[$x$] is the number of arcs at the intersection between the two
  graphs.
\end{itemize}

By summing this formula, we obtain the P-value of the intersection,
i.e. the probability to observe \textit{at least} $x$ arcs at the
intersection.

\begin{eqnarray*}
\label{eq:hypergeometric_density_cdf}
Pval = P(QR>=x)=\sum_{i=x}^{min(Q,R)}P(X=i)=\sum_{i=x}^{min(Q,R)} \frac{C^i_{R}C^{Q-i}_{M-R}}{C^Q_{M}}
\end{eqnarray*}

We can replace the symbols by the numbers of our study case.

\begin{eqnarray*}
\label{eq:hypergeometric_density_cdf}
Pval & = & P(QR>=122) \\
 & = & \sum_{i=x}^{min(865,786)} \frac{C^i_{786}C^{865-i}_{922761-786}}{C^{865}_{922761}}  \\
 & =  2.5e-228
\end{eqnarray*}

This probabilty is so small that it comes close to the limit of
precision of our program ($\approx 10^{-321}$).

\end{enumerate}

\subsubsection{Summary}

In summary, the comparison revealed that the number of arcs found in
common between the two datasets (Ito and Uetz) is highly significant,
despite the apparently small percentage of the respective graphs it
represents (14.10\% of Ito, and 15.52\% of Uetz).

\section{Strengths and weaknesses of the approach}



\section{Exercises}

\begin{enumerate}

\item Using the tool the tool \program{network randomization},
  generate two random graphs of 1000 nodes and 1000 arcs each (you
  will need to store these random networks on your hard drive).  Use
  the tool \program{network comparison} to compare the two random
  graphs.
  Discuss the result, including the following questions: 
  \begin{enumerate}
  \item What is the size of the intersection ? Does it correspond to
    the expected value ? 
  \item Which P-value do you obtain ? How do you interpret this P-value ? 
  \end{enumerate}

\item Randomize Ito's network with the tool \program{network
    randomization}, and compare this randomized graph with Uetz'
  network. Discuss the result in the same way as for the previous
  exercise.

\end{enumerate}

\section{Troubleshooting}

\begin{enumerate}

\item The P-value of the intersection between two graphs is 0. Does it
  mean that it is impossible to have such an intersection by chance
  alone ?

  No. Any intersection that you observe in practice might occur by
  chance, but the limit of precision for the hypergeometric P-value is
  $\approx 10^{-321}$. Thus, a value of 0 can be interpreted as $Pval <
  10^{-321}.$

\item The web server indicates that the result will appear, and after
  a few minutes my browser displays a message ``No response the
  server''.

  How big are the two graphs that you are comparing ? In principle,
  compare-graphs can treat large graphs in a short time, but if your
  graphs are very large (e.g. several hundreds of thousands of arcs),
  the processing time may exceed the patience of your browser. In such
  case, you should consider either to install the stand-alone version
  of \neat on your computer, or o write a script that uses \neat via
  their Web services interface.

\end{enumerate}