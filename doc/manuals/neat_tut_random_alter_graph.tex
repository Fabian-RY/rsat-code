\chapter{Graph randomization and alteration}

\section{Introduction}
Although negative controls and method evaluation are crucial points to the experimental biologist, 
this is far from being the same in bioinformatics where, too often, no negative controls are associated to 
the predictions or scientific discoveries.

For this reason, in NeAT we developped programs allowing to randomize and to add some specified levels of noise to networks. This allows the user to apply the techniques used to find relevant results on networks where there is less or no signal and thus were no interesting result should emerge.

NeAT programs are able to generate randomized networks according to three methods. 
\begin{enumerate}
 \item \textit{Node degree conservation} : this approach consists in shuffling the edges, each node keeping the same number of neighbors as in the original graph.
 \item \textit{Node degree distribution conservation} : in which the global distribution of the node degree is conserved but each node presents a different degree than in the original graph.
 \item \textit{Erdos-Renyi randomization} :  where edges are distributed between pairs of nodes with equal probability.
\end{enumerate}


\section{Comparison of the clustering results obtained with an expiremental network to those obtained with a random or an altered graph}
\subsection{Study case}

In this demonstration, we will use the approach developped in \cite{Brohee2006} where we evaluated the performances of different graph clustering algorithms. Network clustering algorithms allow to retrieve in a graph the groups of nodes that contain more connections between them than with the rest of the nodes of the graph. Clustering algorithms are often used in biology in order to extract coherent groups of nodes from networks (complexes detection (e.g. see \cite{Sharan2007,Krogan2006,Brohee2006,Pereira-Leal2004}), protein families detection \cite{Enright2002}, co-expressed genes detection in co-expression networks (e.g see \cite{Lattimore2005}), \ldots). Our web server contain the MCL clustering algorithm developped by Stijn van Dongen \cite{VanDongenPHD2000,Enright2002}. To follow the command-line tools instructions, you will have to install MCL from \url{http://micans.org/mcl/}.

We will use an artificial interaction network created from the complexes annotated in the MIPS database by creating an edge between all the nodes belonging to the same complex. \cite{Mewes2007}. We will then use the MCL clustering algorithm to this network, to a little altered network, to a highly altered network and finally to a randomized network. 

We will then compare these clusters to the MIPS complexes and estimate the Sensibility and the PPV of the different algorithms.

In this case, we only use random alteration, i.e., the edges that are removed are randomly chosen. This is done to mimick what happens really in biological experiments where some inter-relationships between may not be discovered or are erroneously discovered. However the \program{alter-graph} program also allows to alterate the network with targeted attack on important nodes for example. In their study, Spirin and Mirny \cite{Spirin2003} showed the affect of graph targeted attacks on clustering results.

\subsection{Protocol for the web server}

\subsubsection{Dataset download}
Go on the demo dataset download web page.\url{http://rsat.scmbb.ulb.ac.be/rsat/data/neat\_tuto\_data/} and download the MIPS complex network file (\file{complexes\_rm\_00\_ad\_00.tab}) and the complexes (\file{mips\_complexes.tab}).

\subsubsection{Network alteration}

\begin{enumerate}

\item In the \neat menu, select the command \program{network alteration}. 
\item In the \textit{Upload graph from file} text area, load the file \file{complexes\_rm\_00\_ad\_00.tab} containing the not altered network that you just downloaded.
\item In the \option{edges to add} text area, enter 10\%.
\item In the \option{edges to remove} text area, enter 10\%.
\item Click on the button \option{GO}. 
\item Right click on the resulting file and save with name \file{complexes\_rm\_10\_ad\_10.tab}.

Re-do the this alteration procedure using 50\% of edges removal and 100\% of edges addition. Save the resulting file with name \file{complexes\_rm\_50\_ad\_100.tab}.

\end{enumerate}

\subsubsection{Network randomization}

\begin{enumerate}

\item In the \neat menu, select the command \program{network randomization}. 
\item In the \textit{Upload graph from file} text area, load the file \file{complexes\_rm\_00\_ad\_00.tab}.
\item Select the \option{Node degree conservation} randomization type.
\item Click on the button \option{GO}. 
\item Right click on the resulting file and save with name \file{complexes\_rm\_00\_ad\_00\_random.tab}.

\end{enumerate}

\subsubsection{Networks clustering and clustering assessment}

\begin{enumerate}

\item In the \neat menu, select the command \program{graph-based clustering MCL}. 
\item In the \textit{Upload graph from file} text area, load the file \file{complexes\_rm\_00\_ad\_00.tab}.
\item Click on the button \option{GO}. You should now obtain a link to the clustering results and the distribution of the sizes of the different clusters.
\item In the \textit{Next step} pannel, click on the button \textit{Compare these clusters to other clusters}.
\item In the \textit{Upload reference classes from file } text area, load the \file{mips\_complexes.tab}.
\item Choose the \textit{matrix file} output format
\item Click on the button \option{GO}. You now obtain a contingency table, i.e, a table with $N$ rows and $M$ columns ($N$ being the number of MIPScomplexes and $M$, the number of clusters). Each cell contains the number of protein common to one complex and one cluster.
\item To calculate some statistics on this contingency table, click on the \option{contingency-table statistics} button in the \option{Next step} pannel.
\item The contingency-stats form appears. As the contingency table is already uploaded, click on the \option{GO} button. 
\item Save the resulting file under name  \file{contigency\_stats\_rm\_00\_ad\_00.tab}

\end{enumerate}

Repeat these steps for \file{complexes\_rm\_10\_ad\_10.tab}, \file{complexes\_rm\_50\_ad\_100.tab} and \file{complexes\_rm\_00\_ad\_00\_random.tab} and save the resulting files.


\subsection{Protocol for the command-line tools}

If you have installed a stand-alone version of the NeAT distribution,
you can use the programs \program{random-graph} and \program{alter-graph} on the
command-line. This requires to be familiar with the Unix shell
interface. If you don't have the stand-alone tools, you can skip this
section and read the next section (Interpretation of the results).

We will now describe the use of \program{random-graph} and \program{alter-graph} as a command line tool. For this tutorial, you need to have the MCL program installed. 

Start by going on the demo dataset download web page.\url{http://rsat.scmbb.ulb.ac.be/rsat/data/neat\_tuto\_data/} and downloading the MIPS complex network file (\file{complexes\_rm\_00\_ad\_00.tab}) and the complexes (\file{mips\_complexes.tab}).


\subsubsection{Network alteration}

\begin{enumerate}

\item Go in the directory where you downloaded the file.
\item Use the following commands to alter the graph
{\color{Blue} \begin{footnotesize} 
		\begin{verbatim}
			alter-graph	-v 1 -i complexes_rm_00_ad_00.tab \
					-rm_edges 10% -add_edges 10% \
					-o complexes_rm_10_ad_10.tab
		\end{verbatim} \end{footnotesize}
	}
Re-use this command, but modify the percentage of removed (-rm\_edges 50\%) and added edges (-add\_edges 100\%). Save the resulting file with name \file{complexes\_rm\_50\_ad\_100.tab}.
\end{enumerate}
\subsubsection{Network randomization}
\begin{enumerate}
\item Use the following commands to randomize the graph by shuffling the edges. The node degrees will be conserved.
{\color{Blue} \begin{footnotesize} 
		\begin{verbatim}
		random-graph 	-v 1 -i complexes_rm_00_ad_00.tab \
				-random_type node_degree \
				-o  complexes_rm_00_ad_00_random.tab
		\end{verbatim} \end{footnotesize}
	}
\end{enumerate}
\subsubsection{Networks clustering and clustering assessment}
\begin{enumerate}
  \item Use the following commands to apply MCL on the network
  {\color{Blue} \begin{footnotesize} 
		\begin{verbatim}
  		mcl 	complexes_rm_00_ad_00.tab \
  			--abc -I 1.8 -o complexes_rm_00_ad_00_clusters.mcl
  \end{verbatim} \end{footnotesize}}
  \item Convert the cluster file obtain with MCL with the program \program{convert-classes}
    {\color{Blue} \begin{footnotesize} 
		\begin{verbatim}
  		convert-classes	-i complexes_rm_00_ad_00_clusters.mcl 
  				-from mcl -to tab -o complexes_rm_00_ad_00_clusters.tab 
  \end{verbatim} \end{footnotesize}}
  \item Compare the obtained clusters to the MIPS complexes with the program \program{compare-classes}
    {\color{Blue} \begin{footnotesize} 
		\begin{verbatim}
  		compare-classes	-q complexes_rm_00_ad_00_clusters.tab \
  				-r mips_complexes.tab \
  				-matrix QR \
  				-o complexes_rm_00_ad_00_clusters_cc_complexes_matrix.tab 
  \end{verbatim} \end{footnotesize} } 
  \item Study the obtained matrix with the \program{contingency-stats}
    {\color{Blue} \begin{footnotesize} 
		\begin{verbatim}
		contingency-stats -i complexes_rm_00_ad_00_clusters_cc_complexes_matrix.tab \
				  -o contigency_stats_ad_00_rm_00.tab
  \end{verbatim} \end{footnotesize}  }  
\end{enumerate}

Repeat these steps for \file{complexes\_rm\_10\_ad\_10.tab}, \file{complexes\_rm\_50\_ad\_100.tab} and \file{complexes\_rm\_00\_ad\_00\_random.tab} and save the resulting files.


\subsection{Interpretation of the results}