
%%%%%%%%%%%%%%%%%%%%%%%%%%%%%%%%%%%%%%%%%%%%%%%%%%%%%%%%%%%%%%%%
%%%% Introduction
%%%%%%%%%%%%%%%%%%%%%%%%%%%%%%%%%%%%%%%%%%%%%%%%%%%%%%%%%%%%%%%%
\chapter{Introduction}

This tutorial aims at introducing how to use Regulatory Sequence
Analysis Tools (\RSAT) directly from the Unix shell.

\RSAT is a package combining a series of specialized programs for the
detection of regulatory signals in non-coding sequences. A variety of
tasks can be performed: retrieval of upstream or downstream sequences,
pattern discovery, pattern matching, graphical representation of
regulatory regions, sequence conversions, \ldots.

A web interface has been developed for the most common tools, and is
freely available for academic users.

\url{http://rsat.scmbb.ulb.ac.be/rsat/}

All the programs in \RSAT can also be used directly from the Unix
shell. The shell access is less intuitive than the web interface, but
it allows to perform more complex analyses, and it is very convenient
for automatizing repetitive tasks.

This tutorial was written by Jacques van Helden
(\email{Jacques.van.Helden@ulb.ac.be}).  Unless otherwise specified,
the programs presented here were written by Jacques van Helden.

%%%%%%%%%%%%%%%%%%%%%%%%%%%%%%%%%%%%%%%%%%%%%%%%%%%%%%%%%%%%%%%%
%% Prerequisites
%%%%%%%%%%%%%%%%%%%%%%%%%%%%%%%%%%%%%%%%%%%%%%%%%%%%%%%%%%%%%%%%
\section{Prerequisites}

This program requires a basic knowledge of the Unix
environment. Before starting you should be familiar with the concepts
of Unix shell, directory, file, path.

\section{Creating a directory for this tutorial}

During this tutorial, we wll frequently save data and result files. I
propose to create a dedicated directory for these files. In the
following chapters, we will assume that this directory is named
\file{practical\_rsat} and is located at the root of your personnal
account (everyone is of course allowed to change the name and location
of this directory).

To create the directory for the tutorials, you can simply type the
following commands.

{\color{Blue} \begin{footnotesize}
\begin{verbatim}
cd $HOME ## Go to your home directory
mkdir -p practical_rsat ## Create the directory for the tutorial
cd  practical_rsat ## Go to this directory
pwd ## Check the path of your directory
\end{verbatim} \end{footnotesize}
}

From now on, we will assume that all the exercises are executed
from this directory.

%%%%%%%%%%%%%%%%%%%%%%%%%%%%%%%%%%%%%%%%%%%%%%%%%%%%%%%%%%%%%%%%
%%%% Wargning
%%%%%%%%%%%%%%%%%%%%%%%%%%%%%%%%%%%%%%%%%%%%%%%%%%%%%%%%%%%%%%%%

\section{Warning}

This tutorial is under construction. Some sections are still to be
written, and only appear as a title without any further text. The
tutorial will be progressively completed. We provided it as it is.
