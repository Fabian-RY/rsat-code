%%%%%%%%%%%%%%%%%%%%%%%%%%%%%%%%%%%%%%%%%%%%%%%%%%%%%%%%%%%%%%%%
%%%% MULTIPLE FAMILY ANALYSIS
%%%%%%%%%%%%%%%%%%%%%%%%%%%%%%%%%%%%%%%%%%%%%%%%%%%%%%%%%%%%%%%%
\section{Automatized analysis of multiple gene clusters}

The main interest of using \RSAT from the shell is that it allows to
automatize the analysis of multiple data sets. The different programs
of the package can be combined in different ways to apply an extensive
analysis of your data. A typical example is the analysis of clusters
obtained from gene expression data.

When a few tens or hundreds of gene clusters have to be anayzed, it
becomes impossible to manage it manually. \RSAT includes a program,
\program{multiple-family-analysis}, which takes as input a file with
the composition of gene clusters (the \concept{cluster file}), and
automatically performs the following analyses on each cluster :

\begin{description}
\item[directory management] the results are stored in a separate
directory for each cluster. Directories are automatically created
during the execution, and bear the name of the cluster.

\item[sequence retrieval] upstream sequences are retrieved and stored
in fasta format

\item[oligonucleotide analysis] \program{oligo-analysis} is used to
detect over-represented oligonucleotides. \program{dna-pattern} and
\program{feature-map} are used to draw a feature map of the
significant patterns.

\item[dyad analysis] \program{oligo-analysis} is used to
detect over-represented oligonucleotides. \program{dna-pattern} and
\program{feature-map} are used to draw a feature map of the
significant patterns.

\end{description}

In addition to this cluster-per-cluster analysis, results are
summarized in two format.

\begin{description}
\item[synthetic table] A HTML table is generated with one row per
cluster, and a summary of the results (gene composition, significant
oligonucleotides, significant dyads). This table contains links to the
feature maps, making it easy to browse the results.

\item[sql table] The list of significant patterns detected in all the
cluster are compiled in a single result table (a tab-delimited text
file), with one row per pattern and cluster, and one column per
criterion (pattern type, occurrences, significance, ...). 

 The program also automatically exports SQL scripts which allow to
create the appropriate table and load the data in a relational
database management system (RDBMS).
\end{description}

\subsection{Input format}

The input format is a tab-delimited text file with two columns, providing respectively :
\begin{enumerate}
\item gene identifier or name
\item cluster name
\end{enumerate}

An example of cluster file is displayed in Table
\ref{table:cluster_file}. This file describes 3 yeast regulons, each
responding to some specific environmental condition: the NIT family
contains 7 genes expressed under nitrogen depletion, the PHO fmily 5
genes expressed under phosphate stress, and the MET family 11 genes
expressed when methionine is absent fom the culture medium.


\begin{table}
\begin{center}
\begin{tabular}{ll}
\hline
; gene & cluster \\
YGR121C & NIT \\
YJR152W & NIT \\
YKR034W & NIT \\
YKR039W & NIT \\
YNL142W & NIT \\
YOR348C & NIT \\
YPR138C & NIT \\
YAR071W & PHO \\
YBR093C & PHO \\
YDR481C & PHO \\
YGR233C & PHO \\
YML123C & PHO \\
YDR502C & MET \\
YER091C & MET \\
YHL036W & MET \\
YIL046W & MET \\
YJR010W & MET \\
YKL001C & MET \\
YKR069W & MET \\
YLR180W & MET \\
YLR303W & MET \\
YNL241C & MET \\
YNL277W & MET \\
\hline
\end{tabular}
\end{center}
\caption{\label{table:cluster_file} Example of cluster file.}
\end{table}



\subsection{Example of utilization}

Let us assume that the file displayed in Table
\ref{table:cluster_file} has been saved under the name
\file{test.fam}. The following command will automatically perform all
the analyses. 

\begin{verbatim}
multiple-family-analysis  -i test.fam -v 1 \
    -org Saccharomyces_cerevisiae \
    -2str -noorf -noov \
    -task upstream,oligos,maps,synthesis,sql,clean,all \
    -outdir test_fam_results
\end{verbatim}

Once the analysis is finished, you can open the folder
\file{synthetic\_tables} with aweb browser and follow the links.

\subsection{Loading the results in arelational database}

The results were exported in tab-delimited text files in the directory
\file{test\_fam\_results/sql\_export/}. This directory contains 3 files
and one subdirectory :

\begin{verbatim}
Family_genes.tab
Family.tab
Pattern.tab
sql_scripts/
\end{verbatim}

The subdirectory \file{sql\_scripts} contains several SQL scripts for crating
tables in a relational database management system (\concept{RDBMS}),
loading data into these tables, and dropping these tables when you
don't need them anymore.

\begin{verbatim}
family_genes_table_load.ctl
family.mk
family_table_create.sql
family_table_drop.sql
family_table_load.ctl
makefile
pattern.mk
pattern_table_create.sql
pattern_table_drop.sql
pattern_table_load.ctl
\end{verbatim}

The file \file{makefile} allows you to automatically create the tables
and load the data in two operations.

\begin{verbatim}
make create MYSQL='mysql -u [your login] -D multifam'
make load MYSQL='mysql -u [your login] -D multifam'
\end{verbatim}

This requires the existence of a database space 'multifam' in your
\concept{RDBMS}. If you are not familar with relational databases, you
probably need to contact your system administrator to create this
space for you.
