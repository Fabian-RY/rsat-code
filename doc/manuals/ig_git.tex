\section{Downloading \RSAT from the CVS server}

\textbf{Warning:} the \program{git} distribution is reserved to the
members of the \RSAT team. 

For external users, \RSAT is distributed as a tar archive (file
\file{rsat\_YYYY-MM-DD.tar.gz}, where YYYY-MM-DD indicates the
release date). If you are not a member of the \RSAT team you can thus
skip this section.

\subsection{Obtaining a first version of \RSAT sowftare distribution}

The following command should be used the first time you retrieve the
tools from the server (you need to replace \texttt{[mylogin]} by the
login name you received when signing the \RSAT license).

\begin{lstlisting}
git clone git@depot.biologie.ens.fr:rsat
\end{lstlisting}

This will create a directory \file{rsat} on your computer, and store
the programs in it. Note that at this stage the programs are not yet
functional, because you still need to configure \RSAT and install
genomes.

Beware: the git distribution takes some space (>700Mb in August 2014),
and the majority of this disk space is mobilized by the .git file
(469Mb in August 2014). 

\subsection{Updating \RSAT programs}

Once the package has been cloned, you can obtain updates very
easily. For this, you need to change your directory to the rsat
directory, and use the command \texttt{git pull}.

\begin{lstlisting}
cd rsat
git pull
\end{lstlisting}


