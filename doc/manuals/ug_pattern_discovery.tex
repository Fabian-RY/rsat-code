%%%%%%%%%%%%%%%%%%%%%%%%%%%%%%%%%%%%%%%%%%%%%%%%%%%%%%%%%%%%%%%%
%%%% PATTERN DISCOVERY
%%%%%%%%%%%%%%%%%%%%%%%%%%%%%%%%%%%%%%%%%%%%%%%%%%%%%%%%%%%%%%%%
\section{Pattern discovery}

In a pattern discovery problem, you start from a set of
functionally related sequences (e.g.  upstream sequences for a set of
co-regulated genes) and you try to extract motifs (e.g. regulatory
elements) that are characteristic of these sequences.

Several approaches exist, either string-based or matrix-based. For
yeast regulatory elements, string-based approaches give excellent
results. The advantages:

\begin{itemize}
\item Simple to use
\item Deterministic (if you run it repeatedly, you always get the same result)
\item Easily parametrizable
\item Easy to interpret
\item Fast
\item Able to return a negative answer: if no motif is significant,
the programs return an empty list of motifs. This is particularly
important to reduce the rate of false positive.
\end{itemize}

Matrix-based approach can provide a more refined description of motifs
presenting a high degree of degeneracy. The problem of matrix-based
approaches is that it is impossible to analyze all possible
position-weight matrices, and thus one has to use heuristics. There is
thus a risk to miss the global optimum because the program is
attracted to local maxima. Another problem is that there are more
parameters to select (typically, matrix width and expected number of
occurrences of the motif), and their choice drastically affects the
quality of the result. Last problem: the result is not easily
interpretable because the programs always return an answer.

Basically, I would tend to prefer string-based approaches for any
problem of pattern discovery. On the contrary, matrix-based approaches
are much more sensitive for pattern matching problems (see below). The
ideal would thus be to combine string-based pattern discovery and
matrix-based pattern matching.

\subsection{Requirements}
This part of the tutorial assumes that you already performed the
tutorial about sequence retrieval (above), and that you have the
result files in the current directory. Check with the command:

\begin{verbatim}
ls -1
\end{verbatim}

You should see the following file list:
\begin{verbatim}
PHO_genes.txt
PHO_up800.fasta
Escherichia_coli_K12_start_codons.wc
Escherichia_coli_K12_stop_codons.wc
\end{verbatim}

\subsection{oligo-analysis}

The program \texttt{oligo-analysis} is the simplest pattern discovery
program. It counts the number of occurrences of all oligonucleotides
(word) of a given length (typically 6), and compares, for each word,
the observed and expected occurrences, and return words with a
significant level of over-representation.  

Despite its simplicity, this program already returns good results for
many families of co-regulated genes in yeast.

For a first trial, we will simply use the program to count word
occurrences. The application will be to check the start and stop
codons retrieved above.

We will then use \texttt{oligo-analysis} in a pattern discovery
process, to detect over-represented words from the set of 5 upstream
sequences retrieved above (the PHO family).  In a first time, we will
use the appropriate parameters, which have been optimized for pattern
discovery in yeast upstream sequences (van Helden et al., 1998). We
will then use the sub-optimal settings to illustrate the fact that the
success of word-based pattern-discovery crucially depends on a
rigorous statistical approach.

\subsubsection{Counting word occurrences and frequencies}

Try the following command:

\begin{verbatim}
oligo-analysis -i Escherichia_coli_K12_start_codons.wc \
    -format wc -l 3 -1str
\end{verbatim}

Call the on-line option description to understand the meaning of the options you used:
\begin{verbatim}
oligo-analysis -help
\end{verbatim}

Or, to obtain more details:
\begin{verbatim}
oligo-analysis -h
\end{verbatim}


You can also ask some more information (verbose) and store the result
in a file:

\begin{verbatim}
oligo-analysis -i Escherichia_coli_K12_start_codons.wc \
    -format wc -l 3 -1str \
    -return occ,freq -v \
    -o Escherichia_coli_K12_start_codon_frequencies
\end{verbatim}

Read the result file:

\begin{verbatim}
more Escherichia_coli_K12_start_codon_frequencies
\end{verbatim}

Note the effect of the verbose. You receive information about sequence
length, number of possible oligonucleotides, the content of the output
columns, ...

\textbf{Exercise:} check the frequencies of \textit{E.coli} stop codons.

\subsubsection{Pattern discovery in yeast upstream regions}

Try the following command:

\begin{verbatim} 
oligo-analysis -i PHO_up800.fasta -format fasta      \
    -v -l 6 -2str                                    \
    -return occ,proba -lth occ_sig 0 -bg upstream \
    -org Saccharomyces_cerevisiae -sort              \
    -o PHO_up800_6nt_2str_ncf_sig0 
\end{verbatim}

Note that the return fields (``occ'', and ``proba'') are separated by
a comma \textit{without} space.

Call the on-line help to understand the meaning of the parameters.

\begin{verbatim} 
oligo-analysis -h
\end{verbatim}

Note that we used pre-calibrated tables as estimators of expected word
frequencies. These tables have been previously calculated (with
oligo-analysis) by counting hexanucleotide frequencies in the whole
set of yeast upstream regions. Our experience is that these
frequencies are the optimal estimator for discovering regulatory
elements in upstream sequences of co-regulated genes.

Look the result file:

\begin{verbatim}
more PHO_up800_6nt_2str_ncf_sig0
\end{verbatim}

A few questions:
\begin{enumerate}
\item How many hexanucleotides can be formed with the 4-letter alphabet A,T,G,C ?
\item How many possible oligonucleotides are indicated ? Is it the number you would expect ? Why ?
\item How many patterns have been selected as significant ?
\item Do you see some similarity between some of the selected patterns ?
\end{enumerate}

\subsubsection{Answers}

\begin{enumerate}
\item $4^6=4,096$
\item $2,080$. This is due to the fact that the analysis was performed on
both strands. Each oligonucleotide is thus equivalent to its reverse
complement.
\item $9$
\item some pairs of words are mutually overlapping
  (e.g. \texttt{cACGTG} and \texttt{ACGTGc}).
\end{enumerate}

\subsubsection{Assembling the patterns}

A separate program, \texttt{pattern-assembly}, allows to assemble a
list of patterns, in order to group those that overlap mutually. Try:

\begin{verbatim}
pattern-assembly -i PHO_up800_6nt_2str_ncf_sig0 \
    -v -sc 7 -subst 1 \
    -2str -o PHO_up800_6nt_2str_ncf_sig0.assemb
\end{verbatim}


Call the on-line help to have a look at the assembly parameters. 
\begin{verbatim}
pattern-assembly -h
\end{verbatim}

Look at the result. There are two alignments (with two contigs), and
two isolated patterns. Each alignment is made of strongly overlapping
patterns. The first alignment (cgcacgtgcg) corresponds to the high
affinity binding site for Pho4p, the protein controlling
transcriptional response to Phosphate in yeast. the second alignment
(cgcacgttt) corresponds to the medium affinity binding site for
Pho4p. Medium affinity binding sites have been shown to participate in
the transcriptional response to some PHO genes.

\begin{verbatim}
more PHO_up800_6nt_2str_ncf_sig0.assemb
\end{verbatim}

\subsubsection{Suboptimal settings}

This chapter only aims at emphasizing how crucial is the choice of
appropriate statistical parameters. We saw above that the optimal
parameters give good results with the PHO family: despite the
simplicity of the algorithm (counting non-degenerate hexanucleotide
occurrences), we were able to extract a description of the regulatory
motif over a larger width than 6 (by pattern assembly), and we got
some description of the degeneracy (the high and low affinity sites).

We will now intentionally try other parameter settings and see how
they affect the quality of the results.

\textit{\textbf{Equiprobable oligonucleotides}}

Let us try the simplest approach: each word is considered
equiprobable. For this, we simply suppress the options
\texttt{-bg upstream -org yeast} fom the above commands. We
also omit to specify the output file, so results will immediately
appear on the screen.

\begin{verbatim} 
oligo-analysis -i PHO_up800.fasta -format fasta \ 
    -v -l 6 -2str \
    -return occ,proba -lth occ_sig 0 -sort 
\end{verbatim}

Note that
\begin{itemize} 
\item The number of selected motifs is higher (27) than in the previous trial
\item The most significant motifs have nothing to do with Pho4p binding
sites. All these false positives are A-rich motifs (or T-rich, since we
are grouping patterns with their reverse-complement).
\item Two patterns (\texttt{acgttt} and \texttt{acgtgc}) are selected
which are related to Pho4p binding site. However, they come at the
12th and 14th positions only.
\end{itemize}

You can combine oligo-analysis and pattern-assembly in a single
command, by using the pipe character as below.

\begin{verbatim}
oligo-analysis -i PHO_up800.fasta -format fasta -v \
    -l 6 -2str  -return occ,proba -lth occ_sig 0 -sort \
    | pattern-assembly -2str -sc 7 -subst 1 -v
\end{verbatim}

On unix systems, this special character is used to concatenate
commands, i.e. the output of the first command (in this case
oligo-analysis) is not printed to the screen, but is sent as input for
the second command (in this case pattern-assembly).

Note that the most significant patterns are associated to the poly-A
(aaaaaa) contig. The true positive come isolated. Due to the bad
choice of expected frequencies (all hexanucleotides were considered
equiprobable here), regulatory sites were lost within a majority of
false positive, and their description is much less accurate than with
the option \texttt{-bg upstream}.

\textit{\textbf{Markov chains}}

Another possibility is to use Markov chain models to estimate expected
word frequencies. Try the following commands and compare the
results. None is as good as the option \texttt{-bg upstream},
but in case one would not have the pre-calibrated non-coding
frequencies (for instance if the organism has not been completely
sequenced), markov chains can provide an interesting approach.

\begin{verbatim}
oligo-analysis -markov 0 \
    -i PHO_up800.fasta -format fasta \
    -l 6 -lth occ_sig 0 -sort \
    -2str -return occ,proba \
    | pattern-assembly -2str -sc 7 -subst 1 -v

oligo-analysis -markov 1 \
    -i PHO_up800.fasta -format fasta \
    -2str -return occ,proba \
    -l 6 -lth occ_sig 0 -sort \
    | pattern-assembly -2str -sc 7 -subst 1 -v
	
oligo-analysis -markov 2 \
    -i PHO_up800.fasta -format fasta \
    -2str -return occ,proba \
    -l 6 -lth occ_sig 0 -sort \
    | pattern-assembly -2str -sc 7 -subst 1 -v
	
oligo-analysis -markov 3 \
    -i PHO_up800.fasta -format fasta \
    -2str -return occ,proba \
    -l 6 -lth occ_sig 0 -sort \
    | pattern-assembly -2str -sc 7 -subst 1 -v
	
oligo-analysis -markov 4 \
    -i PHO_up800.fasta -format fasta \
    -2str -return occ,proba \
    -l 6 -lth occ_sig 0 -sort \
    | pattern-assembly -2str -sc 7 -subst 1 -v
\end{verbatim}

\textit{\textbf{Remarks}}
\begin{itemize}
\item 
Markov 0 returns AT-rich patterns with the highest significance, but
the Pho4p high affinity site is described with a good accuracy. The
medium affinity site appears as a single word (acgttt) in the isolated
patterns.
\item 
Markov order 1 returns less AT-rich motifs. The poly-A (aaaaaa) is
however still associated with the highest significance, but comes as
isolated pattern.
\item 
The higher the order of the markov chain, the most stringent are the
conditions. For small sequence sets, selecting a too high order
prevents from selecting any pattern. Most of the patterns are missed
with a Markov chain of order 2, and higher orders don't return any
single significant word.
\end{itemize}

\subsection{dyad-analysis}



\subsection{gibbs motif sampler (program developed by Andrew Neuwald)}




\subsection{consensus (program developed by Jerry Hertz)}

An alternative approach for matrix-based pattern discovery is
\textit{consensus}, a program written by Jerry hertz, an based on a
greedy algorithm. We will see how to extract a profile matrix from
upstream regions of the PHO genes.

\subsubsection{Getting help}

As for RSAT programs, there are two ways to get help from Jerry Hertz'
proigrams: a detailed manual can be obtained with the option
\texttt{-h}, and a summary of options with \texttt{-help}. Try these
options and read the manual.

\begin{verbatim}
consensus -h
consensus -help
\end{verbatim}

\subsubsection{Sequence conversion}


\textit{consensus} uses a custom sequence format. Fortunately, the RSAT
package contains a sequence conversion program (\textit{convert-seq})
which supports Jerry Hertz' format. We will thus start by converting
the fasta sequences in this format. 

\begin{verbatim}
convert-seq -i PHO_up800.fasta -from fasta -to wc \ 
    -o PHO_up800.wc
\end{verbatim}

\subsubsection{Running consensus}

Using consensus requires to choose the appropriate value for a series
of parameters. We found the following combination of parameters quite
efficient for discovering patterns in yeast upstream sequences.

\begin{verbatim}
consensus -L 10 -f PHO_up800.wc -A a:t c:g -c2 -N 10
\end{verbatim}

The two main options of this command are 

\begin{description}
\item[-L 10] we guess that the pattern has a length of about 10 bp

\item[-N 10] we expect about 10 occurrences in the sequence set. Since
there are 5 genes in the family, this means that we expect on average
2 regulatory sites per gene, which is generally a good guess for
yeast.

\end{description}

Notice that several matrices are returned. Each matrix is followed by
the alignment of the sites on which it is based. Notice that the 4
matrices are highly similar, basically they are all made of several
occurrences of the high afinity site CACGTG, and matrices 1 and 3
contain one occurrence of the medium affinity site CACGTT. 

Also notice that these matrices are not made of exactly 10 sites
each. \textit{consensus} is able to adapt the number of sites in the
alignment in order to get the highest information content. The option
\texttt{-N 10} was an indication rather than a rigid requirement.

To save the result in a file, you can use the symbol ``greater than''
($>$) which redirects the output of a program to a file.

\begin{verbatim}
consensus -L 10 -f PHO_up800.wc -A a:t c:g -c2 -N 10 \
    > PHO_consensus
\end{verbatim}

(this may take a few minutes)

Once the task is achieved, check the result.

\begin{verbatim}
more PHO_consensus
\end{verbatim}
